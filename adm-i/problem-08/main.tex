\documentclass[9pt]{extarticle}

\usepackage{amsmath, amsthm, amssymb}
\usepackage{mathtools}
\usepackage[many]{tcolorbox}
\usepackage{xcolor}
\usepackage{titlesec}
\usepackage{titling}
\usepackage{enumitem}   
\usepackage{physics}
\usepackage{xfrac,unicode-math}
\usepackage{geometry}
\usepackage{array}   % for \newcolumntype macro
\usepackage{nicefrac}

\usepackage{tikz}
\newcommand*\circled[1]{\tikz[baseline=(char.base)]{
            \node[shape=circle,draw,inner sep=1pt] (char) {#1};}}

\geometry{a4paper,left=40mm,right=40mm}

% -----------------------
% Font configuration
% -----------------------
\usepackage{fontspec}

\setmainfont[
    Path=fonts/
]{NewYorkMedium-Regular.otf}
\setmainfont[
    Path = fonts/,
    ItalicFont={NewYorkMedium-RegularItalic.otf},
    BoldItalicFont={NewYorkMedium-BoldItalic.otf},
    BoldFont={NewYorkMedium-Bold.otf}
]{NewYorkMedium-Regular.otf}
\setsansfont[
    Path = fonts/,
    BoldFont={SF-Pro-Display-Bold.otf}
]{SF-Pro-Display-Medium.otf}

\setmathfont{STIX2Math}[Path=fonts/]
\definecolor{grey}{rgb}{0.5,0.5,0.5}
\definecolor{lightgrey}{rgb}{0.8,0.8,0.8}
\definecolor{darkgrey}{rgb}{0.3,0.3,0.3}
\definecolor{orange}{rgb}{0.94, 0.55, 0.294}
\definecolor{pink}{rgb}{0.94, 0.29, 0.7}
\definecolor{yellow}{rgb}{1, 0.749, 0}
\definecolor{green}{rgb}{0.235,0.702,0.443}

\newcommand{\chapfnt}{\fontsize{16}{19}}
\newcommand{\secfnt}{\fontsize{18}{17}}
\newcommand{\ssecfnt}{\fontsize{14}{14}}

\renewcommand{\hline}{\noindent\makebox[\linewidth]{\rule{12cm}{1pt}}}

\titleformat{\chapter}[display]
{\normalfont\chapfnt\bfseries}{\chaptertitlename\ \thechapter}{20pt}{\chapfnt}

\titleformat{\section}
{\normalfont\sffamily\secfnt\bfseries}{\thesection}{}{}

\titleformat{\subsection}
{\normalfont\sffamily\ssecfnt\mdseries}{\thesubsection}{}{}

\titleformat{\subsubsection}
{\normalfont\sffamily\ssecfnt\mdseries\color{grey}}{\thesubsection}{}{}

\titlespacing*{\chapter} {0pt}{50pt}{40pt}
\titlespacing*{\section} {0pt}{0pt}{8pt}
\titlespacing*{\subsection} {0pt}{12pt}{8pt}

\linespread{1.3}

\renewcommand{\baselinestretch}{1.4} 
\makeatletter
\let\old@rule\@rule
\def\@rule[#1]#2#3{\textcolor{lightgrey}{\old@rule[#1]{#2}{#3}}}
\makeatother
\newcommand{\vip}[1]{\textit{\textbf{#1}}}
\newcommand{\R}{\mathbb{R}} % Reelle Zahlen
\newcommand{\N}{\mathbb{N}} % Natürliche Zahlen
\newcommand{\Z}{\mathbb{Z}} % Ganze Zahlen
\newcommand{\C}{\mathbb{C}} % Komplexe Zahlen
\newcommand{\Q}{\mathbb{Q}} % Rationale Zahlen
\newcommand{\K}{\mathbb{K}}
\DeclareMathOperator{\spn}{span}
\DeclareMathOperator{\ran}{ran}
% -----------------------
% Font configuration
% -----------------------


% -----------------------
% Font configuration
% -----------------------
\newtcbtheorem[auto counter,number within=section]{theorem}%
  {Theorem}{
  		fonttitle=\upshape, 
  		fontupper=\upshape,
  		boxrule=0pt,
  		leftrule=3pt,
  		arc=0pt,auto outer arc,
  		colback=white,
  		colframe=pink,
  		colbacktitle=white,
  		coltitle=pink,
  		oversize,
  		enlarge top by=1mm,
  		enlarge bottom by=1mm,
    	enhanced jigsaw,
    	interior hidden, 
    	before skip=12pt,
    	overlay={
    		\draw[line width=1.5pt,pink] (frame.north west) -- (frame.south west);
  		}, 
  		frame hidden}{theorem}
  		
\newtcbtheorem[]{proposition}%
  {Proposition}{
        theorem name,
  		fonttitle=\upshape, 
  		fontupper=\upshape,
  		boxrule=0pt,
  		leftrule=3pt,
  		arc=0pt,auto outer arc,
  		colback=white,
  		colframe=pink,
  		colbacktitle=white,
  		coltitle=pink,
  		oversize,
  		enlarge top by=1mm,
  		enlarge bottom by=1mm,
    	enhanced jigsaw,
    	interior hidden, 
    	before skip=12pt,
    	after skip=0pt,
    	overlay={
    		\draw[line width=1.5pt,pink] (frame.north west) -- (frame.south west);
  		}, 
  		frame hidden}{proposition}

\newtcbtheorem[auto counter,number within=section]{lemma}%
  {Lemma}{
  		fonttitle=\upshape, 
  		fontupper=\upshape,
  		boxrule=1pt,
  		toprule=0pt,
  		leftrule=3pt,
  		arc=0pt,auto outer arc,
  		colback=white,
  		colframe=orange,
  		colbacktitle=white,
  		coltitle=orange,
  		oversize,
  		enlarge top by=1mm,
  		enlarge bottom by=1mm,
    	enhanced jigsaw,
    	interior hidden, 
    	before skip=12pt,
    	after skip=0pt,
    	overlay={
    		\draw[line width=1.5pt,orange] (frame.north west) -- (frame.south west);
  		}, 
  		frame hidden}{lemma}
  		
 \newtcbtheorem[auto counter,number within=section]{definition}%
  {Definition}{
  		fonttitle=\upshape, 
  		fontupper=\upshape,
  		boxrule=1pt,
  		toprule=0pt,
  		leftrule=3pt,
  		arc=0pt,auto outer arc,
  		colback=white,
  		colframe=orange,
  		colbacktitle=white,
  		coltitle=orange,
  		oversize,
  		enlarge top by=1mm,
  		enlarge bottom by=1mm,
    	enhanced jigsaw,
    	interior hidden, 
    	before skip=12pt,
    	overlay={
    		\draw[line width=1.5pt,orange] (frame.north west) -- (frame.south west);
  		}, 
  		frame hidden}{definition}
  		
  		 \newtcbtheorem[]{goal}%
  {Goal}{
  		theorem name,
  		fonttitle=\upshape, 
  		fontupper=\upshape,
  		boxrule=1pt,
  		toprule=0pt,
  		leftrule=3pt,
  		arc=0pt,auto outer arc,
  		colback=white,
  		colframe=orange,
  		colbacktitle=white,
  		coltitle=orange,
  		oversize,
  		enlarge top by=1mm,
  		enlarge bottom by=1mm,
    	enhanced jigsaw,
    	interior hidden, 
    	before skip=12pt,
    	overlay={
    		\draw[line width=1.5pt,orange] (frame.north west) -- (frame.south west);
  		}, 
  		frame hidden}{goal}
  		
\newtcbtheorem[]{important}%
  {Wichtig}{
  		fonttitle=\upshape, 
  		fontupper=\upshape,
  		boxrule=0pt,
  		leftrule=3pt,
  		arc=0pt,auto outer arc,
  		colback=white,
  		colframe=pink,
  		colbacktitle=white,
  		coltitle=pink,
  		oversize,
  		enlarge top by=1mm,
  		enlarge bottom by=1mm,
    	enhanced jigsaw,
    	interior hidden, 
    	before skip=12pt,
    	overlay={
    		\draw[line width=1.5pt,pink] (frame.north west) -- (frame.south west);
  		}, 
  		frame hidden}{important}
  		
  		
  		\newtcbtheorem[]{explanation}%
  {Explanation}{
        theorem name,
  		fonttitle=\upshape, 
  		fontupper=\upshape,
  		boxrule=0pt,
  		leftrule=3pt,
  		arc=0pt,auto outer arc,
  		colback=white,
  		colframe=green,
  		colbacktitle=white,
  		coltitle=green,
  		oversize,
  		enlarge top by=1mm,
  		enlarge bottom by=1mm,
    	enhanced jigsaw,
    	interior hidden, 
    	before skip=12pt,
    	after skip=12pt,
    	overlay={
    		\draw[line width=1.5pt,green] (frame.north west) -- (frame.south west);
  		}, 
  		frame hidden}{explanation}
  		
  		
  		\newtcbtheorem[]{congrats}%
  {We used an assumption!}{
        theorem name,
  		fonttitle=\upshape, 
  		fontupper=\upshape,
  		boxrule=0pt,
  		leftrule=3pt,
  		arc=0pt,auto outer arc,
  		colback=white,
  		colframe=green,
  		colbacktitle=white,
  		coltitle=green,
  		oversize,
  		enlarge top by=1mm,
  		enlarge bottom by=1mm,
    	enhanced jigsaw,
    	interior hidden, 
    	before skip=12pt,
    	after skip=12pt,
    	overlay={
    		\draw[line width=1.5pt,green] (frame.north west) -- (frame.south west);
  		}, 
  		frame hidden}{congrats}
    	


\begin{document}
\newcolumntype{L}{>{$}l<{$}} % math-mode version of "l" column type

\section*{ADM I Problem Sheet 08}
\subsubsection*{Team 42 Linear and Combinatorial Optimization Workgroup}

\textbf{Affiliated Members:} Jacky Behrendt (391049) and Viet Duc Nguyen (395220)

\noindent \textbf{Tutorial:} Thursday 10am - 12pm, Christos
\\

\noindent $
\begin{array}{l}
\includegraphics[width=0.5cm]{instagram.png}
\end{array}
$ Follow Team 42 on \vip{instagram.com/official\_team42} 


\sep
\subsection*{Exercise 22 (Helly’s Theorem for polyhedra)}
\begin{enumerate}[label=(\roman*)]
    \item Let $\mathcal F = \{ P_1, P_2, P_3 \}$ with 
    \begin{itemize}
        \item $P_1 = \{ (x,y) \in \mathbb R^2 : y \geq 0 \}$,
        \item $P_2 = \{ (x,y) \in \mathbb R^2 : x \geq 0 \}$, and
        \item $P_3 = \{ (x,y) \in \mathbb R^2 : -(x+y) \geq 4 \}$.
    \end{itemize}
    
    We see that $(0,0) \in P_1 \cap P_2$, $(-4,0) \in P_1 \cap P_3$ and $(0,-4) \in P_2 \cap P_3$. So, \textbf{every polyhedron in $\mathcal F$ has a point in common.}
    
    However, there is no point in $P_1 \cap P_2 \cap P_3$. If this were the case, one would obtain a point $(x,y) \in \mathbb R^2_{\geq 0}$, but then $-(x+y) \leq 0$, which contradicts the inequality of $P_3$. \textbf{Therefore, all polyhedra in $\mathcal F$ share no point.} $\qed$

    \item Sources: http://www.wisdom.weizmann.ac.il/~robi/teaching/AdvAlgs2008/handout4.pdf and https://math.stackexchange.com/questions/85560/proving-hellys-theorem
    
    Let $A = (a_i^T)_{i=1,...,m} \in \mathbb R^{m \times n}$, and let $b \in \mathbb R^m$. Assume $Ax \geq b$ has no solutions. Take a look at the following linear program:
    \begin{align}\label{papa}
        \min \quad &0^Tx \\
        \text{s.t.} \quad &Ax \geq b \nonumber
    \end{align}
    For $Ax \geq b$ has no solutions, it follows that the dual program
    \begin{alignat}{2}\label{hans}
        \max \quad &p^Tb&  \\
        \text{s.t.} \quad & A^Tp &= 0 \nonumber \\
        & p &\geq 0 \nonumber 
    \end{alignat}
    is either \emph{unbounded} or \emph{infeasible}. Since $p=0$ is a solution of the dual program \eqref{hans}, \textbf{the dual program must be unbounded.}

    Next, due to the simplex algorithm and the fact that \eqref{hans} is \emph{unbounded} it follows that there exists a basic variables $p_{B(1)},...,p_{B(n)}$ of \eqref{hans} and an non-basic index $j \notin \{B(i)\}_{i=1,...,n}$ such that it holds $B^{-1}A^T_j \leq 0$, where $B$ is the associated basis matrix of indexes $B(1),...,B(n)$, i.e. $B=(A^T_{B(1)},...,A^T_{B(n)})$. So, one obtains a feasible solution $p \in \R^m$ which satisfies 
    \begin{itemize}
        \item $A^Tp = 0$ and $p \geq 0$,
        \item $-p^T b < 0$, and
        \item $p_{i} = 0$ for all non-basic indexes $i \notin \{ B(k) \}_{k=1,...,n} \cup \{ j \}$.
    \end{itemize}
    In other words, it holds
    \[
        Bp_{B'} = 0 \quad \text{and} \quad p_{B'}^T b_{B'} < 0,
    \]
    where $B'$ denotes the associated basis indexes $B(1),...,B(n)$ and $j$, as well. Since the variable $p_j$ can be increased arbitrarily the cost $p_{B'}^T b_{B'}$ is unbounded. So, we obtain the following \emph{unbounded} linear program
    \begin{alignat*}{2}
        \max \quad &p_{B'}^Tb_{B'}&  \\
        \text{s.t.} \quad & Bp_{B'} &= 0 \nonumber \\
        & p_{B'} &\geq 0 \nonumber 
    \end{alignat*}
    The dual of this program results in a modified primal linear program where only $n+1$ rows of $A$ have been selected. This modified primal linear program is \textbf{infeasible} for the dual is unbounded. That's exactly what we wanted to prove. $\qed$

    \item \textbf{Claim:} If for all $F \subset \mathcal F$ with $|F|=n+1$ it holds $\bigcap_{P \in F} P \neq \emptyset$, it follows that $\bigcap_{P \in \mathcal F} P \neq \emptyset$.

    \textbf{Proof by contraposition:} Let $\mathcal F$ be a family of polyhedra such that $\bigcap_{P \in \mathcal F} P = \emptyset$. From exercise 1(ii) there exists a family $F \subset \mathcal F$ with 
    \begin{itemize}
        \item $|F|=n+1$, and
        \item $\bigcap_{P \in F} P = \emptyset$.
    \end{itemize}
    This ends the proof since the claim follows by contraposition. $\qed$
\end{enumerate}
\end{document}