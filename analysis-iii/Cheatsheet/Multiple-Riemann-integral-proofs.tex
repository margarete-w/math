\documentclass[a4paper, 11pt]{article}

\usepackage[utf8]{inputenc}
\usepackage{amsmath,amsthm,amssymb}
\usepackage{mathtools}
\usepackage{geometry} 
\usepackage{marvosym}
\usepackage[toc,titletoc,title]{appendix}
\usepackage[hidelinks]{hyperref}
\usepackage{framed}
\usepackage{enumitem}
\usepackage{parskip}

\usepackage{xcolor}
\hypersetup{
	colorlinks,
	linkcolor={red!50!black},
	citecolor={red!50!black},
	urlcolor={red!50!black}
}

\makeatletter
\def\thm@space@setup{%
	\thm@preskip=5mm
	\thm@postskip=\thm@preskip % or whatever, if you don't want them to be equal
}
\makeatother

% bold title for optional title in theorems
\makeatletter
\def\th@plain{%
	\thm@notefont{}% same as heading font
	\itshape % body font
}
\def\th@definition{%
	\thm@notefont{}% same as heading font
	\normalfont % body font
}
\makeatother

\theoremstyle{plain}
\newtheorem{theorem}{Theorem}
\newtheorem{lemma}[theorem]{Lemma}
\newtheorem{collorary}[theorem]{Collorary}
\newtheorem{proposition}{Proposition}


\theoremstyle{definition}
\newtheorem{definition}[theorem]{Definition}
\newtheorem*{example}{Example}
\newtheorem*{remark}{Remark}

% roman number
\newcommand{\rom}[1]{\uppercase\expandafter{\romannumeral #1\relax}}



\begin{document}

\title{Proofs: Multiple Riemann integral}
\author{Viet Duc Nguyen\\ Technical University of Berlin\\ Analysis III}
\date{December 23, 2018}
\maketitle
\tableofcontents

\section{One dimensional integral}
\begin{itemize}
	\item One may differentiate under the integral sign and the derivative is continuous.
	\begin{proof}
		First, we will show the following theorem: Let $f: [a,b] \times [c,d] \to \mathbb R$ be continuous and continuously partially differentiable in $[c,d]$. Consider a sequence $(y_k) \subset [c,d]$ with $y_k \to y$ and $y_k \neq y$. Then
		\[
			x \mapsto \frac{f(x,y_k) - f(x,y)}{y_k - y} \quad \text{uniformly converges to} \quad x \mapsto \frac{\partial}{\partial y}f(x,y).
		\]
		It holds
		\[
			\lim_{k \to \infty} \frac{\int^b_a f(x,u_k) dx - \int^b_a f(x,u) dx}{u_k - u} = \int^b_a \lim_{k \to \infty} \frac{f(x,u_k) - f(x,u)}{u_k -u}dx = \int^b_a \frac{\partial}{\partial u}f(x,u) dx.
		\]
		After \eqref{lena}, the integral is continuous, since $\frac{\partial}{\partial u}f(x,u)$ is continuous.
	\end{proof}
\end{itemize}


\section{Multiple integral on compact cuboids}
\begin{itemize}
	\item If $f: [a,b] \times U \to \mathbb R$ is continuous, then
	 \begin{align}\label{lena}
		(u_1,...,u_n) \mapsto \int_b^{a}f(x,u_1,...,u_n) dx \text{ is continuous}
	\end{align}
	\begin{proof}
		First, we need a theorem that states: Let $f: [a,b] \times U \to \mathbb R$ be continuous. Consider a sequence $(u_k) \subset U$ with $u_k \to u$. It holds:
		\begin{align}\label{auto}
			x \mapsto f(x,u_k) \quad \text{ uniformly converges for $k \to \infty$ to } \quad  x \mapsto f(x,u).
		\end{align}
		To show that $F(u_1,...,u_n) =  \int_b^{a}f(x,u_1,...,u_n) dx$ is continuous, we will prove for any sequence $(u_k) \subset U$ with $u_k \to u$ that $\lim_{k\to \infty}F(u_k) = F(u)$. It holds
		\[
			\lim_{k \to \infty} F(u_k) = \lim_{k \to \infty} \int^{a}_b f(x,u_k)dx \overset{(*)}{=} \int^{a}_b \lim_{k \to \infty} f(x,u_k)dx = \int^{a}_b f(x,u)dx = F(u).
		\]
		In the step $(*)$, we used that $f(x,u_k)$ uniformly converges to $f(x,u)$ due to \eqref{auto}. Therefore, we can exchange limit and integral.
	\end{proof}

	\item \textbf{Theorem of Fubini:} For any continuous function $f: [a,b] \times [c,d] \to \mathbb R$, it holds:
	\[
		\int^d_c\left(\int^b_a f(x,y) dx\right) dy = \int^b_a\left(\int^d_c f(x,y) dx\right) dy.
	\]
	\begin{proof}
		Consider the derivative of $y \mapsto  \int^b_a\left(\int^y_c f(x,u) du\right) dx$. It is
		\[
			\int^b_a \left(\frac{d}{dy} \int^y_c f(x,y)dy\right)dx = \int^b_a f(x,y) dx.
		\]
		Now, we will immediately see the result
		\[
				\int^d_c\left(\int^b_a f(x,y) dx\right) dy = \int^d_c  \frac{d}{dy} \left( \int^b_a \left( \int^y_c f(x,y)dy\right)dx \right) dy = \int^b_a \left( \int^d_c f(x,y)dy\right)dx
		\]
		Theorems we used to show the proofs: $\int f$ is continuous $\implies$ the double integral is well defined and one may differentiate under the integral sign.
	\end{proof}

	\begin{proof}
		Alternatively, the permutation is given by an orthogonal matrix $A$. Then $\int f(Ax) dx = \int f(x)dx$.
	\end{proof}
\end{itemize}

\section{Linear, monotonic and translation-invariant functionals} 
\begin{itemize}
	\item Let $f_k \to f$ uniformly and $f_k,f \in \mathcal C_c(\mathbb R^d)$. The support of all $f_k$ is contained in a compact cuboid $Q$. Then $J(f_k) \to J(f)$.
	
	\begin{proof}
		There is continuous $\Phi: \mathbb R^d \to [0,1]$ with $\Phi|_{Q} = 1$ and compact support $Q$. Then $-\Vert f_k - f\Vert \Phi \leq  f_k - f \leq \Vert f_k - f\Vert \Phi$ due to $\mathrm{supp}(f_k - f) \subset Q$. Applying the functional yields due to monotocity of $J$
		\[
			J(-\Vert f_k - f\Vert \Phi) \leq J(f_k - f)\leq J(\Vert f_k - f\Vert \Phi).
		\] 
		Here we need that there exists such a $\Phi$ that is continuous and compact. We need it here otherwise we cannot apply the functional on the term $\Vert f_k - f\Vert $ since it is not compact. We use linearity:
		\[
					-\Vert f_k - f\Vert J(\Phi) \leq J(f_k) - J(f)\leq \Vert f_k - f\Vert J(\Phi) \iff |J(f_k) - J(f)| \leq  \Vert f_k - f\Vert J(\Phi).
		\]
		Thus $J(f_k) \to J$ since $ \Vert f_k - f\Vert \to 0$.
	\end{proof}

	\item Every functional on $\mathcal C_c(\mathbb R^d)$ that is linear, monotonic and translation-invariant is unique up to a constant.
	
	\begin{proof}
		Let $c \coloneqq J(\Psi)$, It holds $J(\Psi_{2^{-n}}) = c I(\Psi_{2^{-n}})$. We approximate $f$ by $f_n = \sum f(k2^{-n})\theta_{k2^{-n}}\Psi_{2^{-n}}$. Since it is translation-invariant and linear, it follows $J(f_n) = cI(f_n)$. All $f_n, f$ are contained in a compact cuboid $Q$. Additionally, $f_n \to f$ uniformly. Thus, $J(f) = cI(f)$.
	\end{proof}
\end{itemize}

\section{Integration by substitution}
\begin{itemize}
	\item $\int_{\mathbb R^d} f(Ax) dx = \int_{\mathbb R^d} f(x)dx$ if $A$ is orthogonal.
	
	\begin{proof}
		There exists $c$ such that $\int f(Ax)dx = c \int f(x)dx$ because $\int \circ A$ is a linear, monotonic and translation-invariant functional. Then $f_0(x) = (1-\Vert x \Vert )_+$ implies $c = 1$ because $A$ is orthogonal.
	\end{proof}
\end{itemize}

\section{Volumes} 
\begin{itemize}
	\item 
\end{itemize}

\end{document}
