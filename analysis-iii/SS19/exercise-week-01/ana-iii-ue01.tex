\documentclass[a4paper, landscape,twocolumn,fontsize=9pt]{scrartcl}

\usepackage{fontspec}
\setmainfont[Ligatures=TeX]{Georgia}
\setsansfont[BoldFont="HelveticaNeue-Medium"]{Helvetica Neue}

\usepackage{amsmath, amsthm, amssymb}
\usepackage{mathtools}
\usepackage[most]{tcolorbox}
\usepackage{blindtext}
\usepackage{xcolor}
\usepackage{titlesec}
\usepackage{titling}


\newfontfamily\Menlo[Ligatures=TeX]{Menlo}

\definecolor{grey}{rgb}{0.5,0.5,0.5}
\definecolor{lightgrey}{rgb}{0.9,0.9,0.9}
\definecolor{darkgrey}{rgb}{0.3,0.3,0.3}
\definecolor{orange}{rgb}{0.94, 0.55, 0.294}
\definecolor{pink}{rgb}{0.94, 0.29, 0.7}
\definecolor{yellow}{rgb}{1, 0.749, 0}

\newcommand{\chapfnt}{\fontsize{16}{19}}
\newcommand{\secfnt}{\fontsize{18}{17}}
\newcommand{\ssecfnt}{\fontsize{14}{14}}
\renewcommand{\hline}{\noindent\makebox[\linewidth]{\rule{12cm}{1pt}}}
\newcommand{\code}[1]{{\Menlo{\color{darkgrey}#1}}}
\newcommand{\vip}[1]{\textit{\textbf{#1}}}

\titleformat{\chapter}[display]
{\normalfont\chapfnt\bfseries}{\chaptertitlename\ \thechapter}{20pt}{\chapfnt}

\titleformat{\section}
{\normalfont\sffamily\secfnt\mdseries}{\thesection}{1em}{}

\titleformat{\subsection}
{\normalfont\sffamily\ssecfnt\mdseries\color{grey}}{\thesubsection}{1em}{}

\titlespacing*{\chapter} {0pt}{50pt}{40pt}
\titlespacing*{\section} {0pt}{24pt}{16pt}
\titlespacing*{\subsection} {0pt}{6pt}{1.5ex plus .2ex}

    
\usepackage{geometry}
\setlength{\columnsep}{32mm}
\geometry{
 left=22mm,
 right=22mm,
 bottom=32mm,
 top = 20mm
}


\newtcbtheorem[auto counter,number within=section]{theorem}%
  {Theorem}{
  		fonttitle=\upshape, 
  		fontupper=\upshape,
  		boxrule=0pt,
  		leftrule=3pt,
  		arc=0pt,auto outer arc,
  		colback=white,
  		colframe=pink,
  		colbacktitle=white,
  		coltitle=pink,
  		oversize,
  		enlarge top by=1mm,
  		enlarge bottom by=1mm,
    	enhanced jigsaw,
    	interior hidden, 
    	before skip=4pt,
    	overlay={
    		\draw[line width=1.5pt,pink] (frame.north west) -- (frame.south west);
  		}, 
  		frame hidden}{theorem}
  		
 \newtcbtheorem[auto counter,number within=section]{definition}%
  {Definition}{
  		fonttitle=\upshape, 
  		fontupper=\upshape,
  		boxrule=0pt,
  		leftrule=3pt,
  		arc=0pt,auto outer arc,
  		colback=white,
  		colframe=orange,
  		colbacktitle=white,
  		coltitle=orange,
  		oversize,
  		enlarge top by=1mm,
  		enlarge bottom by=1mm,
    	enhanced jigsaw,
    	interior hidden, 
    	before skip=4pt,
    	overlay={
    		\draw[line width=1.5pt,orange] (frame.north west) -- (frame.south west);
  		}, 
  		frame hidden}{definition}

\newtcbtheorem[auto counter,number within=section]{lemma}%
  {Lemma}{
  		fonttitle=\upshape, 
  		fontupper=\upshape,
  		boxrule=1pt,
  		toprule=0pt,
  		leftrule=3pt,
  		arc=0pt,auto outer arc,
  		colback=white,
  		colframe=orange,
  		colbacktitle=white,
  		coltitle=orange,
  		oversize,
  		enlarge top by=1mm,
  		enlarge bottom by=1mm,
    	enhanced jigsaw,
    	interior hidden, 
    	before skip=4pt,
    	overlay={
    		\draw[line width=1.5pt,orange] (frame.north west) -- (frame.south west);
  		}, 
  		frame hidden}{lemma}
    	
\newtcbtheorem[auto counter,number within=section]{example}%
  {Beispiel}{
  		fonttitle=\upshape, 
  		fontupper=\upshape,
  		boxrule=0pt,
  		leftrule=3pt,
  		arc=0pt,auto outer arc,
  		colback=white,
  		colframe=grey,
  		colbacktitle=white,
  		coltitle=grey,
  		oversize,
  		enlarge top by=1mm,
  		enlarge bottom by=1mm,
    	enhanced jigsaw,
    	interior hidden, 
    	before skip=4pt,
    	overlay={
    		\draw[line width=1.5pt,grey] (frame.north west) -- (frame.south west);
  		}, 
  		frame hidden}{example}
    	
\newtcbtheorem[auto counter,number within=section]{note}%
  {Notiz}{
  		fonttitle=\upshape, 
  		fontupper=\upshape,
  		boxrule=0pt,
  		leftrule=3pt,
  		arc=0pt,auto outer arc,
  		colback=white,
  		colframe=yellow,
  		colbacktitle=white,
  		coltitle=yellow,
  		oversize,
  		enlarge top by=1mm,
  		enlarge bottom by=1mm,
    	enhanced jigsaw,
    	interior hidden, 
    	before skip=4pt,
    	overlay={
    		\draw[line width=1.5pt,yellow] (frame.north west) -- (frame.south west);
  		}, 
  		frame hidden}{note}
  		
\newtcbtheorem[]{important}%
  {Wichtig}{
  		fonttitle=\upshape, 
  		fontupper=\upshape,
  		boxrule=0pt,
  		leftrule=3pt,
  		arc=0pt,auto outer arc,
  		colback=white,
  		colframe=pink,
  		colbacktitle=white,
  		coltitle=pink,
  		oversize,
  		enlarge top by=1mm,
  		enlarge bottom by=1mm,
    	enhanced jigsaw,
    	interior hidden, 
    	before skip=4pt,
    	overlay={
    		\draw[line width=1.5pt,pink] (frame.north west) -- (frame.south west);
  		}, 
  		frame hidden}{important}
    	
\renewcommand{\baselinestretch}{1.4} 
\makeatletter
\let\old@rule\@rule
\def\@rule[#1]#2#3{\textcolor{lightgrey}{\old@rule[#1]{#2}{#3}}}
\makeatother

\begin{document}

\section{Implizite Funktionen}
\begin{theorem}{Satz über implizite Funktionen}{}
	Sei $F: \mathbb R^n \times \mathbb R^m \to \mathbb R^m$ stetig differenzierbar mit $F(x_0,y_0) =$ und $\frac{\partial}{\partial y} F(x_0,y_0)$ ist invertierbar. Dann existieren offene Umgebungen $U$ und $V$ von $x_0$ bzw. $y_0$ und eine eindeutige Funktion $f: U \to V$ mit 
	\[
		F(x, f(x)) = 0
	\]	
	und $Df(x) = - (\frac{\partial F}{\partial y}(x,f(x)))^{-1} \frac{\partial F}{\partial x}(x,f(x)), \quad \forall x \in U$. 
\end{theorem}

\textbf{Aufgabe:} Zeige, dass das Gleichungssystem 
\[
	2xy + \cos x^2 + \sin y^2 -4x + y = 1
\]
in der Nähe des Punktes $(x_0,y_0) = (0,0)$ eine Lösung der Form $y = f(x)$ hat und bestimmte das Taylorpolynom zweiten Grades in $x_0 = 0$.

\textbf{Lösung:} Setze $F(x,y) = 2xy + \cos x^2 + \sin y^2 - 4x + y -1$. Dann ist $$F(0,0) = 0.$$ Bestimme die Ableitung und überprüfe, ob sie invertierbar ist.
\begin{align*}
	\frac{\partial F}{\partial y}(x,y) = 2x + 2y \cos y^2 + 1 \\
	\frac{\partial F}{\partial y}(0,0) = 1 \neq 0.
\end{align*}
Mit dem Satz über implizite Funktionen erhalten wir Umgebungen $U, V \subset \mathbb R$ von $0$ und Funktion $f: U \to V$ mit $F(x, f(x)) = 0$.

Für die Ableitung gilt
\begin{align*}
	f'(x) &= - (\frac{\partial F}{\partial y}(x,f(x)))^{-1} \frac{\partial F}{\partial x}(x,f(x)) \\
		  &= - \frac{1}{2x + 2f(x) \cos(f(x)^2) + 1}(2f(x) - 2x \sin x^2 -4) \\
	f'(0) &= -\frac{1}{0+0+1}(0 - 0 - 4) = 4.
\end{align*}

\section{Mannigfaltigkeiten}
\begin{definition}{Untermannigfaltigkeit}{ewrwer}
	Sei $\mathcal M \subset \mathbb R^n$. $\mathcal M$ ist eine $k$-dimensionale Untermannigfaltigkeit des $\mathbb R^n$ genau dann, wenn zu jedem $x \in \mathcal M$ eine offene Umgebung $V \subset \mathbb R^n$ um $x$ existiert und offene Umgebung $U \subset \mathbb R^k$ und ein immersiver Homöomorphismus $\varphi: U \to \mathcal M$ ($\varphi$ ist bijektiv, stetig diffbar, $D\varphi$ injektiv und $\varphi^{-1}$ stetig), sodass
	\[
		\varphi(U) = \mathcal M \cap V.
	\]
\end{definition}


\textbf{Aufgabe:} Wie sehen die $n$- bzw. $0$-dimensionalen Untermannigfaltigkeiten des $\mathbb R^n$ aus?

\textbf{Lösung:} Die $n$-dimensionalen Untermannigfaltigkeiten des $\mathbb R^n$ sind genau die offenen Teilmengen $\mathbb R^n$. 

Sei $U \subset \mathbb R^n$ offen. Wähle $\varphi = Id |_U$, $V=U$. $\varphi$ ist dann ein immersersiver Homöomorphismus.

Sei $M$ eine $n$-dimensionale Untermannigfaltigkeit. Sei $x \in \mathcal M$ beliebig. Es existiert ein Flachmacher um $x$, d.h. ein Diffeomorphismus $\psi: U \to V$ mit $U,V \subset \mathbb R^n$ offen, $x \in U$ mit $\psi(\mathcal M \cap U) = \mathbb R^n \cap V = V$. Es folgt, dass $M \cap U = \psi^{-1}(V)$ ist offen. Es existiert ein $\epsilon > 0$, sodass $U_{\epsilon}(x) \subset U \cap M \subset \mathcal M \implies x$ ist ein innerer Punkt von $\mathcal M$.

Die $0$-dimensionalen Untermannigfaltigkeiten des $\mathbb R^n$ sind genau die diskreten Teilmengen des $\mathbb R^n$, d.h. die Mengen $M$, sodass für jedes $x \in M$ ein $\epsilon > 0$ existiert mit $M \cap U_{\epsilon}(x) = \{ x \}$.

Sei $M \subset \mathbb R^n$ diskret. Sei $x \in M$ beliebig. Es existiert ein $\epsilon > 0$, sodass $M \cap U_\epsilon(x) = \{ x\}$. Definiere $\psi: U \to V, v \mapsto v - x$. Setze $U = U_{\epsilon}(x), V = U_\epsilon(x) - x = U_\epsilon(0)$. Dann ist $\psi$ ein Flachmacher, da 
\[
	\psi(M \cap U) = \psi(\{ x \}) = \{ 0 \} = \{ (x_1,...,x_n) \in \mathbb R^n: x_1=...=x_n = 0 \} \cap V.
\] 
Sei $M$ eine $0$-dimensionale Untermannigfaltigkeit des $\mathbb R^n$. Sei $x \in M$ beliebig. Es existiert ein Flachmacher, das heißt, ein Diffeomorphismus $\psi: U \to V$ mit $U,V \subset \mathbb R$ offen, $x \in U$, sodass
\[
	\psi(M \cap U) = \{ (x_1,...,x_n) \in \mathbb R^n: x_1 = ... = x_n = 0 \} \cap V = \{ 0 \} \cap V = \{ 0 \}.
\]
Damit $M \cap U = \psi^{-1} (\{ 0 \} ) = \{ x \} \implies M$ ist diskret, da $x$ beliebig war.

\textit{Zu den Hausaufgaben:} $\mathcal l^p$ sind Untermannigfaltigkeiten des $\mathbb R^n$, wenn $p > 1$. Bei $p = 1$ gibt es Probleme bei den Ecken.

\section{Lagrange Verfahren}
\textbf{Ziel:} Finde Extrema von $f: \mathbb R^n \supset G \to \mathbb R$, wobei $G$ offen ist. Unter Nebenbedingung $g(x) = 0$, wobei $g=(g_1,...,g_m) : G \to \mathbb R^m$.

\subsection*{Verfahren}
\begin{enumerate}
	\item \textbf{Kandidaten für Extrema:} Definiere die Lagrange-Funktion 
	\[
		L(x_1,..,x_n, \lambda_1,...,\lambda_m) = f(x_1,...,x_n) + \sum^m_{i=1} \lambda_i g_i(x_1,...,x_n)
	\]
	Kandidaten erfüllen
	\begin{align*}
		0 &= \frac{\partial}{\partial x_i} L(x_1,...,x_n,\lambda_1,...,\lambda_m) \quad \forall i = 1,...,n \\
		0 &= \frac{\partial}{\partial \lambda_j} L(x_1,...,x_n,\lambda_1,...,\lambda_m) \quad \forall j = 1,...,m
	\end{align*}
	Gleichungssystem lösen.
	
	
	\item \textbf{Art der Extrema bestimmen:} Betrachte die geänderte Hesse-Matrix
	\begin{align*}
		H(\lambda_1,...,\lambda_m, x_1,...,x_n) &= \begin{pmatrix}
			\frac{\partial^2 L}{\partial \lambda_1^2} & ... & \frac{\partial^2 L}{\partial \lambda_1 \partial \lambda_m} & \frac{\partial L}{\partial \lambda_1 \partial x_1} & ... \\
			\vdots & \ddots & \\
			\frac{\partial L}{\partial \lambda_1 \partial \lambda_m} & ... & \frac{\partial^2 L}{\partial \lambda_m^2} \\
			\vdots & & & \frac{\partial^2L}{\partial^2 x_1} \\
			& & & & \ddots \\
			& & & & & \frac{\partial^2 L}{\partial^2 x_n}
		\end{pmatrix} \\
		&= \begin{pmatrix}
			0 & ... & 0 \\
			\vdots & \ddots & \\
			0 & ... & 0 \\
			\frac{\partial^2 L}{\partial \lambda_1 \partial x_1} &... && \frac{\partial^2L}{\partial x_1^2} \\
			\vdots 
		\end{pmatrix}
	\end{align*}
	Betrachte die Vorzeichenfolge der führenden Hauptminoren (führende $k > 2m$ Hauotminoren).
	
	\textbf{Beispiel:} eine Nebenbedingung $m = 1$ und $n \geq 2$.
	\begin{itemize}
		\item 3. Hauptminor positiv und danach folgende alternierend $\implies$ Maximum
		\item 3. Hauptminor und alle folgenden negativ $\implies$ Minimum
	\end{itemize}
\end{enumerate}

\end{document}
