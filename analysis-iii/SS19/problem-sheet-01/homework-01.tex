\documentclass[a4paper]{article}

\usepackage{fontspec}
\setmainfont[Ligatures=TeX]{Georgia}
\setsansfont[BoldFont="HelveticaNeue-Medium"]{Helvetica Neue}

\usepackage{amsmath, amsthm, amssymb}
\usepackage{mathtools}
\usepackage[most]{tcolorbox}
\usepackage{blindtext}
\usepackage{xcolor}
\usepackage{titlesec}
\usepackage{titling}


\newfontfamily\Menlo[Ligatures=TeX]{Menlo}

\definecolor{grey}{rgb}{0.5,0.5,0.5}
\definecolor{lightgrey}{rgb}{0.9,0.9,0.9}
\definecolor{darkgrey}{rgb}{0.3,0.3,0.3}
\definecolor{orange}{rgb}{0.94, 0.55, 0.294}
\definecolor{pink}{rgb}{0.94, 0.29, 0.7}
\definecolor{yellow}{rgb}{1, 0.749, 0}

\newcommand{\chapfnt}{\fontsize{16}{19}}
\newcommand{\secfnt}{\fontsize{18}{17}}
\newcommand{\ssecfnt}{\fontsize{14}{14}}
\renewcommand{\hline}{\noindent\makebox[\linewidth]{\rule{12cm}{1pt}}}
\newcommand{\code}[1]{{\Menlo{\color{darkgrey}#1}}}
\newcommand{\vip}[1]{\textit{\textbf{#1}}}

\titleformat{\chapter}[display]
{\normalfont\chapfnt\bfseries}{\chaptertitlename\ \thechapter}{20pt}{\chapfnt}

\titleformat{\section}
{\normalfont\sffamily\secfnt\mdseries}{\thesection}{1em}{}

\titleformat{\subsection}
{\normalfont\sffamily\ssecfnt\mdseries\color{grey}}{\thesubsection}{1em}{}

\titlespacing*{\chapter} {0pt}{50pt}{40pt}
\titlespacing*{\section} {0pt}{0pt}{16pt}
\titlespacing*{\subsection} {0pt}{12pt}{8pt}



\newtcbtheorem[auto counter,number within=section]{theorem}%
  {Theorem}{
  		fonttitle=\upshape, 
  		fontupper=\upshape,
  		boxrule=0pt,
  		leftrule=3pt,
  		arc=0pt,auto outer arc,
  		colback=white,
  		colframe=pink,
  		colbacktitle=white,
  		coltitle=pink,
  		oversize,
  		enlarge top by=1mm,
  		enlarge bottom by=1mm,
    	enhanced jigsaw,
    	interior hidden, 
    	before skip=12pt,
    	overlay={
    		\draw[line width=1.5pt,pink] (frame.north west) -- (frame.south west);
  		}, 
  		frame hidden}{theorem}

\newtcbtheorem[auto counter,number within=section]{lemma}%
  {Lemma}{
  		fonttitle=\upshape, 
  		fontupper=\upshape,
  		boxrule=1pt,
  		toprule=0pt,
  		leftrule=3pt,
  		arc=0pt,auto outer arc,
  		colback=white,
  		colframe=yellow,
  		colbacktitle=white,
  		coltitle=yellow,
  		oversize,
  		enlarge top by=1mm,
  		enlarge bottom by=1mm,
    	enhanced jigsaw,
    	interior hidden, 
    	before skip=12pt,
    	overlay={
    		\draw[line width=1.5pt,yellow] (frame.north west) -- (frame.south west);
  		}, 
  		frame hidden}{lemma}
  		
 \newtcbtheorem[auto counter,number within=section]{definition}%
  {Definition}{
  		fonttitle=\upshape, 
  		fontupper=\upshape,
  		boxrule=1pt,
  		toprule=0pt,
  		leftrule=3pt,
  		arc=0pt,auto outer arc,
  		colback=white,
  		colframe=orange,
  		colbacktitle=white,
  		coltitle=orange,
  		oversize,
  		enlarge top by=1mm,
  		enlarge bottom by=1mm,
    	enhanced jigsaw,
    	interior hidden, 
    	before skip=12pt,
    	overlay={
    		\draw[line width=1.5pt,orange] (frame.north west) -- (frame.south west);
  		}, 
  		frame hidden}{definition}
    	
\newtcbtheorem[auto counter,number within=section]{example}%
  {Beispiel}{
  		fonttitle=\upshape, 
  		fontupper=\upshape,
  		boxrule=0pt,
  		leftrule=3pt,
  		arc=0pt,auto outer arc,
  		colback=white,
  		colframe=grey,
  		colbacktitle=white,
  		coltitle=grey,
  		oversize,
  		enlarge top by=1mm,
  		enlarge bottom by=1mm,
    	enhanced jigsaw,
    	interior hidden, 
    	before skip=12pt,
    	overlay={
    		\draw[line width=1.5pt,grey] (frame.north west) -- (frame.south west);
  		}, 
  		frame hidden}{example}
    	
\newtcbtheorem[auto counter,number within=section]{note}%
  {Notiz}{
  		fonttitle=\upshape, 
  		fontupper=\upshape,
  		boxrule=0pt,
  		leftrule=3pt,
  		arc=0pt,auto outer arc,
  		colback=white,
  		colframe=yellow,
  		colbacktitle=white,
  		coltitle=yellow,
  		oversize,
  		enlarge top by=1mm,
  		enlarge bottom by=1mm,
    	enhanced jigsaw,
    	interior hidden, 
    	before skip=12pt,
    	overlay={
    		\draw[line width=1.5pt,yellow] (frame.north west) -- (frame.south west);
  		}, 
  		frame hidden}{note}
  		
\newtcbtheorem[]{important}%
  {Wichtig}{
  		fonttitle=\upshape, 
  		fontupper=\upshape,
  		boxrule=0pt,
  		leftrule=3pt,
  		arc=0pt,auto outer arc,
  		colback=white,
  		colframe=pink,
  		colbacktitle=white,
  		coltitle=pink,
  		oversize,
  		enlarge top by=1mm,
  		enlarge bottom by=1mm,
    	enhanced jigsaw,
    	interior hidden, 
    	before skip=12pt,
    	overlay={
    		\draw[line width=1.5pt,pink] (frame.north west) -- (frame.south west);
  		}, 
  		frame hidden}{important}
    	
\renewcommand{\baselinestretch}{1.4} 
\makeatletter
\let\old@rule\@rule
\def\@rule[#1]#2#3{\textcolor{lightgrey}{\old@rule[#1]{#2}{#3}}}
\makeatother

\begin{document}
\section*{Analysis Problem Sheet 01}
\textit{Duc (395220), Moritz Bichlmeyer (392374)}

\subsection*{Exercise 1}
\textbf{Zeige:} Das Gleichungssystem besitzt eine Lösung $(x,y,z,u,v)$ der Form $g: V \to W, (x,y,z) \mapsto (u,v)$ mit $g(2,0,1) = (1,0)$, offene Umgebungen $V \subset \mathbb R^3$ und $W \subset \mathbb R^2$.

\begin{proof}
Schreibe das Gleichungssystem in der Form
\[
	f(x,y,z,u,v) = \begin{pmatrix}
		xe^y +uz + \cos v - 4 \\
		u \cos y + x^2v + yz^2 - 1
	\end{pmatrix}.
\]
Dann suchen wir Umgebungen $V,W$ und $g: V \to W$, sodass $f^{-1}(\{ 0 \}) = \{ (\alpha, \beta) \in  \mathbb R^3 \times \mathbb R^2 : g(\alpha) = \beta \}$. Wir verwenden das Theorem über \emph{implizite Funktionen}.

\begin{enumerate}
\item $f$ besitzt bei $(2,0,1,1,0)$ eine Nullstelle. Das rechnet man nach
\[
	f(2,0,1,1,0) = \begin{pmatrix}
		2 + 1 + 1 - 4 \\
		1 + (-1)
	\end{pmatrix} = \mathbf 0.
\]

\item $f$ ist stetig differenzierbar nach $(u,v)$: 
\[
	\frac{\partial f}{\partial (u,v)} (x,y,z,u,v) = \begin{pmatrix}
	z & -\sin(v) \\
	\cos(y) & x^2
	\end{pmatrix}
\]
Wir sehen, dass $\frac{\partial f}{\partial (u,v)}(2,0,1,1,0) = \begin{pmatrix}
1 & 0 \\ 1 & 4
\end{pmatrix}$. Die Ableitung ist also an der Stelle $(2,0,1,1,0)$ nicht singulär, denn die Determinante beträgt $4$.
\end{enumerate}

Nach Satz über implizite Funktionen gibt es offene Umgebungen $(2,0,1) \in V$ und $(1,0) \in W$ und eine stetig differenzierbare Funktion $g: V \to W, (x,y,z) \mapsto (u,v)$ mit $f(x,y,z, g(x,y,z)) = 0$ für alle $(x,y,z,u,v) \in V \times W$. Außerdem gilt: $g(2,0,1) = (1,0)$.
\end{proof}

\textbf{Suche:} Das erste Taylor Polynom von $g$ mit Entwicklungspunkt $\mathbf a = (2,0,1)$. 

Sei $\mathbf x \in \mathbb V$.
\[
	T_1g(\mathbf x, \mathbf a) = g(\mathbf a) + D_{\mathbf a}g(\mathbf x - \mathbf a).
\]
Wir wissen, dass $g(\mathbf a) = g(2,0,1) = (1,0)$. Als nächstes berechne $D_{\mathbf a}g$:
\begin{align*}
	D_{\mathbf a}g& = -(\frac{\partial f}{\partial(u,v)}(\mathbf a, g(\mathbf a)))^{-1} \frac{\partial f}{\partial (x,y,z)}(\mathbf a, g(\mathbf a)) \\
	& = -(\frac{\partial f}{\partial(u,v)}(2,0,1,1,0))^{-1} \frac{\partial f}{\partial (x,y,z)}(2,0,1,1,0) \\
	&= - \begin{pmatrix}
			1 & 0 \\ 1 & 4
	\end{pmatrix}^{-1} \begin{pmatrix}
		e^y & xe^y & z \\
		2xv & -u\sin(y) + z^2 & 2zy
	\end{pmatrix}\Bigg \vert_{\mathbf x = (2,0,1,1,0)} \\
	&= -\frac{1}{4}\begin{pmatrix}
		4 & 0 \\ -1 & 1
	\end{pmatrix} \begin{pmatrix}
		1 & 2 & 1 \\ 0 & 1 & 0
	\end{pmatrix} \\
	&= -\frac{1}{4}\begin{pmatrix}
		4 & 8 & 4 \\
		-1 & -1 & -1
	\end{pmatrix}
\end{align*}
Berechne nun $D_\mathbf{a}g(\mathbf a) = -\frac{1}{4}\begin{pmatrix}
		4 & 8 & 4 \\
		-1 & -1 & -1
	\end{pmatrix}\begin{pmatrix}
		2 \\ 0 \\ 1
	\end{pmatrix} = \begin{pmatrix}
		-3 \\ \frac{3}{4}
	\end{pmatrix}$. 
Das Taylorpolynom lautet dann für $\mathbf x = (x,y,z) \in V$:
\begin{align*}
	T_1g(\mathbf x, \mathbf a) &= \begin{pmatrix}
	1 \\ 0 
	\end{pmatrix} - \begin{pmatrix}
		-3 \\ \frac{3}{4}
	\end{pmatrix} - \frac{1}{4}\begin{pmatrix}
		4 & 8 & 4 \\
		-1 & -1 & -1
	\end{pmatrix} \begin{pmatrix}
		x \\ y \\ z
	\end{pmatrix}  \\
	&= \begin{pmatrix}
		4 \\ -\frac{3}{4}
	\end{pmatrix} + \begin{pmatrix}
		-1 & -2 & -1 \\ \frac{1}{4} & \frac{1}{4} & \frac{1}{4}
	\end{pmatrix}\begin{pmatrix}
	x \\ y \\ z
	\end{pmatrix} \\
	&= \begin{pmatrix}
	4 -x -2y - z \\
	\frac{1}{4}(- 3 + x + y + z)
	\end{pmatrix}
\end{align*}

\hline


\subsection*{Exercise 2}
\textbf{Zeige:} Nullstellen eines Polynoms hängen  von einer stetig differenzierbaren Funktion $f$ ab

\begin{proof}
Sei $F: \mathbb R^{n+1} \times \mathbb R \to \mathbb R, (\mathbf a, x) = \sum^n_{i=0}a_i x^{i}$. Sie ist stetig differenzierbar:
\[
	D_{(\mathbf a,x)}f = \begin{pmatrix}
		x & x^2 & ... & x^n & \sum^n_{i=1} a_iix^{i-1}
	\end{pmatrix}, \quad \mathbf a \in \mathbb R^{n+1}, x \in \mathbb R.
\]
Sei $\mathbf a \in \mathbb R^{n+1}$ beliebig und nehmen man an, dass es eine Nullstelle $x_0 \in \mathbb R$ gibt mit $F(\mathbf a, x_0) = 0$ und $\frac{\partial F}{\partial_2}(\mathbf a, x_0) \neq 0$. 

Nach Satz über implizite Funktionen gibt es eine Umgebung $U \subset \mathbb R^{n+1}$ mit $\mathbf a \in U$ und $V \subset \mathbb R$, sodass $F(\mathbf a, f(\mathbf a)) = 0$ für alle $\mathbf a \in U$ für eine stetig differenzierbare Funktion $f: U \to V$.

Damit haben wir solch eine Funktion $f$ gefunden.
\end{proof}



\subsection*{Exercise 3}

\textbf{Zeige:} Für $p > 1$ ist $S_p$ eine Untermannigfaltigkeit.
\begin{proof}
Wir versuchen $S_p$ als Lösungsmenge eines Gleichungssystems mit einer \textit{stetig differenzierbaren} $f$ darzustellen. Sei $p > 1$. Betrachte $f: \mathbb R^n \to \mathbb R, \mathbf x \mapsto \sum^n_{i = 1} |x_i|^p - 1$. Es gilt
\[
	S_p = f^{-1}{(\{0\})}.
\]
Als nächstes müssen wir zeigen, dass $f$ stetig differenzierbar auf eine Umgebung $U_{\epsilon(a)}(a)$ für beliebige $a \in S_p$ und $\epsilon(a) > 0$. Dies zeigen wir, indem wir die folgende stärkere Aussage beweisen: $f$ ist stetig differenzierbar auf ganz $\mathbb R^n$. 
\hline

\begin{enumerate}
\item Sei $\mathbf x = (x_1,...,x_n) \in \mathbb R^n$ mit $x_i \neq 0$ für alle $i = 1,...,n$. Dann ist
\[
	D_{\mathbf x}f = p \begin{pmatrix}
		x_1 ||x_1||^{p - 2} & ... & x_n ||x_n||^{p-2}
	\end{pmatrix},
\]
denn $D_x ||x|| = D_x \sqrt{x^2} = \frac{x}{||x||}$ und mit der Kettenregel ergibt sich $D_x||x||^p = p || x ||^{p-1} \cdot \frac{x}{||x||} = px||x||^{p-2}$. 

Wir sehen, dass $f$ natürlich stetig ist und auch wohldefiniert ist für alle $\mathbf x$ mit $x_i \neq 0$.

\item Sei $\mathbf x = (x_1,...,x_n)$ mit $x_i = 0$ für alle $i \in I \subset \{ 1,..., n \}$, wobei $I$ nicht leer ist. Die Frage ist, ob $D_\mathbf{x}f$ existiert. Die kritischen Stellen sind die $x_i = 0$ mit $i \in I$. Untersuche also, ob der Grenzwert von $x||x||^{p-2}$ für $x \to 0$ existiert. Sei $q = p-2$ und es gilt $q > -1$ wegen $p > 1$. Daher gilt
\[
	\lim_{x \to 0} x||x||^q = (\pm 1) \cdot \lim_{x \to 0}x^{\overbrace{1+q}^{>0}} = 0.
\]
Der Grenzwert existiert also.

\item Zusammengefasst: Sei $\varphi(x) = \begin{cases}
	x||x||^{p-2} \quad & \text{falls } x \neq 0 \\
	0 & \text{sonst}
\end{cases}$. Dann ist für alle $\mathbf x \in \mathbb R^n$:
\[
	D_\mathbf{x}f = \begin{pmatrix}
		\varphi(x_1) & ... & \varphi(x_n)
	\end{pmatrix}.
\]
Die Ableitung existiert damit überall. Sie ist natürlich auch stetig; insbesondere um der Umgebung $\mathbf x = \mathbf 0$, wie wir gezeigt haben.
\end{enumerate}

\hline

Als nächstes zeigen wir, dass $D_\mathbf{x}f$ injektiv ist. Dafür muss $D_\mathbf{x}f \neq \mathbf 0$ für alle $\mathbf x \in S_p$. Wir sehen, dass $D_\mathbf{x} f = \mathbf 0 \iff \mathbf x = \mathbf 0$, aber $\mathbf 0 \notin S_p$. Somit ist $S_p$ eine $n-1$-Untermannigfaltigkeit in $\mathbb R^n$, denn $S_p$ kann als Lösungsmenge eines Gleichungssystem (mit einer Gleichung) dargestellt werden.
\end{proof}

\hline

\textbf{Zeige:} Für $p = 1$ ist $S_p$ keine Untermannigfaltigkeit.

\begin{proof}
Siehe handschriftlich beschriebenes Blatt.
\end{proof}



\subsection*{Exercise 4}
Gegeben ist das folgende Minimierungsproblem
\begin{align*}
	\begin{cases}
		\min f(x,y) \\
		g(x,y) = 0
	\end{cases}
\end{align*}
mit $f(x,y) = \sqrt{x^2 + y^2}$ und $g(x,y) = x^2 + y^2 + xy - 1$. Da die Wurzelfunktion monoton ist, betrachten wir das einfachere Problem
\begin{align*}
	\begin{cases}
		\min x^2 + y^2 \\
		g(x,y) = 0
	\end{cases}.
\end{align*}
Wir führen die Lagrangefunktion ein:
\[
	L(x,y,\lambda) = x^2 + y^2 + \lambda(x^2 + y^2 +xy - 1).
\]
Diese leiten wir ab und erhalten das folgende Gleichungssystem:
\begin{align*}
	\frac{\partial L}{\partial x} &= 2x + \lambda(2x + y) = 0 \\
	\frac{\partial L}{\partial y} &= 2y + \lambda(2y + x) = 0 \\
		\frac{\partial L}{\partial \lambda} &= x^2 + y^2 +xy - 1 = 0
\end{align*}
\textbf{1. Fall:} $y \neq -\frac{1}{2}x$ und $y \neq -2x$. 
\begin{align*}
	\frac{2x}{2x + y} + \lambda &= 0 \\
	\frac{2y}{2y + x} + \lambda &= 0 \\ 
	 x^2 + y^2 +xy - 1 &= 0
\end{align*}
Subtraktion der ersten beiden Gleichungen ergeben
\begin{align*}
2x(2y + x) - 2y(2x + y)  = 0
\end{align*}
und somit 
\[
	2x^2 - 2y^2 = 0 \implies x = \pm y.
\]
\begin{enumerate}
\item Setzt man $x = y$, so ergibt sich:
\[
	3x^2 = 1 \implies x = \pm \frac{1}{\sqrt{3}}.
\]
Daher $x = y = \frac{1}{\sqrt{3}}$ und $x = y = -\frac{1}{\sqrt{3}}$. Es gilt: $\lambda = - \frac{1}{\sqrt 6}$ in beiden Fällen.

\item Für $x = -y$:
\[
	x^2 = 1 \implies x = \pm 1.
\]
Daher $x = 1, y= -1$ und $x = -1, y = 1$. In beiden Fällen ist $\lambda = -\frac{1}{\sqrt 2}$.
\end{enumerate}
Die Lösungsmenge des Gleichungssystems lautet demnach
\begin{align*}
	x &= \frac{1}{\sqrt{3}}, y = \frac{1}{\sqrt{3}}, \lambda = -\frac{1}{\sqrt 6} \\
	x &= -\frac{1}{\sqrt{3}}, y = -\frac{1}{\sqrt{3}}, \lambda = -\frac{1}{\sqrt 6} \\
	x &= 1, y = -1, \lambda = -\frac{1}{\sqrt 2} \\
	 x &= -1, y = 1, \lambda = -\frac{1}{\sqrt 2}
\end{align*}

\textbf{2. Fall:} $y = -\frac{1}{2}x$. Wir setzten das in die zweite Gleichung ein und erhalten$2y = 0$, sodass $x = y = 0$. Die Nebenbedingung ist nicht erfüllt.

\textbf{3. Fall:} $y = -2x$. Wir setzten das in die zweite Gleichung ein und erhalten$2x = 0$, sodass $x = y = 0$. Die Nebenbedingung ist nicht erfüllt.

\hline 

\textbf{Art der Extrema:} Siehe handschriftlich beschriebenes Blatt.
\end{document}