\documentclass[a4paper]{article}

\usepackage{fontspec}
\setmainfont[Ligatures=TeX]{Georgia}
\setsansfont[BoldFont="HelveticaNeue-Medium"]{Helvetica Neue}

\usepackage{amsmath, amsthm, amssymb}
\usepackage{mathtools}
\usepackage[most]{tcolorbox}
\usepackage{blindtext}
\usepackage{xcolor}
\usepackage{titlesec}
\usepackage{titling}
\usepackage{enumitem}% http://ctan.org/pkg/enumitem


\newfontfamily\Menlo[Ligatures=TeX]{Menlo}

\definecolor{grey}{rgb}{0.5,0.5,0.5}
\definecolor{lightgrey}{rgb}{0.9,0.9,0.9}
\definecolor{darkgrey}{rgb}{0.3,0.3,0.3}
\definecolor{orange}{rgb}{0.94, 0.55, 0.294}
\definecolor{pink}{rgb}{0.94, 0.29, 0.7}
\definecolor{yellow}{rgb}{1, 0.749, 0}

\newcommand{\chapfnt}{\fontsize{16}{19}}
\newcommand{\secfnt}{\fontsize{18}{17}}
\newcommand{\ssecfnt}{\fontsize{14}{14}}
\renewcommand{\hline}{\noindent\makebox[\linewidth]{\rule{12cm}{1pt}}}
\newcommand{\code}[1]{{\Menlo{\color{darkgrey}#1}}}
\newcommand{\vip}[1]{\textit{\textbf{#1}}}

\titleformat{\chapter}[display]
{\normalfont\chapfnt\bfseries}{\chaptertitlename\ \thechapter}{20pt}{\chapfnt}

\titleformat{\section}
{\normalfont\sffamily\secfnt\mdseries}{\thesection}{1em}{}

\titleformat{\subsection}
{\normalfont\sffamily\ssecfnt\mdseries\color{grey}}{\thesubsection}{1em}{}

\titlespacing*{\chapter} {0pt}{50pt}{40pt}
\titlespacing*{\section} {0pt}{0pt}{16pt}
\titlespacing*{\subsection} {0pt}{12pt}{8pt}



\newtcbtheorem[auto counter,number within=section]{theorem}%
  {Theorem}{
  		fonttitle=\upshape, 
  		fontupper=\upshape,
  		boxrule=0pt,
  		leftrule=3pt,
  		arc=0pt,auto outer arc,
  		colback=white,
  		colframe=pink,
  		colbacktitle=white,
  		coltitle=pink,
  		oversize,
  		enlarge top by=1mm,
  		enlarge bottom by=1mm,
    	enhanced jigsaw,
    	interior hidden, 
    	before skip=12pt,
    	overlay={
    		\draw[line width=1.5pt,pink] (frame.north west) -- (frame.south west);
  		}, 
  		frame hidden}{theorem}

\newtcbtheorem[auto counter,number within=section]{lemma}%
  {Lemma}{
  		fonttitle=\upshape, 
  		fontupper=\upshape,
  		boxrule=1pt,
  		toprule=0pt,
  		leftrule=3pt,
  		arc=0pt,auto outer arc,
  		colback=white,
  		colframe=yellow,
  		colbacktitle=white,
  		coltitle=yellow,
  		oversize,
  		enlarge top by=1mm,
  		enlarge bottom by=1mm,
    	enhanced jigsaw,
    	interior hidden, 
    	before skip=12pt,
    	overlay={
    		\draw[line width=1.5pt,yellow] (frame.north west) -- (frame.south west);
  		}, 
  		frame hidden}{lemma}
  		
 \newtcbtheorem[auto counter,number within=section]{definition}%
  {Definition}{
  		fonttitle=\upshape, 
  		fontupper=\upshape,
  		boxrule=1pt,
  		toprule=0pt,
  		leftrule=3pt,
  		arc=0pt,auto outer arc,
  		colback=white,
  		colframe=orange,
  		colbacktitle=white,
  		coltitle=orange,
  		oversize,
  		enlarge top by=1mm,
  		enlarge bottom by=1mm,
    	enhanced jigsaw,
    	interior hidden, 
    	before skip=12pt,
    	overlay={
    		\draw[line width=1.5pt,orange] (frame.north west) -- (frame.south west);
  		}, 
  		frame hidden}{definition}
    	
\newtcbtheorem[auto counter,number within=section]{example}%
  {Beispiel}{
  		fonttitle=\upshape, 
  		fontupper=\upshape,
  		boxrule=0pt,
  		leftrule=3pt,
  		arc=0pt,auto outer arc,
  		colback=white,
  		colframe=grey,
  		colbacktitle=white,
  		coltitle=grey,
  		oversize,
  		enlarge top by=1mm,
  		enlarge bottom by=1mm,
    	enhanced jigsaw,
    	interior hidden, 
    	before skip=12pt,
    	overlay={
    		\draw[line width=1.5pt,grey] (frame.north west) -- (frame.south west);
  		}, 
  		frame hidden}{example}
    	
\newtcbtheorem[auto counter,number within=section]{note}%
  {Notiz}{
  		fonttitle=\upshape, 
  		fontupper=\upshape,
  		boxrule=0pt,
  		leftrule=3pt,
  		arc=0pt,auto outer arc,
  		colback=white,
  		colframe=yellow,
  		colbacktitle=white,
  		coltitle=yellow,
  		oversize,
  		enlarge top by=1mm,
  		enlarge bottom by=1mm,
    	enhanced jigsaw,
    	interior hidden, 
    	before skip=12pt,
    	overlay={
    		\draw[line width=1.5pt,yellow] (frame.north west) -- (frame.south west);
  		}, 
  		frame hidden}{note}
  		
\newtcbtheorem[]{important}%
  {Wichtig}{
  		fonttitle=\upshape, 
  		fontupper=\upshape,
  		boxrule=0pt,
  		leftrule=3pt,
  		arc=0pt,auto outer arc,
  		colback=white,
  		colframe=pink,
  		colbacktitle=white,
  		coltitle=pink,
  		oversize,
  		enlarge top by=1mm,
  		enlarge bottom by=1mm,
    	enhanced jigsaw,
    	interior hidden, 
    	before skip=12pt,
    	overlay={
    		\draw[line width=1.5pt,pink] (frame.north west) -- (frame.south west);
  		}, 
  		frame hidden}{important}
    	
\renewcommand{\baselinestretch}{1.4} 
\makeatletter
\let\old@rule\@rule
\def\@rule[#1]#2#3{\textcolor{lightgrey}{\old@rule[#1]{#2}{#3}}}
\makeatother

\begin{document}
\section*{Analysis III Problem Sheet 04}
\textit{Viet Duc Nguyen (395220), Moritz Bichlmeyer (392374)}

\noindent\textit{Tutor: Fabian - Donnerstag 12-14 Uhr}


\subsection*{Exercise 2}
\begin{enumerate}[label=(\roman*)]
	\item Seien $A_k \in \mathcal A$ Nullmengen in einem Maßraum $(\Omega, \mathcal A, \mu)$. Dann ist $\bigcup_{k \in \mathbb N} A_k$ auch eine Nullmenge. Wir wissen, es gilt, dass $\mu(A_k) = 0$ für alle $k \in \mathbb N$. Auch ist $\bigcup A_k \in \mathcal A$ wegen der Abgeschlossenheit gegenüber der Vereinigung. Mit der Subadditivität folgt, dass $0 \leq \mu(\bigcup A_k) \leq \sum \mu(A_k) \leq 0$. Daher $\mu(\bigcup A_k) = 0$. Somit ist $\bigcup A_k$ eine Nullmenge.
	
	\item Sei $N \in \mathcal A_k$ eine Nullmenge bzgl. des Lebesguemaßes. Wir wollen zeigen, dass $\mathbb R^n \setminus N$ dicht in $\mathbb R^n$ liegt. Sei dazu $x \in \mathbb R^n$. 
	
	Falls $x \notin N$ so ist $x \in \mathbb R^n \setminus N$. Also gilt für jedes $\epsilon > 0$, dass $U_{\epsilon}(x) \cap \mathbb R^n \setminus N \neq \emptyset$ wegen $x \in U_{\epsilon}(x)$ und $x \in \mathbb R^n \setminus N$.
	
	Falls $x \in N$, so ist $x \notin \mathbb R^n \setminus N$. Wir machen einen Widerspruchsbeweis. Angenommen, es gäbe ein $\epsilon > 0$, sodass $U_{\epsilon}(x) \cap \mathbb R^n \setminus N = \emptyset$. Also $U_\epsilon(x) \subset N$.
	
	 \textbf{Behauptung:} Dann muss für jede elementargeometrische Menge $E \subset \mathbb R^n$ mit $U_{\epsilon}(x)  \subset E$ gelten, dass $\lambda(E) > 0$. 
	 
	 Dazu betrachte ein Quadrat $Q \subset U_{\epsilon}(x)$ mit Länge $\frac{\epsilon}{2}$. Dieses Quadrat hat ein Maß $\lambda(Q) = \frac{\epsilon^2}{4} > 0$. Nun gilt aufgrund der Subadditivität von $\lambda$, dass $\lambda(E) \geq \lambda(Q) = \frac{\epsilon^2}{4} > 0$. 
	 
	 Für jede Überdeckung $\bigcup U_i \supset N$, wobei $U_i$ elementargeometrisch sind, muss dann auch $U_\epsilon(x) \subset \bigcup U_i$ gelten. Dann ist wegen der Subadditivität $$\sum \lambda(U_i) \geq \lambda(Q) = \frac{\epsilon^2}{4} > 0.$$ Somit ist das äußere Maß $\lambda^*(N) > 0$. Widerspruch, denn $N$ ist eine Nullmenge. Also liegt $\mathbb R^n \setminus N$ dicht in $\mathbb R^n$.
	 
	 \item Wir zeigen, dass $G_f(i) \coloneqq \{ (x,f(x)) : x \in [i, i+1]^d \}$ eine Nullmenge in $\mathbb R^{d+1}$ bzgl. des Lebesguemaßes ist. Wie in (a) gezeigt, ist dann auch
	 \[
	 	\bigcup_{i \in \mathbb N}( G_f(i) \cup G_f(-i)) = G_f
	 \]
	 als abzählbare Vereinigung von Nullmengen wieder eine Nullmenge.
	 
	 Sei $i \in \mathbb Z$ beliebig. $f$ ist als stetige Funktion auf dem Kompaktum $A \coloneqq [i, i+1]^d$ gleichmäßig stetig. Sei $\epsilon > 0$. Also gibt es ein $\delta(\epsilon) > 0$, sodass für alle $x,y \in A$ mit $||x-y|| < \delta(\epsilon)$ gilt, dass
	 \[
	 	|f(x)-f(y)| < \epsilon.
	 \]
Wir unterteilen das Intervall $[i,i+1]^d$ so in paarweise disjunkte Quader $(J_n)_{n = 1,...,m}$, sodass $$\forall n \in \{1,...,m\}: J_n \subset [i,i+1]^d, \quad \bigcup_{n=1}^m J_n = [i,i+1]^d$$ und $$\forall n \in \{1,...,m\}: \lambda(J_n) \leq \delta(\epsilon).$$ Betrachte die Familie von Quadern $(R_n)_{n = 1,...,m}$ mit $R_n = J_n \times U_{2\epsilon}(f(x))$ für ein beliebiges $x \in J_n$. Dann gilt also, dass
\[
	\lambda(\bigcup^m_{n=1} R_n) \leq \sum^m_{n=1}\lambda(R_n) = 2\epsilon \sum_{n=1}^m |J_n| = 2\epsilon.
\]
Wir haben also eine elementargeometrische Überdeckung von $G_f(i)$ gefunden, sodass das Maß der Überdeckung beliebig klein werden kann. Also ist $G_f(i)$ eine Nullmenge, da $\inf\{ \sum \lambda(R_i) : \bigcup R_n \supset G_f(i), R_n \in \mathcal R \} \leq 2\epsilon$ für alle $\epsilon > 0$.

\end{enumerate}


\subsection*{Exercise 3}
Sei $n \geq 2$. $M_1$ ist Lebesgue-messbar, denn die Menge $M_1$ ist abgeschlossen. Somit ist es eine Borelmenge und alle Borelmengen sind Lebesgue-messbar. Man sieht ganz leicht, dass $M_1$ abgeschlossen ist, da für jedes $x = (x_1,...,x_n)$ mit $x_i \neq x_j$ für irgendein $i \neq j$ gilt, dass $U_\epsilon(x) \cap M_1 = \emptyset$, indem man $\epsilon = \frac{d}{2}$ wählt, wobei $d > 0$ der Abstand von $x \notin M_1$ zur Geraden $M_1$ ist. 

$M_1$ ist eine Nullmenge, da $M_1$ eine Teilmenge des Graphen von $h: \mathbb R^{n-1} \to \mathbb R, (x_1,...,x_{n-1}) \mapsto x_1$  in  $\mathbb R^n$ aufgefasst werden kann. Mit Aufgabe 2(iii) ergibt sich, dass der Graph von $h$ eine Nullmenge ist und da jede Teilmenge einer Nullmenge in einem vollständigen Maßraum wieder eine Nullmenge ist, ergibt sich dann die Behauptung, denn das Lebesguemaß ist vollständig.

Hier sind wir unsicher: Falls $x=(x_1,...,x_n) \in M_2$ mit $x_1 \in \mathbb Q$, so sind alle anderen Komponenten $x_j, j \geq 2$ ebenfalls rational. Ansonsten wäre nämlich $x_1 - x_j \notin Q$. Das heißt, $\mathbb Q^n \cap [0,1]^n \subset M_2$. $\mathbb Q^n$ ist Lebesgue-messbar, denn $\mathbb Q^n$ ist abzählbar. 

 Falls $x=(x_1,...,x_n) \in M_2$ mit $x_1 \in \mathbb R$, so ist $x_j = \lambda x_1$ für alle $\lambda \in \mathbb Q$, da ansonsten $x_1 - x_j \notin \mathbb Q$. Das heißt, $$L = \{ ( x, \lambda_1 x,..., \lambda_{n-1}x )  : x \in \mathbb R \setminus \mathbb Q, \lambda_i \in \mathbb Q \} \cap [0,1]^n \subset M_2.$$
Wir sehen, dass $L$ eine Borelmenge ist, da $$L = ([0,1]^n \setminus ((\mathbb R \setminus \mathbb Q) \times \mathbb Q^{n-1})) \cap (\mathbb R \setminus \mathbb Q \times \mathbb R^{n-1}).$$ Somit ist $L$ messbar. Nun ist $$\Gamma = (\mathbb R \setminus \mathbb Q) \times \mathbb Q^{n-1}$$ eine Nullmenge bzgl des vollständigen Lebesguemaßes, da $$\Gamma \subset \bigcup_{x \in \mathbb Q} \mathbb R^{n-1}\times \{ x\}$$ und $\mathbb R^{n-1}\times \{ x\}$ ist eine Nullmenge, wie wir im Tutorium gezeigt haben. Also ist $L \cup \Gamma = [0,1]^n \cap ((\mathbb R \setminus \mathbb Q) \times \mathbb R^{n-1})$. Die Menge $(\mathbb R \setminus \mathbb Q) \times \mathbb R^{n-1}$, welche eine Borelmenge ist, liegt dicht in $\mathbb R^n$. Es hat also das Lebesguemaß $$\lambda([0,1]^n \cap (\mathbb R \setminus \mathbb Q) \times \mathbb R^{n-1}) = 1,$$ da das Lebesguemaß die Vervollständigung des Lebesgue-Borelmaßes ist. Also $\lambda(L) = 1$, da $\Gamma$ eine Nullmenge ist.
 
 Insgesamt ist $M_2$ als Vereinigung zweier messbarer Mengen $$M_2 = (\mathbb Q^n \cap [0,1]^n )\cup L$$ wieder messbar, denn der Raum der messbaren Mengen ist eine $\sigma$-Algebra. Damit $\lambda(M_2) = 1$.
\end{document}