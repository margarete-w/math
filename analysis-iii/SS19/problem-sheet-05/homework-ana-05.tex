\documentclass[a4paper]{article}

\usepackage{fontspec}
\setmainfont[Ligatures=TeX]{Georgia}
\setsansfont[BoldFont="HelveticaNeue-Medium"]{Helvetica Neue}

\usepackage{amsmath, amsthm, amssymb}
\usepackage{mathtools}
\usepackage[most]{tcolorbox}
\usepackage{blindtext}
\usepackage{xcolor}
\usepackage{titlesec}
\usepackage{titling}
\usepackage{enumitem}% http://ctan.org/pkg/enumitem


\newfontfamily\Menlo[Ligatures=TeX]{Menlo}

\definecolor{grey}{rgb}{0.5,0.5,0.5}
\definecolor{lightgrey}{rgb}{0.9,0.9,0.9}
\definecolor{darkgrey}{rgb}{0.3,0.3,0.3}
\definecolor{orange}{rgb}{0.94, 0.55, 0.294}
\definecolor{pink}{rgb}{0.94, 0.29, 0.7}
\definecolor{yellow}{rgb}{1, 0.749, 0}

\newcommand{\chapfnt}{\fontsize{16}{19}}
\newcommand{\secfnt}{\fontsize{18}{17}}
\newcommand{\ssecfnt}{\fontsize{14}{14}}
\renewcommand{\hline}{\noindent\makebox[\linewidth]{\rule{12cm}{1pt}}}
\newcommand{\code}[1]{{\Menlo{\color{darkgrey}#1}}}
\newcommand{\vip}[1]{\textit{\textbf{#1}}}

\titleformat{\chapter}[display]
{\normalfont\chapfnt\bfseries}{\chaptertitlename\ \thechapter}{20pt}{\chapfnt}

\titleformat{\section}
{\normalfont\sffamily\secfnt\mdseries}{\thesection}{1em}{}

\titleformat{\subsection}
{\normalfont\sffamily\ssecfnt\mdseries\color{grey}}{\thesubsection}{1em}{}

\titlespacing*{\chapter} {0pt}{50pt}{40pt}
\titlespacing*{\section} {0pt}{0pt}{16pt}
\titlespacing*{\subsection} {0pt}{12pt}{8pt}



\newtcbtheorem[auto counter,number within=section]{theorem}%
  {Theorem}{
  		fonttitle=\upshape, 
  		fontupper=\upshape,
  		boxrule=0pt,
  		leftrule=3pt,
  		arc=0pt,auto outer arc,
  		colback=white,
  		colframe=pink,
  		colbacktitle=white,
  		coltitle=pink,
  		oversize,
  		enlarge top by=1mm,
  		enlarge bottom by=1mm,
    	enhanced jigsaw,
    	interior hidden, 
    	before skip=12pt,
    	overlay={
    		\draw[line width=1.5pt,pink] (frame.north west) -- (frame.south west);
  		}, 
  		frame hidden}{theorem}

\newtcbtheorem[auto counter,number within=section]{lemma}%
  {Lemma}{
  		fonttitle=\upshape, 
  		fontupper=\upshape,
  		boxrule=1pt,
  		toprule=0pt,
  		leftrule=3pt,
  		arc=0pt,auto outer arc,
  		colback=white,
  		colframe=yellow,
  		colbacktitle=white,
  		coltitle=yellow,
  		oversize,
  		enlarge top by=1mm,
  		enlarge bottom by=1mm,
    	enhanced jigsaw,
    	interior hidden, 
    	before skip=12pt,
    	overlay={
    		\draw[line width=1.5pt,yellow] (frame.north west) -- (frame.south west);
  		}, 
  		frame hidden}{lemma}
  		
 \newtcbtheorem[auto counter,number within=section]{definition}%
  {Definition}{
  		fonttitle=\upshape, 
  		fontupper=\upshape,
  		boxrule=1pt,
  		toprule=0pt,
  		leftrule=3pt,
  		arc=0pt,auto outer arc,
  		colback=white,
  		colframe=orange,
  		colbacktitle=white,
  		coltitle=orange,
  		oversize,
  		enlarge top by=1mm,
  		enlarge bottom by=1mm,
    	enhanced jigsaw,
    	interior hidden, 
    	before skip=12pt,
    	overlay={
    		\draw[line width=1.5pt,orange] (frame.north west) -- (frame.south west);
  		}, 
  		frame hidden}{definition}
    	
\newtcbtheorem[auto counter,number within=section]{example}%
  {Beispiel}{
  		fonttitle=\upshape, 
  		fontupper=\upshape,
  		boxrule=0pt,
  		leftrule=3pt,
  		arc=0pt,auto outer arc,
  		colback=white,
  		colframe=grey,
  		colbacktitle=white,
  		coltitle=grey,
  		oversize,
  		enlarge top by=1mm,
  		enlarge bottom by=1mm,
    	enhanced jigsaw,
    	interior hidden, 
    	before skip=12pt,
    	overlay={
    		\draw[line width=1.5pt,grey] (frame.north west) -- (frame.south west);
  		}, 
  		frame hidden}{example}
    	
\newtcbtheorem[auto counter,number within=section]{note}%
  {Notiz}{
  		fonttitle=\upshape, 
  		fontupper=\upshape,
  		boxrule=0pt,
  		leftrule=3pt,
  		arc=0pt,auto outer arc,
  		colback=white,
  		colframe=yellow,
  		colbacktitle=white,
  		coltitle=yellow,
  		oversize,
  		enlarge top by=1mm,
  		enlarge bottom by=1mm,
    	enhanced jigsaw,
    	interior hidden, 
    	before skip=12pt,
    	overlay={
    		\draw[line width=1.5pt,yellow] (frame.north west) -- (frame.south west);
  		}, 
  		frame hidden}{note}
  		
\newtcbtheorem[]{issue}%
  {To prove}{
        theorem name,
  		fonttitle=\upshape, 
  		fontupper=\upshape,
  		boxrule=0pt,
  		leftrule=3pt,
  		arc=0pt,auto outer arc,
  		colback=white,
  		colframe=pink,
  		colbacktitle=white,
  		coltitle=pink,
  		oversize,
  		enlarge top by=1mm,
  		enlarge bottom by=1mm,
    	enhanced jigsaw,
    	interior hidden, 
    	before skip=12pt,
    	after skip=0pt,
    	overlay={
    		\draw[line width=1.5pt,pink] (frame.north west) -- (frame.south west);
  		}, 
  		frame hidden}{issue}
    	
\renewcommand{\baselinestretch}{1.4} 
\makeatletter
\let\old@rule\@rule
\def\@rule[#1]#2#3{\textcolor{lightgrey}{\old@rule[#1]{#2}{#3}}}
\makeatother

\begin{document}
\section*{Analysis III Problem Sheet 05}
\textit{Viet Duc Nguyen (395220), Moritz Bichlmeyer (392374)}

\noindent\textit{Tutor: Nils - Freitag 12-14 Uhr, MA551}

\subsection*{Exercise 1}

\begin{enumerate}[label=(\roman*)]
\item Sei $(\Omega, \mathcal A)$ ein messbarer Raum und $f: \Omega \to \mathbb R$ messbar. Definiere die Funktion $g: \Omega \to \mathbb R$ als
\[
	g(x) = \begin{cases}
		\frac{1}{f(x)} \quad &\text{falls $f(x) \neq 0$} \\
		0 & \text{sonst}
	\end{cases}.
\]
\begin{issue}{}{}
	$g$ ist eine messbare Funktion.
\end{issue}
\begin{proof}
Dazu zeigen wir, dass $g^{-1}((c, \infty)) \in \mathcal A$ für alle $c \in \mathbb R$ gilt.
\begin{itemize}
\item \textbf{Fall 1:} $c \geq 0$. Dann ist 
\begin{align*}
g^{-1}((c, \infty)) &= \{ x: f(x) = \frac{1}{d} \text{ für ein $d \in (c,\infty)$} \} \\
&= \{ x: f(x) \in (0, \frac{1}{c}) \}.
\end{align*}
Beachte, dass $c \neq 0$, sodass die Menge in der zweiten Zeile wohldefiniert ist. Daher ist $g^{-1}((c, \infty)) = f^{-1}((0, \frac{1}{c})) \in \mathcal A$.

\textbf{Fall 2:} $c \leq 0$. Dann ist 
\begin{align*}
g^{-1}((c, \infty)) = g^{-1}((c, 0)) \cup g^{-1}((0, \infty)) \cup g^{-1}(\{0\}).
\end{align*}
Nun gilt $ g^{-1}(\{0\}) = f^{-1}(\{ 0 \}) \in \mathcal A$, da $f$ messbar ist. Außerdem haben wir
\begin{align*}
 g^{-1}((c, 0)) &=  \{ x: f(x) = \frac{1}{d} \text{ für ein $d \in (c,0)$} \} \\
 &=  \{ x: f(x) = d \text{ für ein $d \in (-\infty, \frac{1}{c})$} \} \\
 &= f^{-1}((-\infty, \frac{1}{d})) \in \mathcal A.
\end{align*}
Aus Fall 1 wissen wir, dass $ g^{-1}((0, \infty)) \in \mathcal A$. Damit ist $$g^{-1}((c, \infty)) \in \mathcal A$$ als Vereinigung messbarer Mengen in $\mathcal A$. Somit ist $g$ eine $\mathcal A$-messbare Funktion.
\end{itemize}
\end{proof}

\item Sei $a \in \mathbb R$. Es gilt
\[
	(\sup_{n \in \mathbb N} f_n)^{-1}((a,\infty]) = \bigcup_{n \in \mathbb N} f_n^{-1}(a,\infty] \in \mathcal A.
\]
Dann ist auch $\inf_{n \in \mathbb N} f_n$ messbar, da $\inf_{n \in \mathbb N} f_n = - \sup_{n \in \mathbb N}(-f_n)$.

Es gilt $$\limsup_{n \to \infty} f_n = \inf_{m \in \mathbb N} g_m$$ mit $g_m = \sup_{n \geq m}f_n$. Da $g_m$ messbar ist, ist auch $\inf_{m \in \mathbb N} g_m$ messbar und somit $\limsup_{n \to \infty} f_n$ messbar

Ebenso kann man zeigen, dass $\liminf_{n \to \infty} f_n$ messbar ist, indem,
\[
	\liminf_{n \to \infty} f_n = \sup_{m \in \mathbb N}g_m
\]
mit $g_m = \inf_{n \geq m}f_n$. 

Daher ist $f = \lim_{n \to \infty}f_n$ ebenfalls messbar, falls $f_n$ für alle $n \in \mathbb N$ messbar ist, da $f = \limsup_{n \to \infty}f_n = \liminf_{n \to \infty}f_n$.


\item Sei $f: \mathbb R \to \mathbb R$ differenzierbar. 
\begin{issue}{}{}
$f$ ist messbar.
\end{issue}
\begin{proof}
Aus der Vorlesung wissen wir, dass stetige Funktionen messbar sind \emph{(siehe Beispiel 1.44 (2) im Skript von Mehl, Kapitel 1, Maß- und Integrationstheorie)}. Da $f$ differenzierbar ist, ist $f$ insbesondere auch stetig. Daher ist $f$ messbar.
\end{proof}

\begin{issue}{}{}
$f'$ ist messbar.
\end{issue}
\begin{proof}
Es gilt
\[
	f'(x) = \lim_{h \to 0} \frac{f(x+h) - f(x)}{h}, \qquad \forall x \in \mathbb R.
\]
Definiere die Funktionenfolge $f_n(x) \coloneqq \frac{f(x+\frac{1}{n}) - f(x)}{\frac{1}{n}}$ für alle $n \in \mathbb N$. Dann konvergiert $f_n \to f'$ punktweise. Insbesondere ist $f_n$ für alle $n \in \mathbb N$ messbar, da $f$ messbar ist.
\end{proof}
\end{enumerate}



\subsection*{Exercise 2}
\begin{enumerate}[label=(\roman*)]
\item Sei $(\Omega, \mathcal A, \mu)$ ein vollständiger Maßraum. Seien $f$ und $g$ numerische Funktionen mit $f(x) =g(x)$ für alle $x \in \Omega \setminus N$ mit $\mu(N) = 0$.
\begin{issue}{}{}
$f$ ist messbar genau dann, wenn $g$ messbar ist.
\end{issue}
\begin{proof}
Sei $f$ messbar. Wir wollen zeigen, dass $g$ messbar ist. Sei $X \in \mathcal A$. Wir müssen zeigen, dass $g^{-1}(X) \in \mathcal A$.

Schreibe die Menge $g^{-1}(X)$ als
\[
	g^{-1}(X) = \underbrace{\{ x \in \Omega : g(x) \in X, f(x) = g(x) \}}_{\coloneqq A} \cup \underbrace{\{ x \in \Omega : g(x) \in X, f(x) \neq g(x) \}}_{\coloneqq B}.
\]
Beachte, dass $B \subset N$ und da $\mu$ ein vollständiges Maß ist, ist $B$ als Teilmenge einer Nullmenge ebenfalls eine Nullmenge. Somit ist $B$ messbar, denn Nullmengen sind messbar. Wir erhalten also
\[
	B \in \mathcal A.
\]
Zudem ist $$A = \underbrace{f^{-1}(X)}_{\in \mathcal A} \cap \{ x \in \Omega : f(x) = g(x) \}.$$
Wir sehen auch, dass $\{ x \in \Omega : f(x) = g(x) \} = \Omega \setminus N  \in \mathcal A$ messbar ist, da $\Omega$ und $N$ messbar sind und $\mathcal A$ unter Differenzbildung abgeschlossen ist. Damit ist $$A \in \mathcal A,$$ da $A$ als Schnitt zweier messbarer Mengen darstellbar ist.
\end{proof}

\item Nicht bearbeitet.
\end{enumerate}

\end{document}