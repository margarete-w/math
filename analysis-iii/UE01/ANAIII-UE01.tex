\documentclass[a4paper, landscape,twocolumn,fontsize=8pt,DIV=1]{scrartcl}

\usepackage[utf8]{inputenc}
\usepackage[ngerman]{babel}     %Wortdefinitionen
\usepackage{mathtools,amssymb,amsthm}
\usepackage{newpxtext} \usepackage[euler-digits]{eulervm}

\usepackage{geometry}
\usepackage{fancyhdr} % Kopfzeile
\usepackage{accents}
\usepackage{skak}
\usepackage{enumitem}
\usepackage{framed}
\usepackage{physics}
\usepackage{ulem}

% benutzerdefinierte Kommandos
\newcommand{\crown}[1]{\overset{\symking}{#1}}
\newcommand{\xcrown}[1]{\accentset{\symking}{#1}}

\makeatletter
\newcommand*{\rom}[1]{\expandafter\@slowromancap\romannumeral #1@}
\makeatother
%
\theoremstyle{plain}
\newtheorem{lemma}{Lemma}
\newtheorem*{satz}{Satz}
\newtheorem*{zz}{Zu zeigen}
\newtheorem*{formel}{Formel}


% Style
\setlength{\columnsep}{8mm}
\geometry{
 left=8mm,
 right=8mm,
 bottom=16mm,
 top = 16mm
}

% Kopfzeile
\pagestyle{fancy}
\fancyhf{}
\rhead{395220, 391511}
\lhead{\textbf{Analysis III}}
\cfoot{Seite \thepage}

\setlength\parindent{0pt}


\begin{document}

\section*{Aufgabe 3}
\begin{enumerate}[label=(\alph*)]
    \item 
    \begin{proof}
    Sei $I \in \mathbb R$ wie in der Aufgabenstellung definiert. Sei $\Gamma: [0,\infty) \to \mathbb R, x \mapsto \int^\infty_x \frac{1}{u^2(t)}dt$. $\Gamma$ ist auf seinem Definitionsbereich definiert, denn es gilt 
    \[
        \Gamma(x) = I - \int^x_0 \frac{1}{u^2(t)}dt, \quad \forall x \in \mathbb R_{\geq 0}.
    \]
    Insbesondere existiert $\int^x_0 \frac{1}{u^2(t)}dt \in \mathbb R$ für beliebige $x \in \mathbb R_{\geq0}$, da $\frac{1}{u^2}$ über jedes Intervall $[0,\alpha], \alpha \in \mathbb R$ integrierbar sein muss (notwendige Bedingung für die Existenz des uneigentlichen Integrals).
    
    Sei $u: [0,\infty) \to \mathbb R, x \mapsto u(x)$ eine Lösung von $(\star\star)$, das heißt
    \begin{align}\label{xD}
        u'' + pu = 0.
    \end{align}
    Sei $b = u\Gamma$. Die Ableitungen lauten:
    \begin{align*}
        b' &= u'\Gamma + u\Gamma' = u'\Gamma - u\frac{1}{u^2} = u'\Gamma - \frac{1}{u} \\
        b'' &= u'' \Gamma + u' \Gamma' + \frac{u'}{u^2} = u''\Gamma -\frac{u'}{u^2} + \frac{u'}{u^2} = u'' \Gamma.
    \end{align*}
    
    Verifiziere den Ansatz $b = u\Gamma$ für $(\star\star)$. 
    \begin{align*}
        b'' +pb = u'' \Gamma + p u\Gamma = \Gamma(u'' +pu) \overset{\eqref{xD}}{=} \Gamma \cdot 0 = 0. 
    \end{align*}
    \end{proof}
    
    \item \begin{proof}
        Sei $\Psi: [0,\infty) \to \mathbb R, x \mapsto \int^x_0 \frac{1}{u^2(t)}dt$. Die Funktion $\Psi$ ist auf seinem Definitionsbereich definiert, da $\int^x_0 \frac{1}{u^2(t)}dt$ für jedes $x \in [0, \infty)$ existiert aufgrund von $I = \lim_{x \to \infty} \int^x_0\frac{dt}{u^2(t)} = \infty$ (um den Grenzwert überhaupt bilden zu können, muss $\frac{1}{u^2}$ auf $[0,\alpha], \alpha \in \mathbb R$ integrierbar sein).
        
        Sei $w \coloneqq u\Psi$, wobei $u$ wie in Aufgabe 3(a) definiert ist.
        \begin{align*}
            w' &= u' \Psi + u \Psi' = u'\Psi + \frac{u}{u^2} = u'\Psi + \frac{1}{u} \\
            w'' &= u'' \Psi + u'\Psi' - \frac{u'}{u^2} = u'' \Psi + \frac{u'}{u^2} - \frac{u'}{u^2} = u'' \Psi.
        \end{align*}
        Verifiziere den Ansatz $w = u\Psi$ für $(\star\star)$. 
        \begin{align*}
            w'' +pw = u'' \Psi + p u\Psi = \Psi(u'' +pu) \overset{\eqref{xD}}{=} \Psi \cdot 0 = 0. 
        \end{align*}
    \end{proof}
    
    
    \item Sei $u$ eine Lösung mit $I \coloneqq \int^\infty_0\frac{dx}{u^2(x)} \in \mathbb R_{>0} \cup \{ \infty \}$. 
    
    \begin{itemize}
        \item Wie in Aufgabe (a) und (b) gezeigt, existieren zwei Lösungen $u$ und $u \Phi$ mit $$\Phi(x) \coloneqq \begin{cases} \Gamma(x), &\quad \text{ falls $I < \infty$} \\ \Psi(x), &\quad \text{ falls $I = \infty$.} \end{cases}$$.
        
        \item $u$ ist nach Voraussetzung positiv. $u \Phi$ ist ebenfalls positiv, denn $\Phi$ ist positiv. Das sieht man wie folgt: $u$ muss eine stetige Funktion sein, sonst würde es nicht die Differentialgleichung lösen. Daher ist auch $\frac{1}{u^2}$ stetig wegen $u(x) \neq 0$ für alle $x \in \mathbb R_{>0}$. Daher ist das Integral von $\frac{1}{u^2}$ auf einem beliebigen Intervall $[\alpha, \beta] \subset \mathbb R_{\geq 0}$ positiv nach dem Mittelwertsatz der Integralrechnung.
        \begin{align*}
            \underbrace{\frac{1}{u(\xi)^2}}_{> 0}\underbrace{\frac{1}{(\beta - \alpha)}}_{>0} = \int^\beta_\alpha\frac{1}{u^2(t)}dt, \quad \xi \in [\alpha, \beta]
        \end{align*}
        
        \item Falls $I < \infty$:
        \begin{gather*}
            y_1 \coloneqq u\Phi, y_1' = u'\Phi - \frac{1}{u}, y_2 \coloneqq u, y_2' = u'\\
            y_1y_2'-y_1'y_2 = uu'\Phi - (uu'\Phi -1) = 1.
        \end{gather*}
        Falls $I = \infty:$ 
        \begin{gather*}
            y_1 \coloneqq u, y_1' = u', y_2 \coloneqq u\Phi, y_2' = u'\Phi + \frac{1}{u}\\
            y_1y_2'-y_1'y_2 = uu'\Phi +1 - u'u\Phi = 1.
        \end{gather*}

        \item Falls $I < \infty$: $y_1 \coloneqq u\Phi,  y_2 \coloneqq u$.
        \begin{gather*}
            \Omega \coloneqq \frac{y_1}{y_2} = \Phi = \Gamma \\
            \Omega' = \Gamma' = -\frac{1}{u^2}  < 0.
        \end{gather*}
        
        Falls $I = \infty$: $y_1 \coloneqq u,  y_2 \coloneqq u\Phi$.
        \begin{gather*}
            \Omega \coloneqq \frac{y_1}{y_2} = \frac{u}{u\Phi}  = \frac{1}{\Psi} \\
            \Omega' = -\frac{\Psi'}{\Psi^2} = - \underbrace{\frac{1}{u^2}}_{>0} \underbrace{\frac{1}{\Psi^2}}_{>0} < 0
        \end{gather*}
        
        
        \item Falls $I < \infty$: $y_1 \coloneqq u\Phi,  y_2 \coloneqq u$. Es gilt $$\frac{y_1(x)}{y_2(x)} = \Gamma(x) = I - \int^x_0 \frac{1}{u^2(t)}dt.$$ Nun ist $\int^x_0 \frac{dt}{u^2(t)} \to I$ für $x \to \infty$. Also $\Gamma(x)\to 0$ für $x \to \infty$.
        
        Falls $I = \infty$: $y_1 \coloneqq u,  y_2 \coloneqq u\Phi$. Es gilt 
        $$\frac{y_1(x)}{y_2(x)} = \frac{1}{\Psi(x)} = \frac{1}{\int^x_0 \frac{1}{u^2(t)}dt}.$$
        Da für $x \to \infty$ die Funktion $\Psi(x)$ gegen $I = \infty$ läuft, ist $\frac{1}{\Psi(x)} \to 0$ für $x\to \infty$.
    \end{itemize}
\end{enumerate}

\end{document}
\frac{2}{9}(10+\ln0.5) - \frac{1}{3}(5-\ln0.5)