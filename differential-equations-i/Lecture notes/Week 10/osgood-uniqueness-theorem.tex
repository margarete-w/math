\documentclass[a4paper, 11pt]{article}

\usepackage[utf8]{inputenc}
\usepackage{amsmath,amsthm,amssymb}
\usepackage{mathtools}
\usepackage{geometry} 
\usepackage{marvosym}
\usepackage[toc,titletoc,title]{appendix}
\usepackage[hidelinks]{hyperref}
\usepackage{framed}
\usepackage{enumitem}
\usepackage{parskip}

\usepackage{xcolor}
\hypersetup{
	colorlinks,
	linkcolor={red!50!black},
	citecolor={red!50!black},
	urlcolor={red!50!black}
}

\makeatletter
\def\thm@space@setup{%
	\thm@preskip=5mm
	\thm@postskip=\thm@preskip % or whatever, if you don't want them to be equal
}
\makeatother

% bold title for optional title in theorems
\makeatletter
\def\th@plain{%
	\thm@notefont{}% same as heading font
	\itshape % body font
}
\def\th@definition{%
	\thm@notefont{}% same as heading font
	\normalfont % body font
}
\makeatother

\newtheorem{theorem}{Theorem}
\newtheorem*{theorem*}{Theorem}
\newtheorem{lemma}[theorem]{Lemma}
\newtheorem{collorary}[theorem]{Collorary}
\newtheorem{proposition}{Proposition}


\newtheorem{definition}[theorem]{Definition}
\newtheorem*{definition*}{Definition}
\newtheorem*{example}{Example}
\newtheorem*{remark}{Remark}

% roman number
\newcommand{\rom}[1]{\uppercase\expandafter{\romannumeral #1\relax}}



\begin{document}

\title{Differential Equations I (Week 10) }
\author{ Lecture by Hans-Christian Kreusler \\ 17th December - 21th December 2018 \\ Technical University Berlin}
\date{ Notes written by Viet Duc Nguyen}
\maketitle

\section*{Appendix: Osgood's Uniqueness Theorem}
Consider the following initial value problem stated as
\begin{align}\label{krass}
\begin{cases}
u'(t) = f(t,u(t)) \\ u(0) = u_0
\end{cases}
\end{align}
with $U \subset \mathbb R^d$ open and $f: [0,T] \times U \to \mathbb R^d \text{ and } u_0 \in \mathbb R^d$. We know that there exists a unique solution if the Cauchy-Lipschitz condition is satisfied. However, Osgood could prove that there exists a weaker condition that asserts the uniqueness of a solution for a local neighbourhood around $0$. For that, we need to define when functions	 satisfy the \emph{Osgood condition}. 

\begin{definition}
	A function $\omega: [0, \infty) \to [0,\infty)$ is said to satisfy \textbf{Osgood's condition} if $\omega(0) = 0$, $\omega(z) > 0$ for all $z > 0$ and for any $\delta > 0$ it holds
	\[
	\lim_{\epsilon \searrow 0} \int^{\delta}_{\epsilon} \frac{1}{\omega(r)}dr = \infty.
	\]
\end{definition}

\begin{theorem}
	Let $U \subset \mathbb R^d$ be open. Let $f: [0,T] \times U \to \mathbb R^d$ be continuous and let $\omega: [0,\infty) \to [0,\infty)$ satisfy Osgood's condition. If 
	\[
	|f(t,v) - f(t,u)| \leq \omega(|v-u|) \quad \forall t \in [0,T], \forall u,v \in U
	\]
	then for any $u_0 \in U$ there is a $\delta > 0$ for which there exists a unique solution $u: [0,\delta] \to U$ of the initial value problem defined in \eqref{krass}.
\end{theorem}


\end{document}