\documentclass[10pt,a4paper]{article}
\usepackage[utf8]{inputenc}
\usepackage[german]{babel}
\usepackage[T1]{fontenc}
\usepackage{amsmath}
\usepackage{amsfonts}
\usepackage{amssymb}
\usepackage{amsthm}
\usepackage{mathtools}
\usepackage{pgfplots}
\usepackage{enumerate}
\usepackage{enumitem}% http://ctan.org/pkg/enumitem

\usepackage[
  %showframe,% Seitenlayout anzeigen
  left=3cm,
  right=2cm,
  top=2.5cm,
  bottom=2cm,
  %includeheadfoot
]{geometry}
\title{DGL I, 7. Übungsblatt}
\author{Duc Nguyen (395220), Jan Walczak (371626)}
\date{}

% roman numbers
\makeatletter
\newcommand*{\rom}[1]{\expandafter\@slowromancap\romannumeral #1@}
\makeatother

\begin{document}
\maketitle

\section*{Aufgabe 1}
\begin{enumerate}[label=(\roman*)]
	\item Die Folge $F$ mit $f_n(x) = \arctan(nx)$ ist auf $[0,1]$ nicht gleichgradig stetig und somit auch nicht relativ kompakt. $F$ ist nicht gleichgradig stetig, da man für $\epsilon = 1$ immer ein $n(\delta) \in \mathbb N$ für jedes $\delta > 0$ findet, sodass es ein $0 < x \leq \delta$ gibt mit
	\begin{align}\label{las}
		\arctan(nx) \geq \epsilon. 
	\end{align}
	Sei $\delta > 0$. Nun gilt aufgrund der Monotonie von $\tan$ auf $[0,1]$:
	\[
		\arctan(nx) \geq \epsilon \iff nx \geq \tan(\epsilon) \iff n \geq \frac{\tan(\epsilon)}{x}.
	\]
	Daher sei $n(\delta) \coloneqq \frac{\tan(\epsilon)}{\delta}$ und man hat mit $x = \delta$ solch ein $x$ gefunden, das \eqref{las} erfüllt. Somit ist die Folge $F$ nicht gleichgradig stetig.
	
	\item Die Folge $F$ mit $f_n(x) = e^{-\frac{x}{n}}$ ist auf $[0,\infty)$ gleichgradig stetig. Man zeigt dafür, dass $\frac{d}{dx}e^{-\frac{x}{n}}$ für $n=1$ und $x=0$ sein Maximum mit $1$ annimmt. Daraus folgt, dass $f_n$ lipschitz stetig mit Lipschitz Konstante $1$ für alle $n \in \mathbb N$ ist. Daher ist $F$ gleichgradig stetig.
	
	Es gilt nun
	\[
		\frac{d}{dt}e^{-\frac{x}{n}} = -\frac{e^{-\frac{x}{n}}}{n} \implies \max{|-\frac{e^{-\frac{x}{n}}}{n}|} = 1,
	\]
	wobei das Maximum bei $n=1$ und $x=0$ angenommen wird.
	
	Auch ist die Folge $F$ relativ kompakt, da sie gleichmäßig beschränkt ist mit Konstante $c=1$. Es gilt
	\[
		\forall n \in \mathbb N,\forall x \in \mathbb R_{\geq 0}: e^{-\frac{x}{n}} = \frac{1}{e^{\frac{x}{n}}} \leq 1, 
	\]
	da $e^x \geq 1$ für alle $x \geq 0$. 
	
	\item Auch diese Folge ist gleichgradig stetig mit der selben Argumentation wie in (ii). Es gilt nämlich
	\[
		\frac{d}{dt}ne^{-\frac{x}{n}} = -e^{-\frac{x}{n}} \implies \max{|-e^{-\frac{x}{n}}|} = 1,
	\]
	für $x=0$. Aber $F$ ist nicht relativ kompakt, da $F$ nicht gleichmäßg beschränkt ist. Für jedes $c>0$ ist nämlich $(c+1)e^{-\frac{x}{c+1}} > c$ für $x=0$.
\end{enumerate}

\end{document}
