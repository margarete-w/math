\documentclass[fontsize=11pt, paper=a4]{scrartcl} 

\usepackage[english,ngerman]{babel}
\usepackage[utf8]{inputenc}
\usepackage{enumitem}
\usepackage{xstring}
\StrBehind*{\jobname}{solution}[\sheetno]
\usepackage[colorinlistoftodos,bordercolor=orange,backgroundcolor=orange!20,linecolor=orange,textsize=scriptsize]{todonotes}
\usepackage{amsfonts,amsmath,amssymb,amsthm}       
\usepackage{tikz}

\parindent0em
\parindent0em 

\newcommand{\norm}[1]{\left\lVert #1 \right\rVert}
\newcommand{\floor}[1]{\lfloor #1 \rfloor}
\newcommand{\betrag}[1]{\left\lvert #1 \right\rvert}
\newcommand{\eins}{ \mathds{1} } 
\newcommand{\anfzeichen}[1]{"`#1"'}
\newcommand{\entspricht}{\mathrel{\widehat{=}}}

\newcommand{\R}{\mathbb{R}}
\newcommand{\C}{\mathbb{C}}
\newcommand{\N}{\mathbb{N}}
\newcommand{\Q}{\mathbb{Q}}
\newcommand{\Z}{\mathbb{Z}}

\DeclareMathOperator{\diag}{diag}
\DeclareMathOperator{\grad}{grad}
\DeclareMathOperator{\konv}{konv}
\DeclareMathOperator{\cone}{cone}
\DeclareMathOperator{\ospan}{span}
\DeclareMathOperator{\aff}{aff}
\DeclareMathOperator{\rang}{rang}
\DeclareMathOperator{\rint}{rint}
\DeclareMathOperator{\signum}{signum}
\DeclareMathOperator{\diff}{diff}
\DeclareMathOperator{\ggt}{ggT}
\DeclareMathOperator{\ch}{ch}

\newcommand{\eps}{\varepsilon}
\providecommand{\dotcup}{\mathbin{\dot{\cup}}}
\providecommand{\comment}[1]{}
\def\dcup{\overset{.}{\cup}}
\newcommand{\statistisch}[1]{}
\newcommand{\pivot}[1]{\fcolorbox{red}{yellow}{#1}}

\newcommand{\vorlesung}{Discrete Geometry I Summer term 2020}


\newcommand{\pkt}[1]{
\ifnum #1=1
\textbf{(#1 Punkt)} \vspace{0.1cm}
\else
\textbf{(#1 Punkte)} \vspace{0.1cm}
\fi
}

\newcommand{\pfad}[1]{[#1]} \vspace{0.1cm}

\theoremstyle{definition}
\newtheorem{exercise}{Exercise}


\newtheorem{standardsolution}{standard solution for exercise}

\makeatletter
\g@addto@macro{\thm@space@setup}{\thm@headpunct{}}
\makeatother

\newcommand{\head}[1]{\clearpage
\setcounter{exercise}{0}
\setcounter{standardsolution}{0}
\setcounter{page}{1}
\hrule\vspace{3mm}

\textbf{\centering{\vorlesung}} \hfill Viet Duc Nguyen (395220)\\
Tutor: Ole or Holger \hfill  \\
Homework group 9 \hfill  
\begin{center}
{\Large \textbf{Exercise sheet #1}} 
\end{center}
\hrule

}


\newcommand{\grading}[1]{
\vspace{0.3cm}
\textcolor{red}{grading:} \hfill \framebox(28,14){\textcolor{red}{/#1}}
}

\newcommand{\ex}[2]{
\begin{exercise}\hfill \textbf{#2 Points}\\
\input{exercises/ex#1}
\end{exercise}
\textbf{our solution:}
}

\newcommand{\totalgrade}[1]{
\begin{center}
\hrule
\vspace{0.4cm}
\textbf{total grade:}

\vspace{0.1cm}
 \framebox(40,20){\textbf{\textcolor{red}{/#1}}}
\end{center}
}

\newcommand{\holger}[1]{\todo[size=\tiny,color=blue!20]{#1 \hfill --- H.E.}}
\newcommand{\Holger}[1]{\todo[size=\scriptsize,inline,color=blue!20]{#1 \hfill --- H.E.}}
\newcommand{\ole}[1]{\todo[size=\tiny,color=green!20]{#1 \hfill --- O.L.}}
\newcommand{\Ole}[1]{\todo[size=\scriptsize,inline,color=green!20]{#1 \hfill --- O.W.}}

\begin{document}

%------------------------------------------------------------------------

\head{\sheetno}
%------------------------------------------------------------------------

%%%%%%%%%%%%%%%%%%% exercise 1 %%%%%%%%%%%%%%%%%%% 
\ex{1}{4}

%% write solution for ex 1 here
\begin{enumerate}[label=(\alph*)]
    \item Consider two hyperplanes that have no point in common. The dimension of the intersection of both hyperplanes is zero.

    For the case that both hyperplanes $H_1$ and $H_2$ have a point in common in $\mathbb R^n$, note that one can shift any linear subspace to contain the origin without affecting its dimension. This is known as $\dim(a + V) = \dim(V)$ for any linear subspace $V$. The intersection of two hyperplanes results in an affine linear subspace that can be shifted to include the origin. For the sake of simplicity, let $H_1$ and $H_2$ denote the shifted hyperplanes whose intersection holds the origin. Then, the dimension formula which reads $\dim(H_1 \cap H_2) = \dim H_1 + \dim H_2 - \dim (H_1+H_2)$ holds. The Minkowski sum of $H_1$ and $H_2$ either results in $H_1$ if both hyperplanes are identical, or $\mathbb R^n$ otherwise. Thus, the dimension of $H_1 \cap H_2$ takes the value $2n - 2 - (n - 1) = n-1$ or $2n - 2 - n = n- 2$.

    All in all, possible values for $\dim(H_1 \cap H_2)$ are zero, $n-1$ or $n-2$.


    \item \textbf{Find $\dim(H_1 \cap ... \cap H_7)$.}

    The idea is to apply the previous result inductively on all the hyperplanes. First, all hyperplanes contain the origin, yet are distinct due to the linear independence of the normal vectors. Thus, each intersection of two hyperplanes shrinks the dimension by one as seen in the previous statement. Consequently, the dimension is $9 - 7 = 2$.
\end{enumerate}


\grading{4}

%%%%%%%%%%%%%%%%%%% exercise 2 %%%%%%%%%%%%%%%%%%% 
\ex{2}{2}

%% write solution for ex 2 here

\begin{itemize}
    \item $\mathrm{conv}(X+Y) \subset \mathrm{conv}(X) + \mathrm{conv}(Y)$: First, $X+Y \subset \mathrm{conv}(X) + \mathrm{conv}(Y)$. In addition, $\mathrm{conv}(X) + \mathrm{conv}(Y)$ is convex as the Minkowski sum of convex sets is again convex. Since $\mathrm{conv}(X+Y)$ is the smallest convex set containing $X+Y$, the inclusion stated above is proven.
    
    \item $\mathrm{conv}(X+Y) \supset \mathrm{conv}(X) + \mathrm{conv}(Y)$: Any point $x \in \mathrm{conv}(X)$ translated by a vector $y \in Y$ is in $\mathrm{conv}(X+Y)$ as it is shown below: 
    \begin{align*}
        x + y = \sum_{i=1}^k \alpha_i (x_i + y),
        \quad x_i \in X, \sum^k_{i=1} \alpha_i = 1.
    \end{align*}
    If one translates $x$ not by $y$ but instead by a convex combination of vectors $y_1,...,y_k$ from $Y$, one obtains 
    \begin{align*}
        x + \sum_{i=1}^k \alpha_i y_i 
        = \sum_{i=1}^k \alpha_i(x + y_i) 
        \in \mathrm{conv}\left(\mathrm{conv}(X+Y)\right) 
        = \mathrm{conv}(X+Y).
    \end{align*}
\end{itemize}

\grading{2}

%%%%%%%%%%%%%%%%%%% exercise 3 %%%%%%%%%%%%%%%%%%% 
\ex{3}{6}

%% write solution for ex 3 here
\begin{enumerate}[label=(\alph*)]
    \item \textbf{Statement:} If a set is closed, then its convex hull is also closed.

    \textbf{Answer:} This statement is \emph{wrong}. Consider the set that contains all members of the sequence $(\frac{1}{n})_{n \in \mathbb N}$. This set is closed. Its convex hull yields the interval from zero to one, including one but excluding zero. Here is the reason why it does not contain zero: every member of the sequence is strictly positive, and thus the convex combination must be strictly positive, as well.

    \item \textbf{Statement:} If a set is convex, then its closure is convex, too.
    
    \textbf{Answer:} This is \emph{true}. Consider two points $a$ and $b$ from the closure of $X$. Then, both $a$ and $b$ are the limits of two sequences $(a_n)$ and $(b_n)$ in $X$, respectively. We will see that any point $c = a + (b-a) \lambda$, where $\lambda \in (0,1)$, is in the closure of $X$, too. This follows from:
    \begin{align*}
        \underbrace{a_n + (b_n - a_n) \lambda}_{\in X} \to a + (b - a) \lambda = c \in \bar X, \quad n \to \infty.
    \end{align*}

    \item \textbf{Statement:} The convex hull of an open set is open.

    \textbf{Answer:} This statement is \emph{true}. Let $A \subset \mathbb R^n$ be open and $x \in \mathrm{conv}(A)$. We want to find an open neighborhood of $x$ that is a subset of $A$. Let $x = \sum^n_{i=1}\alpha_i x_i$ be a convex combination of $x$. Without loss of generality, let $\alpha_1 \neq 0$. Define the continuous function $\gamma_x: \mathbb R^n \to \mathbb R^n$ as
    $$
        \gamma_x(y): \frac{y - \sum^n_{i=2} \alpha_i x_i}{\alpha_1}.
    $$

    Its inverse function reads
    $$
        \gamma_x^{-1}(\tilde x) = \alpha_1 \tilde x + \sum^n_{i=2}\alpha_ix_i.
    $$

    We see that $\gamma_x(x) = x_1 \in A$. Therefore, $x \in \gamma_x^{-1}(A)$. So, we found an open neighborhood for $x$, namely $\gamma_x^{-1}(A)$.
\end{enumerate}


\grading{6}

%%%%%%%%%%%%%%%%%%% exercise 4 %%%%%%%%%%%%%%%%%%% 
\ex{4}{4}

%% write solution for ex 4 here
\begin{itemize}
    \item If the convex hull of the given points has four or more vertices, then select any four vertices, and we are done.
    \begin{center}
        \begin{tikzpicture}[scale=2.0]
            \draw[fill=black] (0,0) circle[radius=1pt] node[anchor=south west] {$x_1$};
            \draw[fill=black] (-0.5,-0.5) circle[radius=1pt] node[anchor=south east] {$x_2$};
            \draw[fill=black] (0.5,-0.5) circle[radius=1pt] node[anchor=south west] {$x_5$};
            \draw[fill=black] (0.25,-1) circle[radius=1pt] node[anchor=north west] {$x_4$};
            \draw[fill=black] (-0.25,-1) circle[radius=1pt] node[anchor=north east] {$x_3$};
            \draw (0,0) -- (-0.5,-0.5);
            \draw (0,0) -- (0.5,-0.5);
            \draw (-0.5,-0.5) -- (-0.25,-1);
            \draw (0.5,-0.5) -- (0.25,-1);
            \draw (0.25,-1) -- (-0.25,-1);
        \end{tikzpicture} 
    \end{center}
    

    \item In the other case, the convex hull has three vertices $v_1,v_2,v_3$, and two interior points $i_1,i_2$. The interior point $i_1$ partitions the triangle $\overline{v_1v_2v_3}$ into smaller subtriangles $\triangle_1, \triangle_2, \triangle_3$. The same for $i_2$; there are three smaller subtriangles $\tilde \triangle_1, \tilde \triangle_2, \tilde \triangle_3$. Then, $i_2$ must lie in one triangle $\triangle_i$, and $i_1$ must lie in one triangle $\tilde \triangle_j$ for some $i,j \in \{1,2,3\}$. Therefore, there are two vertices $v_k$ and $v_l$ left such that $\overline{v_kv_li_1} \neq \triangle_{i}$ and $\overline{v_kv_li_2} \neq \tilde \triangle_j$.
    
    The claim is that $\mathrm{conv}\{v_k, v_l, i_1, i_2\}$ is a convex set, i.e. a quadrilateral. If it were not a convex set, then $i_1$ or $i_2$ must be a convex combination of the other three points. But by construction and because of the fact that no three points lie on a line, this is not possible. Thus, $\mathrm{conv}\{v_k, v_l, i_1, i_2\}$ is a convex set.
        
    \begin{center}
        \begin{tikzpicture}[scale=2.0]
            \draw[fill=black] (0,0) circle[radius=1pt] node[anchor=south west] {$v_1$};
            \draw[fill=black] (-1,-1) circle[radius=1pt]node[anchor=south east] {$v_2$};
            \draw[fill=black] (1,-1) circle[radius=1pt] node[anchor=south west] {$v_3$};
            \draw[fill=black] (0.0,-0.6) circle[radius=1pt] node[anchor=south west] {$i_1$};
            \draw[fill=black] (-0.3,-0.9) circle[radius=1pt] node[anchor=south west] {$i_2$};
            \draw[] (0,0) -- (1,-1);
            \draw[] (0,0) -- (-1,-1);
            \draw[] (-1,-1) -- (1,-1);

            \draw[dashed] (0.0,-0.6) -- (0,0);
            \draw[dashed] (0.0,-0.6) -- (-1,-1);
            \draw[dashed] (0.0,-0.6) -- (1,-1);

            \draw[dashed] (-0.3,-0.9) -- (0,0);
            \draw[dashed] (-0.3,-0.9) -- (-1,-1);
            \draw[dashed] (-0.3,-0.9) -- (1,-1);
        \end{tikzpicture} 
    \end{center}
\end{itemize}

\grading{4}
\totalgrade{16}
\end{document}
