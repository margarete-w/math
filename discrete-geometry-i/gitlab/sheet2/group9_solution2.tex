\documentclass[fontsize=11pt, paper=a4]{scrartcl} 

\usepackage[english,ngerman]{babel}
\usepackage[utf8]{inputenc}
\usepackage{enumitem}
\usepackage{xstring}
\StrBehind*{\jobname}{solution}[\sheetno]
\usepackage[colorinlistoftodos,bordercolor=orange,backgroundcolor=orange!20,linecolor=orange,textsize=scriptsize]{todonotes}
\usepackage{amsfonts,amsmath,amssymb,amsthm}   
\usepackage{tikz}
\usetikzlibrary{patterns}

\parindent0em
\parindent0em 

\newcommand{\norm}[1]{\left\lVert #1 \right\rVert}
\newcommand{\floor}[1]{\lfloor #1 \rfloor}
\newcommand{\betrag}[1]{\left\lvert #1 \right\rvert}
\newcommand{\eins}{ \mathds{1} } 
\newcommand{\anfzeichen}[1]{"`#1"'}
\newcommand{\entspricht}{\mathrel{\widehat{=}}}

\newcommand{\R}{\mathbb{R}}
\newcommand{\C}{\mathbb{C}}
\newcommand{\N}{\mathbb{N}}
\newcommand{\Q}{\mathbb{Q}}
\newcommand{\Z}{\mathbb{Z}}

\DeclareMathOperator{\diag}{diag}
\DeclareMathOperator{\grad}{grad}
\DeclareMathOperator{\konv}{konv}
\DeclareMathOperator{\cone}{cone}
\DeclareMathOperator{\ospan}{span}
\DeclareMathOperator{\aff}{aff}
\DeclareMathOperator{\rang}{rang}
\DeclareMathOperator{\rint}{rint}
\DeclareMathOperator{\signum}{signum}
\DeclareMathOperator{\diff}{diff}
\DeclareMathOperator{\ggt}{ggT}
\DeclareMathOperator{\ch}{ch}

\newcommand{\eps}{\varepsilon}
\providecommand{\dotcup}{\mathbin{\dot{\cup}}}
\providecommand{\comment}[1]{}
\def\dcup{\overset{.}{\cup}}
\newcommand{\statistisch}[1]{}
\newcommand{\pivot}[1]{\fcolorbox{red}{yellow}{#1}}

\newcommand{\vorlesung}{Discrete Geometry I Summer term 2020}


\newcommand{\pkt}[1]{
\ifnum #1=1
\textbf{(#1 Punkt)} \vspace{0.1cm}
\else
\textbf{(#1 Punkte)} \vspace{0.1cm}
\fi
}

\newcommand{\pfad}[1]{[#1]} \vspace{0.1cm}

\theoremstyle{definition}
\newtheorem{exercise}{Exercise}


\newtheorem{standardsolution}{standard solution for exercise}

\makeatletter
\g@addto@macro{\thm@space@setup}{\thm@headpunct{}}
\makeatother

\newcommand{\head}[1]{\clearpage
\setcounter{exercise}{0}
\setcounter{standardsolution}{0}
\setcounter{page}{1}
\hrule\vspace{3mm}

\textbf{\centering{\vorlesung}} \hfill Viet Duc Nguyen (395220)\\
Tutor: Ole or Holger \hfill  \\
Homework group 9 \hfill  
\begin{center}
{\Large \textbf{Exercise sheet #1}} 
\end{center}
\hrule

}


\newcommand{\grading}[1]{
\vspace{0.3cm}
\textcolor{red}{grading:} \hfill \framebox(28,14){\textcolor{red}{/#1}}
}

\newcommand{\ex}[2]{
\begin{exercise}\hfill \textbf{#2 Points}\\
\input{exercises/ex#1}
\end{exercise}
\textbf{our solution:}
}

\newcommand{\totalgrade}[1]{
\begin{center}
\hrule
\vspace{0.4cm}
\textbf{total grade:}

\vspace{0.1cm}
 \framebox(40,20){\textbf{\textcolor{red}{/#1}}}
\end{center}
}

\newcommand{\holger}[1]{\todo[size=\tiny,color=blue!20]{#1 \hfill --- H.E.}}
\newcommand{\Holger}[1]{\todo[size=\scriptsize,inline,color=blue!20]{#1 \hfill --- H.E.}}
\newcommand{\ole}[1]{\todo[size=\tiny,color=green!20]{#1 \hfill --- O.L.}}
\newcommand{\Ole}[1]{\todo[size=\scriptsize,inline,color=green!20]{#1 \hfill --- O.W.}}

\begin{document}

%------------------------------------------------------------------------

\head{\sheetno}
%------------------------------------------------------------------------

%%%%%%%%%%%%%%%%%%% exercise 1 %%%%%%%%%%%%%%%%%%% 
\ex{1}{6}

%% write solution for ex 1 here
\begin{enumerate}[label=(\alph*)]
    \item Let $n \in \mathbb N$ be arbitrary, and let $e_i$ denote the $i$-th unit vector in $\mathbb R^n$. Consider the hyperplanes 
    \begin{align*}
        A_i &= \{ x \in \mathbb R^n : \langle x,e_i \rangle = 0 \}, \quad i = 1,...,n \\
        A_{n+1} &= \{ x \in \mathbb R^n: \langle x,\sum_{i=1}^ne_i \rangle = 42 \}
    \end{align*}
    \begin{itemize}
        \item It holds $\bigcap_{i=1}^{n+1} A_i = \emptyset$: we see that $$\bigcap_{i=1}^{n+1} A_i = \left(\bigcap^n_{i=1} A_i\right) \cap A_{n+1} = \{0\} \cap A_{n+1} = \emptyset.$$

        \item $A_i$ is convex for all $i=1,...,n+1$, since affine linear spaces are convex and $A_i$ is the solution space of a linear equation, which is an affine linear space.
        
        \item $\bigcap_{j=1}^n A_{i_j} \neq \emptyset:$ For any $A_{i_1},...,A_{i_n}$, each $A_{i_j}$ represents a hyperplane, whose normal vector is not colinear to the other normal vectors. Therefore, $\bigcap_{j=1}^n A_{i_j}$ defines a system of $n$ linear equations with $n$ variables that has exactly one solution, for the normal vectors are linear independent. Thus, $\bigcap_{j=1}^n A_{i_j}$ is nonempty.
    \end{itemize}
    
    

    \item Let $d \in \mathbb N_{\geq 3}$. Consider the sets $A_i = \{ (x_1,...,x_d) \in \mathbb R^d : x_2,...,x_{d} \geq 0, x_1 \geq i \}$. Then, $(A_i)_{i \in \mathbb N}$ is a family of convex sets, and $\bigcap_{i \in \mathbb N} A_i = \emptyset$. Furthermore, any sets $A_{i_1},...,A_{i_{d+1}}$ share a common point since $(i,0,...,0) \in \bigcap_{j=1,...,d+1} A_{i_j}$ for $i = \max_{j=1,...,d+1}\{i_j \}$.
\end{enumerate}


\grading{6}

%%%%%%%%%%%%%%%%%%% exercise 2 %%%%%%%%%%%%%%%%%%% 
\ex{2}{2}

%% write solution for ex 2 here
\begin{itemize}
    \item \textbf{Closed under multiplication:} Let $f,g \in \mathcal C(\mathbb R^d)$. Then, $f = \sum_{i=1}^{n}\lambda_i [A_i]$ and $g = \sum_{i=1}^m \mu_i [B_i]$, where $A_i$ and $B_i$ are closed convex sets in $\mathbb R^d$. We would like to prove that $f \cdot g$ is a linear combination of indicator functions of closed convex sets. In that case, $f \cdot g \in \mathcal C(\mathbb R^d)$. So,
    
    \begin{align*}
        f \cdot g = \left(\sum_{i=1}^{n}\lambda_i [A_i]\right) \cdot \left(\sum_{i=1}^m \mu_i [B_i]\right) &= \sum_{i=1}^{n}\left(\lambda_i[A_i] \cdot  \sum_{j=1}^m \mu_j [B_j] \right) \\
        &= \sum^n_{i=1}\sum^m_{j=1}\left(\lambda_i[A_i]\mu_j[B_j]\right) \\
        &= \sum^n_{i=1}\sum^m_{j=1} \lambda_i\mu_j[A_i \cap B_j].
    \end{align*}
    For each $i,j$ the set $A_i \cap B_j$ is closed and convex. Thus, $f\cdot g$ is indeed a linear combination of indicator functions of closed convex sets.

    \item $\mathcal C(\mathbb R^d)$ is not a finite dimensional real vector space. Assume it would be. In that case, there exists a basis $\left([A_1],...,[A_n]\right)$ of $\mathcal C(\mathbb R^d)$. Furthermore, we can assume that each $A_i$ and $A_j$ are pairwise disjoint for $i,j = 1,...,n$. If there would be $A_i$ and $A_j$ such that $A_i \cap A_j \neq \emptyset$, then define the following sets: 
    $$
        B_1 = A_i \cap A_j, \quad B_{2} = A_i \setminus A_j, \quad B_3 = A_j \setminus A_i.
    $$
    Note that 
    $$
        \mathrm{span}\{[A_1],...,[A_n]\} \subset \mathrm{span}\{[A_1],...,[A_{i-1}],B_1,B_2,B_3,[A_{i+1}],...,[A_n]\}.
    $$
    Therefore, we can assume that $A_i$ and $A_j$ are indeed disjoint. Next, we show that there exists a convex and closed set $B$ that cannot be a linear combination of $[A_1],...,[A_n]$. First, there is a set $A_i$ that contains more than one point (otherwise, we would have an infinite basis $[A_1],[A_2],...$). Let $B \subset A_i$ with $A_i \setminus B \neq \emptyset$. Now, we can easily see that $[B] \notin \mathrm{span}\{[A_1],...,[A_n]\}$, because $[B]$ can only be combined from $\{[A_i]\}$ but $B$ is a true subset of $A_i$. Thus, $\mathcal C(\mathbb R^d)$ is not a finite dimensional real vector space.

    \begin{figure}[h]
        \begin{center}
            \begin{tikzpicture}
                \draw (0,0) rectangle (2,2) node[pos=0.5]{$A_1$};
                \draw[shift={(2,0)}] (0,0) rectangle (2,2) node[pos=0.5]{$A_2$};
                \draw[shift={(4,0)}] (0,0) rectangle (2,2) node[pos=0.5]{$A_3$};
                \draw (0.25,0.25) rectangle (1.75,0.75) node[pos=0.5]{$B$};
            \end{tikzpicture}
            \caption{Example case for $\mathbb R^2$. We can see that the set $B$ cannot be approximated by the sets $A_1, A_2$ and $A_3$.}
        \end{center}
    \end{figure}
\end{itemize}

\grading{2}

%%%%%%%%%%%%%%%%%%% exercise 3 %%%%%%%%%%%%%%%%%%% 
\ex{3}{4}

%% write solution for ex 3 here

\begin{enumerate}
    \item $I_2$ is an open square in $\mathbb R^2$, i.e. a square without its edges.
    \begin{figure}[htbp]
        \centering
        \begin{tikzpicture}
            \draw[dashed, pattern=north west lines, pattern color=black] (0,0) rectangle (2,2) node[pos=.5]{$I_2$};

            \draw[fill=black, shift={(3,0)}] (0,0) circle[radius=2pt] node[anchor=north]{$v_2$};
            \draw[fill=black, shift={(3,0)}] (0,2) circle[radius=2pt] node[anchor=south]{$v_1$};
            \draw[fill=black, shift={(3,0)}] (2,0) circle[radius=2pt] node[anchor=north]{$v_3$};
            \draw[fill=black, shift={(3,0)}] (2,2) circle[radius=2pt] node[anchor=south]{$v_4$};

            \draw[shift={(6,0)}] (0,0)--node[pos=.5, anchor=north]{$B_3$}(2,0)--node[pos=.5, anchor=west]{$B_4$}(2,2)--node[pos=.5, anchor=south]{$B_1$}(0,2)--node[pos=.5, anchor=east]{$B_2$}(0,0) ;

            \draw[pattern=north west lines, pattern color=black, shift={(9,0)}] (0,0) rectangle (2,2) node[pos=0.5]{$Q$};
        \end{tikzpicture}
        \caption{The set $I_2$, the edges of the unit square, the four vertices and $Q=[0,1]^2$.}
    \end{figure}

    First, 
    $$
        Q = [0,1]^2 = I_2 \cup \bigcup_{i=1}^4 v_i \cup \bigcup^4_{i=1} b_i,
    $$ 
    where $v_i \in \{ (0,0),(0,1),(1,0),(1,1) \}$ are the four vertices of the unit square, and $B_i \in \{\{0\} \times ]0,1[, \{1\} \times ]0,1[, ]0,1[ \times \{0\}, ]0,1[ \times \{1\} \}$ are the four edges of the unit square. Now, it holds $\chi(\{v_i\}) = 1$ since $\{v_i\}$ is closed and convex. Additionally, 
    $$
        1 = \chi(\{0\} \times [0,1]) = \chi(B_1) + \chi(\{(0,0)\}) + \chi(\{(0,1)\}) = \chi(B_1) + 1 + 1
    $$ 
    and from this follows $\chi(B_1) = -1$. The same for $B_2, B_3$ and $B_4$.

    So, 
    $$
        1 = \chi(Q) = \chi(I_2) + \underbrace{\sum^4_{i=1}\chi(\{v_i\})}_{=4} + \underbrace{\sum^4_{i=1}\chi(B_i)}_{=-4} 
    $$

    Therfore, $\chi(I_2) = 1$.

    \item Let's compute $\chi(I_3)$, where $I_3$ is the open unit cube in $\mathbb R^3$. The unit cube $Q = [0,1]^3$ has $8$ vertices, and $12$ edges. Each vertex has an Euler characteristic of $1$ and each edge has an Euler characteristic of $-1$. So, $1=\chi(Q) = \chi(I_3) + 8 - 12 \implies \chi(I_3) = 5$.
\end{enumerate}

\grading{4}

%%%%%%%%%%%%%%%%%%% exercise 4 %%%%%%%%%%%%%%%%%%% 
\ex{4}{4}

%% write solution for ex 4 here

Let $A_1,...,A_n \subset \mathbb R^d$ be closed and convex such that $\bigcap^n_{i=1} A_i \neq \emptyset$. Consider $n = 2$. Then, $\chi(A_1 \cup A_2) = \chi(A_1) + \chi(A_2) - \chi(A_1 \cap A_2) = 1 + 1 - 1 = 1$. Note that $A_1 \cap A_2$ is convex and closed as well as not empty.

$n \leadsto n+1$. Assume $\chi(A_1 \cup ... \cup A_n) = 1$ for some $n \in \mathbb N$ if $A_i$ are closed and convex such that $A_1 \cap ... \cap A_n \neq \emptyset$. Then, 
\begin{align*}
    \chi(A_1 \cup ... \cup A_n \cup A_{n+1}) &= \chi(\left(\bigcup^n_{i=1} A_i\right) \cup A_{n+1})\\
    &= \chi\left(\bigcup^n_{i=1} A_i\right) + \chi(A_{n+1}) - \chi(\bigcup^n_{i=1} A_i \cap A_{n+1}) \\
    &= 1 + 1 - \chi(\bigcup^n_{i=1} A_i \cap A_{n+1})
\end{align*}

Consider the set $B_i = A_i \cap A_{n+1}$. It holds $B_1 \cap ... \cap B_n \neq \emptyset$ and $B_i$ is convex and closed. Therefore, $\chi(B_1 \cup ... \cup B_n) = 1$. Thus, 
$$
\chi(A_1 \cup ... \cup A_n \cup A_{n+1})= 1 + 1 - \chi(\underbrace{\bigcup^n_{i=1} A_i \cap A_{n+1}}_{=B_1 \cup...\cup B_n})= 1+1-1 = 1.
$$

\grading{4}
\totalgrade{16}
\end{document}
