\documentclass[a4paper]{article}

\usepackage{fontspec}
\setmainfont{[Georgia.ttf]}
\setmainfont[
    ItalicFont={[GeorgiaItalic.ttf]},
    BoldItalicFont={[GeorgiaBoldItalic.ttf]},
    BoldFont={[GeorgiaBold.ttf]}
]{[Georgia.ttf]}
\setsansfont[
    BoldFont={[SF-Pro-Display-Bold.otf]}
]{[SF-Pro-Display-Medium.otf]}

\usepackage{amsmath, amsthm, amssymb}
\usepackage{mathtools}
\usepackage[many]{tcolorbox}
\usepackage{xcolor}
\usepackage{titlesec}
\usepackage{titling}
\usepackage{enumitem}   

\definecolor{grey}{rgb}{0.5,0.5,0.5}
\definecolor{lightgrey}{rgb}{0.8,0.8,0.8}
\definecolor{darkgrey}{rgb}{0.3,0.3,0.3}
\definecolor{orange}{rgb}{0.94, 0.55, 0.294}
\definecolor{pink}{rgb}{0.94, 0.29, 0.7}
\definecolor{yellow}{rgb}{1, 0.749, 0}
\definecolor{green}{rgb}{0.235,0.702,0.443}

\newcommand{\chapfnt}{\fontsize{16}{19}}
\newcommand{\secfnt}{\fontsize{18}{17}}
\newcommand{\ssecfnt}{\fontsize{14}{14}}
\renewcommand{\hline}{\noindent\makebox[\linewidth]{\rule{12cm}{1pt}}}
\newcommand{\vip}[1]{\textit{\textbf{#1}}}
\newcommand{\R}{\mathbb{R}} % Reelle Zahlen
\newcommand{\N}{\mathbb{N}} % Natürliche Zahlen
\newcommand{\Z}{\mathbb{Z}} % Ganze Zahlen
\newcommand{\C}{\mathbb{C}} % Komplexe Zahlen
\newcommand{\Q}{\mathbb{Q}} % Rationale Zahlen
\DeclareMathOperator{\spn}{span}

\titleformat{\chapter}[display]
{\normalfont\chapfnt\bfseries}{\chaptertitlename\ \thechapter}{20pt}{\chapfnt}

\titleformat{\section}
{\normalfont\sffamily\secfnt\bfseries}{\thesection}{}{}

\titleformat{\subsection}
{\normalfont\sffamily\ssecfnt\mdseries}{\thesubsection}{}{}

\titleformat{\subsubsection}
{\normalfont\sffamily\ssecfnt\mdseries\color{grey}}{\thesubsection}{}{}

\titlespacing*{\chapter} {0pt}{50pt}{40pt}
\titlespacing*{\section} {0pt}{0pt}{8pt}
\titlespacing*{\subsection} {0pt}{12pt}{8pt}

\linespread{1.3}

\newtcbtheorem[auto counter,number within=section]{theorem}%
  {Theorem}{
  		fonttitle=\upshape, 
  		fontupper=\upshape,
  		boxrule=0pt,
  		leftrule=3pt,
  		arc=0pt,auto outer arc,
  		colback=white,
  		colframe=pink,
  		colbacktitle=white,
  		coltitle=pink,
  		oversize,
  		enlarge top by=1mm,
  		enlarge bottom by=1mm,
    	enhanced jigsaw,
    	interior hidden, 
    	before skip=12pt,
    	overlay={
    		\draw[line width=1.5pt,pink] (frame.north west) -- (frame.south west);
  		}, 
  		frame hidden}{theorem}
  		
\newtcbtheorem[]{issue}%
  {To prove}{
        theorem name,
  		fonttitle=\upshape, 
  		fontupper=\upshape,
  		boxrule=0pt,
  		leftrule=3pt,
  		arc=0pt,auto outer arc,
  		colback=white,
  		colframe=pink,
  		colbacktitle=white,
  		coltitle=pink,
  		oversize,
  		enlarge top by=1mm,
  		enlarge bottom by=1mm,
    	enhanced jigsaw,
    	interior hidden, 
    	before skip=12pt,
    	after skip=0pt,
    	overlay={
    		\draw[line width=1.5pt,pink] (frame.north west) -- (frame.south west);
  		}, 
  		frame hidden}{issue}

\newtcbtheorem[auto counter,number within=section]{lemma}%
  {Lemma}{
  		fonttitle=\upshape, 
  		fontupper=\upshape,
  		boxrule=1pt,
  		toprule=0pt,
  		leftrule=3pt,
  		arc=0pt,auto outer arc,
  		colback=white,
  		colframe=orange,
  		colbacktitle=white,
  		coltitle=orange,
  		oversize,
  		enlarge top by=1mm,
  		enlarge bottom by=1mm,
    	enhanced jigsaw,
    	interior hidden, 
    	before skip=12pt,
    	after skip=0pt,
    	overlay={
    		\draw[line width=1.5pt,orange] (frame.north west) -- (frame.south west);
  		}, 
  		frame hidden}{lemma}
  		
 \newtcbtheorem[auto counter,number within=section]{definition}%
  {Definition}{
  		fonttitle=\upshape, 
  		fontupper=\upshape,
  		boxrule=1pt,
  		toprule=0pt,
  		leftrule=3pt,
  		arc=0pt,auto outer arc,
  		colback=white,
  		colframe=orange,
  		colbacktitle=white,
  		coltitle=orange,
  		oversize,
  		enlarge top by=1mm,
  		enlarge bottom by=1mm,
    	enhanced jigsaw,
    	interior hidden, 
    	before skip=12pt,
    	overlay={
    		\draw[line width=1.5pt,orange] (frame.north west) -- (frame.south west);
  		}, 
  		frame hidden}{definition}
  		
  		 \newtcbtheorem[]{goal}%
  {Goal}{
  		theorem name,
  		fonttitle=\upshape, 
  		fontupper=\upshape,
  		boxrule=1pt,
  		toprule=0pt,
  		leftrule=3pt,
  		arc=0pt,auto outer arc,
  		colback=white,
  		colframe=orange,
  		colbacktitle=white,
  		coltitle=orange,
  		oversize,
  		enlarge top by=1mm,
  		enlarge bottom by=1mm,
    	enhanced jigsaw,
    	interior hidden, 
    	before skip=12pt,
    	overlay={
    		\draw[line width=1.5pt,orange] (frame.north west) -- (frame.south west);
  		}, 
  		frame hidden}{goal}
  		
\newtcbtheorem[]{important}%
  {Wichtig}{
  		fonttitle=\upshape, 
  		fontupper=\upshape,
  		boxrule=0pt,
  		leftrule=3pt,
  		arc=0pt,auto outer arc,
  		colback=white,
  		colframe=pink,
  		colbacktitle=white,
  		coltitle=pink,
  		oversize,
  		enlarge top by=1mm,
  		enlarge bottom by=1mm,
    	enhanced jigsaw,
    	interior hidden, 
    	before skip=12pt,
    	overlay={
    		\draw[line width=1.5pt,pink] (frame.north west) -- (frame.south west);
  		}, 
  		frame hidden}{important}
  		
  		
  		\newtcbtheorem[]{explanation}%
  {Explanation}{
        theorem name,
  		fonttitle=\upshape, 
  		fontupper=\upshape,
  		boxrule=0pt,
  		leftrule=3pt,
  		arc=0pt,auto outer arc,
  		colback=white,
  		colframe=green,
  		colbacktitle=white,
  		coltitle=green,
  		oversize,
  		enlarge top by=1mm,
  		enlarge bottom by=1mm,
    	enhanced jigsaw,
    	interior hidden, 
    	before skip=12pt,
    	after skip=12pt,
    	overlay={
    		\draw[line width=1.5pt,green] (frame.north west) -- (frame.south west);
  		}, 
  		frame hidden}{explanation}
    	
\renewcommand{\baselinestretch}{1.4} 
\makeatletter
\let\old@rule\@rule
\def\@rule[#1]#2#3{\textcolor{lightgrey}{\old@rule[#1]{#2}{#3}}}
\makeatother

\begin{document}

%%%%%%%%%%%%%%%%%%%%%%%%%%%%%
%%%%%% CONTENT HERE
%%%%%%%%%%%%%%%%%%%%%%%%%%%%%

\section*{Functional Analysis Sheet 5}
\subsection*{Featuring Diddle}

\subsubsection*{TO-DO}
\begin{itemize}
    \item Exercise 1 [ ]
    \item Exercise 2 [ in progress ] (Duc)
    \item Exercise 3 [ ]
    \item Exercise 4 [ ]
\end{itemize}

\subsubsection*{GNTM 2019: Caro packt aus - So war es mit Simone wirklich | INTERVIEW}

\vip{Wichtig:} \emph{https://www.youtube.com/watch?v=o-gR9KPCVbo}

\hline

\subsection*{Exercise 1}
\begin{issue}{}{}
If $E$ is separable, then $F \coloneqq \{ x \in E: ||x|| =1 \}$ is separable.
\end{issue}

\begin{proof}
Let $E$ be a normed space, which is separable. This means that there exists a \emph{countable dense} subset $Q \subset E$.

\textbf{Claim:} $\tilde Q \coloneqq \{ \frac{x}{||x||} : x \in Q \}$ is a countable dense subset of $F$.

\begin{enumerate}
    \item \vip{Countable subset:} For all $x \in \tilde Q$ it holds $||x|| = 1$. Thus, $x \in F$.
    
    We know $Q$ is countable. There exists a bijection $\tau: \tilde Q \to Q$ with $$\tau(\tilde q) = \tilde q ||\tilde q||.$$ So, $\tilde Q$ is countable.
    
    \item \vip{Dense in $F$:} Let $x \in F$. We know that $Q$ is dense in $E$. Hence, there exists a sequence $(x_n)_{n \in \mathbb N} \subset Q$ with $x_n \neq 0$ for all $n \in \mathbb N$ such that 
    \[
        \lim_{n \to \infty}x_n = x.
    \]
    Define a new sequence $(y_n)_{n \in \mathbb N}$ with
    \[
        y_n \coloneqq \frac{x_n}{||x_n||} \in \tilde Q \qquad \forall n \in \mathbb N.
    \]
    We will see that $y_n \to x$:
    \[
        \lim_{n \to \infty}y_n = \lim_{n \to \infty} \frac{x_n}{||x_n||} = \frac{\lim_{n \to \infty} x_n}{\underbrace{||\lim_{n \to \infty}x_n||}_{||\cdot|| \text{ is continuous}}} = \frac{x}{\underbrace{||x||}_{= 1}} = x.
    \]
    So, $\tilde Q$ is dense in $F$.
\end{enumerate}
We have shown the claim. Therefore, $F$ is separable per definition, as we have found a countable dense subset of $F$, namely $Q$.
\end{proof}

\begin{issue}{}{}
If $F \coloneqq \{ x \in E: ||x|| =1 \}$ is separable, then there exists a sequence $(x_n)_{n \in \mathbb N} \subset E$ such that $$\overline{\mathrm{span}((x_n)_{n \in \mathbb N})} = E.$$
\end{issue}
\begin{proof}
Let $F = \{ x \in E: ||x|| =1 \}$ be separable. Hence, we find a countable subset $Q \subset F$, that is dense in $F$ where $Q$ is of the Form $Q = (x_n)_{n\in\mathbb N}\subset E$ and $||x_n|| = 1$ for all $n\in\mathbb N$. Let now $x\in E$ be arbitrary. We want to find a sequence $(y_n)_{n\in\N}$ in $Q$ with $y_n\to x$ for $n\to\infty$ and every $y_n$ a finite linear combination of our $(x_n)$

Note that in any case where $x\neq 0$ (for $x=0$ everything is clear) we know $\frac{x}{||x||} = 1$. Thus, $\frac{x}{||x||}\in F$ so by assumption we can find a sequence $(\tilde y_n)_{n\in \N} \subset Q $ with $\tilde y_n \overset{n\to\infty}{\longrightarrow} \frac{x}{||x||}$ and . This of course means 
\[
    \lim_{n\to\infty} \tilde y_n = \frac{x}{||x||} \iff ||x||\lim_{n\to\infty}\tilde y_n = \lim_{n\to\infty} ||x||\tilde y_n = x.
\]
Since for all natural numbers $n$ we can find an $i_n\in\N$ with $y_n = x_{i_n}$ and hence we know that $||x||y_n \in\ \mathrm{span}((x_n)_n)$ and that means
\[
    x \in \overline{ \mathrm{span} ((x_n)_n) } \implies E = \overline{\mathrm{span}((x_n)_n)}
\]



\pagebreak
\emph{To show:} $\overline{\mathrm{span}((x_n)_{n \in \mathbb N})} \subset E$. Let $x \in\overline{\mathrm{span}((x_n)_{n \in \mathbb N})}$.

We know that $\C_{\Q} := \{z = a+ib : a,b\in\Q\}$ is countable and dense in $\C$. Thus, $\mathrm{span}_{\Q} := \left\{x \in E\ | \ \exists m_x\in \N: x = \sum_{i=1}^{m_x}\lambda_i x_{\tau_x(i)}\in \mathrm{span}((x_n)_n) \text{ and } \lambda_i \in \C_{\Q}\right\}$ - where $\tau_x$ is some permutation of $\N$ - is also dense in $\mathrm{span}((x_n)_n)$

Now we can see that
\begin{align*}
    \mathrm{span}_\Q &= \bigcup_{m\in\N} \left\{x = \sum_{i=1}^{m}\lambda_i x_{\tau_x(i)} \in E \ | \ \lambda_i \in \C_\Q \text{ for all } i\in\{1,...,m\} \right\}\\
    &= \bigcup_{m\in\N} \bigcup_{(n_1,...,n_m)\in \N^m} \bigcup_{(\lambda_1,...,\lambda_m)\in(\C_\Q)^m} \left\{\sum_{i=1}^{m}\lambda_i x_{n_i}\right\}
\end{align*}
which is countable. Thus we have a countable set for which
\[
    \overline{\mathrm{span}_\Q} = \overline{\mathrm{span}((x_n)_n)} = E.
\]
Hence $E$ is separable.


Now for the second part of the task: We show that $E^*$ being separable implies $E$ being separable by proving that $\mathrm{span}((x_n)_n)$ is dense in E for a sequence $(x_n)$ as discussed in the task. We will, however, assume that $||x_n|| = 1$ for all natural $n$. We can do that because $1 = ||\ell_n|| = \sup_{||x|| = 1} |\ell(x)|$ implies that for all positive $\epsilon$ we can find an $x$ with $||x|| = 1$ and still $|\ell_n(x)| > 1-\epsilon$.

Now towards a contradiction assume there exists a $x\in E$ with $x\notin \mathrm{span}((x_n)_n)$. This means we have
\[
    \inf_{y\in\mathrm{span}((x_n)_n)} ||x-y|| > 0.
\]
Then Corollary 4.8 in the script yields us an $\ell\in E^*$ with $\ell|_{\mathrm{span}((x_n)_n)} = 0$ and $||\ell|| = 1$, so $\ell\in S$.

This leads us for any $n\in\N$ to
\begin{align*}
    ||\ell - \ell_n|| = \sup_{x\neq 0}\frac{|\ell(x)-\ell_n(x)|}{||x||} \geq \frac{|\ell(x_n)-\ell_n(x_n)|}{||x_n||} = \frac{|\ell_n(x_n)|}{||x_n||} = \frac{1}{2||x_n||} > 0.
\end{align*}
Thus, $\ell \notin \overline{(\ell_n)_n}$ and hence $(\ell_n)_n$ is not dense in $S$. That is a contradiction!

\end{proof}


\subsection*{Exercise 2}
Let $E$ be a $\mathbb C$ vector space with a seminorm $\rho: E \to \mathbb R$ on $E$, and let $F$ be a linear subspace $F \subset E$. Let $f: F \to \mathbb K$ with
\[
    |f(x) | \leq \rho(x), \quad \forall x \in F.
\]
\begin{issue}{}{}
There exists an extension $\ell: E \to \mathbb K$ such that
\[
    \ell_{|F} = f\text{ and } |\ell(x)| \leq \rho(x) \quad \forall x \in F.
\]
\end{issue}



\subsection*{Exercise 3}
\begin{enumerate}[label=(\roman*)]
    \item     Let $X$ be a normed space and $C \subset X$ be open and convex with $0 \in C$. We show that $p_C: X \to [0,\infty)$ (note that $p_C(x) < \infty$ for all $x \in X$, which we will show later) is a sublinear functional.
    \begin{issue}{}{}
    For all $x,y \in X$ it holds
    \[
    	p_C(x+y) \leq p_C(x) + p_C(y) .
    \] 
    \end{issue}
    \iffalse
     \begin{proof}
        Let $x,y \in X$. Let $C \subset X$ be a convex and open subset of $X$ with $0 \in C$. We will prove the claim by \emph{contradiction}.
 Assume that $$ \underbrace{p_C(x)}_{\coloneqq \alpha} + \underbrace{p_C(y)}_{\coloneqq \beta} < \underbrace{p_C(x+y)}_{\coloneqq \gamma}.$$ 
        So, there exist sequences $(u_n)_{n \in \mathbb N} \subset C$ and $(\alpha_n)_{n \in \mathbb N} \subset \mathbb R_{> 0}$ such that
        \[
            \lim_{n \to \infty} \alpha_n u_n = x \quad \text{and} \quad \lim_{n \to \infty} \alpha_n = \alpha.
        \]
        Likewise, there exist sequences $(v_n)_{n \in \mathbb N} \subset C$ and $(\beta_n)_{n \in \mathbb N} \subset \mathbb R_{> 0}$ such that
        \[
            \lim_{n \to \infty} \beta_n v_n = y \quad \text{and} \quad \lim_{n \to \infty} \beta_n = \beta.
        \]
        Without loss of generality, let $\beta < \alpha$. 
        
		\subsubsection*{Step 1}    
		
		\begin{goal}{}{}
		Show that $\frac{1}{2}\gamma \leq \alpha$.
		\end{goal}
		    
        Define a new sequence $(w_n)_{n \in \mathbb N}$ as
        \[
            \forall n \in \mathbb N: w_n \coloneqq \frac{\beta_n}{\alpha_n}v_n.
        \]
        We will now prove that $w_n \in C$ for all $n \in \mathbb N$. Note that $\frac{\beta_n}{\alpha_n} < 1$. Thus, $w_n = \frac{\beta_n}{\alpha_n}v_n$ lies in the convex hull of $\{ 0, v_n \}$. Note that $0 \in C$ and $v_n \in C$. So, $\mathrm{convex}(\{ 0, v_n \}) \subset C$ since $C$ is a convex set. Thus, $w_n = \frac{\beta_n}{\alpha_n}v_n \in C$.
        
		Consider the sequence $(\tilde a_n)_{n \in \mathbb N}$ defined as $\tilde a_n \coloneqq u_n + w_n$. We would like the sequence $(\tilde a_n)_{n \in \mathbb N}$ to be in the subset $C$. However, this need not be the case. Thus, we scale the sequence $(\tilde a_n)_{n \in \mathbb N}$  by a factor $\xi > 0$ such that $\xi a_n = \xi (u_n + w_n) \in C$ for all $n \in \mathbb N$.  The next steps will accomplish this.
		
		 \begin{itemize}
		 	\item We know that $w_n$ and $u_n$ lie in $C$ for all $n \in \mathbb N$. Since $C$ is a convex set, the convex hull of $w_n$ and $u_n$ lies in $C$ for all $n \in \mathbb N$, i.e.
		 	\[
		 		\forall n \in \mathbb N: \mathrm{conv}({w_n,u_n}) = \{ p : p = \mu w_n + (1-\mu)u_n, \mu \in [0,1] \} \subset C.
		 	\]
		 	
		 	\item Especially, there exists a point $p_n \in \mathrm{conv}(w_n,u_n) \subset C$ with $$p_n = \frac{1}{2}(w_n +
		 	 u_n) \in C \qquad \forall n \in \mathbb N.$$
		 	
		 	\item Finally, choose $\xi = \frac{1}{2}$. Define a new sequence $(a_n)_{n \in \mathbb N}$ as 
		 	\[
		 		a_n \coloneqq \frac{1}{2}\tilde a_n =   \frac{1}{2}(w_n + u_n) = p_n \in C.
		 	\]
		 	Finally, we obtain a sequence $(a_n)_{n \in \mathbb N} \subset C$.
		 \end{itemize}
        
        Now, finding the sequence $(a_n)_{n \in \mathbb N}$ is good news for we see that
        \[
            \lim_{n \to \infty} 2\alpha_n a_n = \lim_{n \to \infty} \alpha_n (u_n +w_n) = \lim_{n \to \infty} \alpha u_n + \beta_n v_n = x + y.
        \]
        Hence, $2\alpha \in \{ t > 0 : x+y \in t C \}$. Remember that $\gamma = \inf \{ t > 0 : x+y \in t C \}$. So,
        \[
        	\gamma \leq 2\alpha \quad \text{ which implies } \quad \frac{1}{2}\gamma \leq \alpha.
        \]
        
        \subsubsection*{Step 2}    
		
		\begin{goal}{}{}
		Show that $\frac{1}{2}\gamma \leq \beta$.
		\end{goal}
       
       Define a new sequence $(t_n)_{n \in \mathbb N}$ as
        \[
            \forall n \in \mathbb N: t_n \coloneqq \frac{\alpha_n}{\beta_n}u_n.
        \]
		We know that $\alpha_n \geq \beta_n$, and hence $t_n$ might not need be in $C$. Yet, we know that $U_n \coloneqq \{ \mu u_n : \mu \in [0,1] \} = \mathrm{conv}(0,u_n) \subset C$ because $C$ is a convex set and $0 \in C$. Therefore, there exists a sequence $(\tilde u_n)_{n \in \mathbb N}$ such that 
\[
	\lim_{n \to \infty} \tilde \alpha_n \tilde u_n = x \quad \text{with} \quad \lim_{n \to \infty} \tilde \alpha_n \eqqcolon \tilde \alpha < \beta
\]		
by choosing
\[
	\tilde \u_n \coloneqq \frac{}{}u_n
\]

       
       \subsubsection*{Conclusion}    
		
		\textbf{Goal:} Show that $\gamma \leq \alpha + \beta$.
    \end{proof}
    \fi
       
    \begin{proof}
    	Let $x,y \in X$. From 3(ii) we know that $C$ is absorbing. Thus, $p_C(x) \neq \infty, p_C(y) \neq \infty$ and $p_C(x+y) \neq \infty$. Let $\epsilon> 0$ be arbitrary. For $x$ there exists $\mu_x \in \mathbb R_{> 0}$ and $z_y \in C$ such that $x = \mu_x z_x$ and $\mu_x < p_C(x) + \frac{\epsilon}{2}$. Likewise for $y$: there is $\mu_y \in \mathbb R_{>0}$ such that $y = \mu_y z_y$ with $\mu_y < p_C(y) + \frac{\epsilon}{2}$. Adding $\mu_x$ and $\mu_y$ together yields
    	\[
    		\mu_x + \mu_y < p_C(x) + p_C(y) + \epsilon.
    	\]
    	
    	\begin{goal}{}{}
    	Next, we try to find a vector $z_{x+y} \in C$ and scalar $\mu_{x+y} > 0$ such that $$\mu_{x+y} z_{x+y} = x+y.$$
    	\end{goal}
    	Consider the vector
    	\[
    		z_{x+y} \coloneqq \frac{\mu_x}{\mu_x + \mu_y}z_x + \frac{\mu_y}{\mu_x + \mu_y}z_y,
    	\]
    	which luckily lies in $C$. To see this, remember that $C$ is a convex set, and note that $z_{x+y}$ is a convex combination of $z_x \in C$ and $z_y \in C$ since
    	\[
    		 \frac{\mu_x}{\mu_x + \mu_y} + \frac{\mu_y}{\mu_x + \mu_y} = 1.
    	\]
    	Define the scalar $\mu_{x+y}$ as $\mu_{x+y} \coloneqq \mu_x + \mu_y > 0$. We see that
    	\[
    		\mu_{x+y}z_{x+y} = \mu_x z_x + \mu_y z_y = x+y.
    	\]
    	    	
    	Putting everything together
    	\[
    		p_C(x+y) \overset{(*)}{\leq} \mu_{x+y} = \mu_x + \mu_y < p_C(x) + p_C(y) + \epsilon.
    	\] 
    	
    	\begin{explanation}{}{}
    	$(*)$ holds by definition of $p_C(x+y)$.
    	\end{explanation}    	
    	
    	As $\epsilon$ was chosen arbitrarily, it holds
    	\[
    		p_C(x+y) \leq  p_C(x) + p_C(y).
    	\]
    	
    	 \end{proof}
    	
    	\hline
    	
    	\begin{issue}{}{}
    		For any $x \in X$ and $\lambda \in \mathbb R_{>0}$ it holds
    		\[
    			p_C(\lambda x) = \lambda p_C(x).
    		\]
    	\end{issue}
    	
    	\begin{proof}
    	Let $x \in X$ and $\lambda > 0$. Then,
    	\begin{align*}
    		p_C(\lambda x) = \inf \{ \alpha > 0 : \lambda x \in \alpha  C \} &= \inf \{ \alpha > 0 : x \in \frac{\alpha}{\lambda}  C \} \\
    		&= \inf \{ \lambda \alpha > 0 : x \in \alpha  C \} \\
    		&= \lambda \inf \{ \alpha > 0 : x \in \alpha  C \} \\
    		&= \lambda p_C(x).
    	\end{align*}
    	\end{proof}
   
   		\hline
    	
    	\textbf{Conclusion:} $p_C: X \to [0,\infty)$ is a sublinear functional.
    
    \item 
    
    \item We want to show:
    \begin{issue}{}{}
        \[
            C = \{ x \in X : p_C(x) < 1 \}
        \]
    \end{issue}
    
    \begin{proof}
    	sadasd
    \end{proof}
    
   
\end{enumerate}




\end{document}