% !TEX program = xelatex
\documentclass[9pt]{extarticle}

\usepackage{amsmath, amsthm, amssymb}
\usepackage{mathtools}
\usepackage[many]{tcolorbox}
\usepackage{xcolor}
\usepackage{titlesec}
\usepackage{titling}
\usepackage{enumitem}   
\usepackage{xfrac,unicode-math}
\usepackage{physics}

% -----------------------
% Font configuration
% -----------------------
\usepackage{fontspec}

\setmainfont[
    Path=fonts/
]{NewYorkMedium-Regular.otf}
\setmainfont[
    Path = fonts/,
    ItalicFont={NewYorkMedium-RegularItalic.otf},
    BoldItalicFont={NewYorkMedium-BoldItalic.otf},
    BoldFont={NewYorkMedium-Bold.otf}
]{NewYorkMedium-Regular.otf}
\setsansfont[
    Path = fonts/,
    BoldFont={SF-Pro-Display-Bold.otf}
]{SF-Pro-Display-Medium.otf}

\setmathfont{STIX2Math}[Path=fonts/]
\definecolor{grey}{rgb}{0.5,0.5,0.5}
\definecolor{lightgrey}{rgb}{0.8,0.8,0.8}
\definecolor{darkgrey}{rgb}{0.3,0.3,0.3}
\definecolor{orange}{rgb}{0.94, 0.55, 0.294}
\definecolor{pink}{rgb}{0.94, 0.29, 0.7}
\definecolor{yellow}{rgb}{1, 0.749, 0}
\definecolor{green}{rgb}{0.235,0.702,0.443}

\newcommand{\chapfnt}{\fontsize{16}{19}}
\newcommand{\secfnt}{\fontsize{18}{17}}
\newcommand{\ssecfnt}{\fontsize{14}{14}}

\renewcommand{\hline}{\noindent\makebox[\linewidth]{\rule{12cm}{1pt}}}

\titleformat{\chapter}[display]
{\normalfont\chapfnt\bfseries}{\chaptertitlename\ \thechapter}{20pt}{\chapfnt}

\titleformat{\section}
{\normalfont\sffamily\secfnt\bfseries}{\thesection}{}{}

\titleformat{\subsection}
{\normalfont\sffamily\ssecfnt\mdseries}{\thesubsection}{}{}

\titleformat{\subsubsection}
{\normalfont\sffamily\ssecfnt\mdseries\color{grey}}{\thesubsection}{}{}

\titlespacing*{\chapter} {0pt}{50pt}{40pt}
\titlespacing*{\section} {0pt}{0pt}{8pt}
\titlespacing*{\subsection} {0pt}{12pt}{8pt}

\linespread{1.3}

\renewcommand{\baselinestretch}{1.4} 
\makeatletter
\let\old@rule\@rule
\def\@rule[#1]#2#3{\textcolor{lightgrey}{\old@rule[#1]{#2}{#3}}}
\makeatother
\newcommand{\vip}[1]{\textit{\textbf{#1}}}
\newcommand{\R}{\mathbb{R}} % Reelle Zahlen
\newcommand{\N}{\mathbb{N}} % Natürliche Zahlen
\newcommand{\Z}{\mathbb{Z}} % Ganze Zahlen
\newcommand{\C}{\mathbb{C}} % Komplexe Zahlen
\newcommand{\Q}{\mathbb{Q}} % Rationale Zahlen
\newcommand{\K}{\mathbb{K}}
\DeclareMathOperator{\spn}{span}
\DeclareMathOperator{\ran}{ran}
% -----------------------
% Font configuration
% -----------------------


% -----------------------
% Font configuration
% -----------------------
\newtcbtheorem[auto counter,number within=section]{theorem}%
  {Theorem}{
  		fonttitle=\upshape, 
  		fontupper=\upshape,
  		boxrule=0pt,
  		leftrule=3pt,
  		arc=0pt,auto outer arc,
  		colback=white,
  		colframe=pink,
  		colbacktitle=white,
  		coltitle=pink,
  		oversize,
  		enlarge top by=1mm,
  		enlarge bottom by=1mm,
    	enhanced jigsaw,
    	interior hidden, 
    	before skip=12pt,
    	overlay={
    		\draw[line width=1.5pt,pink] (frame.north west) -- (frame.south west);
  		}, 
  		frame hidden}{theorem}
  		
\newtcbtheorem[]{proposition}%
  {Proposition}{
        theorem name,
  		fonttitle=\upshape, 
  		fontupper=\upshape,
  		boxrule=0pt,
  		leftrule=3pt,
  		arc=0pt,auto outer arc,
  		colback=white,
  		colframe=pink,
  		colbacktitle=white,
  		coltitle=pink,
  		oversize,
  		enlarge top by=1mm,
  		enlarge bottom by=1mm,
    	enhanced jigsaw,
    	interior hidden, 
    	before skip=12pt,
    	after skip=0pt,
    	overlay={
    		\draw[line width=1.5pt,pink] (frame.north west) -- (frame.south west);
  		}, 
  		frame hidden}{proposition}

\newtcbtheorem[auto counter,number within=section]{lemma}%
  {Lemma}{
  		fonttitle=\upshape, 
  		fontupper=\upshape,
  		boxrule=1pt,
  		toprule=0pt,
  		leftrule=3pt,
  		arc=0pt,auto outer arc,
  		colback=white,
  		colframe=orange,
  		colbacktitle=white,
  		coltitle=orange,
  		oversize,
  		enlarge top by=1mm,
  		enlarge bottom by=1mm,
    	enhanced jigsaw,
    	interior hidden, 
    	before skip=12pt,
    	after skip=0pt,
    	overlay={
    		\draw[line width=1.5pt,orange] (frame.north west) -- (frame.south west);
  		}, 
  		frame hidden}{lemma}
  		
 \newtcbtheorem[auto counter,number within=section]{definition}%
  {Definition}{
  		fonttitle=\upshape, 
  		fontupper=\upshape,
  		boxrule=1pt,
  		toprule=0pt,
  		leftrule=3pt,
  		arc=0pt,auto outer arc,
  		colback=white,
  		colframe=orange,
  		colbacktitle=white,
  		coltitle=orange,
  		oversize,
  		enlarge top by=1mm,
  		enlarge bottom by=1mm,
    	enhanced jigsaw,
    	interior hidden, 
    	before skip=12pt,
    	overlay={
    		\draw[line width=1.5pt,orange] (frame.north west) -- (frame.south west);
  		}, 
  		frame hidden}{definition}
  		
  		 \newtcbtheorem[]{goal}%
  {Goal}{
  		theorem name,
  		fonttitle=\upshape, 
  		fontupper=\upshape,
  		boxrule=1pt,
  		toprule=0pt,
  		leftrule=3pt,
  		arc=0pt,auto outer arc,
  		colback=white,
  		colframe=orange,
  		colbacktitle=white,
  		coltitle=orange,
  		oversize,
  		enlarge top by=1mm,
  		enlarge bottom by=1mm,
    	enhanced jigsaw,
    	interior hidden, 
    	before skip=12pt,
    	overlay={
    		\draw[line width=1.5pt,orange] (frame.north west) -- (frame.south west);
  		}, 
  		frame hidden}{goal}
  		
\newtcbtheorem[]{important}%
  {Wichtig}{
  		fonttitle=\upshape, 
  		fontupper=\upshape,
  		boxrule=0pt,
  		leftrule=3pt,
  		arc=0pt,auto outer arc,
  		colback=white,
  		colframe=pink,
  		colbacktitle=white,
  		coltitle=pink,
  		oversize,
  		enlarge top by=1mm,
  		enlarge bottom by=1mm,
    	enhanced jigsaw,
    	interior hidden, 
    	before skip=12pt,
    	overlay={
    		\draw[line width=1.5pt,pink] (frame.north west) -- (frame.south west);
  		}, 
  		frame hidden}{important}
  		
  		
  		\newtcbtheorem[]{explanation}%
  {Explanation}{
        theorem name,
  		fonttitle=\upshape, 
  		fontupper=\upshape,
  		boxrule=0pt,
  		leftrule=3pt,
  		arc=0pt,auto outer arc,
  		colback=white,
  		colframe=green,
  		colbacktitle=white,
  		coltitle=green,
  		oversize,
  		enlarge top by=1mm,
  		enlarge bottom by=1mm,
    	enhanced jigsaw,
    	interior hidden, 
    	before skip=12pt,
    	after skip=12pt,
    	overlay={
    		\draw[line width=1.5pt,green] (frame.north west) -- (frame.south west);
  		}, 
  		frame hidden}{explanation}
  		
  		
  		\newtcbtheorem[]{congrats}%
  {We used an assumption!}{
        theorem name,
  		fonttitle=\upshape, 
  		fontupper=\upshape,
  		boxrule=0pt,
  		leftrule=3pt,
  		arc=0pt,auto outer arc,
  		colback=white,
  		colframe=green,
  		colbacktitle=white,
  		coltitle=green,
  		oversize,
  		enlarge top by=1mm,
  		enlarge bottom by=1mm,
    	enhanced jigsaw,
    	interior hidden, 
    	before skip=12pt,
    	after skip=12pt,
    	overlay={
    		\draw[line width=1.5pt,green] (frame.north west) -- (frame.south west);
  		}, 
  		frame hidden}{congrats}
    	

\begin{document}

\section*{Functional Analysis II - Exercise 02}
\textbf{Lecturer:} Mones Raslan \\
\noindent\textbf{Date:} 10/28/19

\subsection*{Orthogonal Projections}
\begin{definition}{}{}
Let $E$ be a vector space. A mapping $P: E \to E$ is called a \textbf{projection} if $P^2 = P$.
\end{definition} 
From now on we only consider \textit{linear} projections.

\begin{definition}{}{}
Let $Y \subset E$ be a subspace. A mapping $P: E \to E$ is called a \textbf{projection onto $Y$} if $P^2 = P$ and $\ran(P) = Y$.
\end{definition}

\begin{proposition}{}{}
\begin{enumerate}
\item $P$ is a projection onto $Y$ if and only if
\[
    \forall x \in E: Px \in Y \quad \text{and} \quad P|_Y = \mathrm{Id}|_Y.
\]

\item If $P$ is a projection, then $E = \ker P \dot + \ran P$, where $\dot +$ denotes the direct sum.

\item If $P$ is a projection, so is $\mathrm{Id} - P$. Additionally, $\ker{\mathrm{Id - P}} = \ran P$ and $\ran \mathrm{Id} - P = \ker P$.

\item For every subspace $Y \subset E$ there exists a projection onto $Y$. The proof uses Zorn's lemma.
\end{enumerate}
\end{proposition}
Now, let $E$ be a normed space. Under which circumstances is $P$ continuous?

\begin{proposition}{}{}
Let $P \in L(E)$ be a projection. Then,
\begin{enumerate}
\item $\ker P$ and $\ran P$ are closed
\item $||P|| \geq 1$ or $||P|| = 0$.
\end{enumerate}
\end{proposition}

\begin{proof}
\begin{enumerate}
\item Let $P \in L(E)$. Because of continuity we immediately get that $\ker P = P^{-1}(\{0 \})$ is closed \textit{(preimage of closed sets of continuous functions are closed)}. To show that the range of $P$ is closed note that $\ran P = \ker \mathrm{Id} - P$ and $\mathrm{Id} - P \in L(E)$. We just showed that the $\ker \mathrm{Id} - P$ is closed, and so is $\ran P$.

\item It holds $||P|| = ||P^2|| \leq ||P||^2$. Therefore, $Px \geq 1$ or $Px = 0$. Otherwise, $||P|| \leq ||P||^2$ cannot hold.
\end{enumerate}
\end{proof}

There exist normed spaces $E$ and \vip{closed} subspaces $Y \subset E$ such that there is no continuous projection onto $Y$.

In Hilbert spaces the situation is different. We showed in Functional Analysis I that 
\[
    E = Y \oplus Y^\perp, \quad \text{$E$ is Hilbert and $Y$ is closed.}
\]
For $x = y + z$ where $y \in Y$ and $z \in Y^\perp$ define 
\[
    P: E \to Y, x \mapsto y
\]

\begin{definition}{}{}
    $P$ is called \textbf{orthogonal projection} if $\ker P \perp \ran P$.
\end{definition}

\begin{proposition}{}{}
Let $H$ be a Hilbert space. Let $P \in L(H), P \neq 0$ be a projection. Then the following are equivalent
\begin{enumerate}
\item $P$ is an orthogonal projection
\item $||P|| = 1$
\item $P = P^*$
\item $P^*P = PP^*$
\item $P$ is positive, i.e. $\langle Tx,x \rangle \geq 0$ for all $x \in H$.
\end{enumerate}
\end{proposition}

\begin{proof}
\begin{itemize}
\item $(i) \implies (ii)$: Let $P$ be an orthogonal projection. We want to show that $||P|| = 1$. Let $x \in H$ with $x = y + z = Py + z$ where $y \in \ran P$ and $x \in \ker P$. Then,
\[
    ||Px||^2 = ||Py||^2 \leq ||Py||^2 + ||z||^2 \overset{(*)}{=} ||x||^2
\]
where $(*)$ is true because $Py \perp z$. To see this, note that $$||Py||^2 + ||z||^2 = ||Py||^2 + 2\langle Py, z \rangle + ||z||^2 = ||Py + z||^2.$$

\item $(ii) \implies (i)$: Let $x \in \ker P$ and $y \in \mathrm{ran} P$. To show that $P$ is an orthogonal projection we need to prove that $\ker P \perp \ran P$. For any $\lambda$ we have 
\[
    ||\lambda y||^2 = ||P(\lambda y + x)||^2 \leq ||\lambda y + x||^2 = || x||^2 + 2\Re(\bar \lambda \langle x, y \rangle) + \lambda ||y||^2    
\]
Hence, $-2\Re(\bar \lambda \langle x, y \rangle) \leq ||x||^2$. Since this is true for all $\lambda \in \mathbb{C}$ we have $\langle x,y \rangle = 0$.

\item $(iii) \implies (iv)$: This is clear.

\item $(i) \implies (iii)$: Homework exercise.

\item $(i) \implies (v)$: Let $P$ be an orthogonal projection. We want to show that $P$ is positive. Let $x = y + z \in H$ with $y \in \ran P$ and $z \in \ker P$. Then,
\[
    \langle Px, x \rangle = \langle y, y+z \rangle = ||y||^2 \geq 0.    
\]
Note that $\langle Px, x\rangle \in \mathbb{R}$ because of $||y||$.

\item $(iii) \implies (i)$: Let $P$ be self-adjoint. We want to show that $P$ is an orthogonal projection. Let $y \in H$ and $z \in \ker P$. Then, $\langle Py,z \rangle = \langle y, Pz \rangle = 0$ as $Pz = 0$.
\item $(v) \implies (iii)$: Let $P$ be positive. We want to show that $P = P^*$. $P \geq 0 \implies \langle Px, x \rangle \in \mathbb{R} \implies P = P^*$ (by exactly the same homework)

\item $(iv) \implies (i)$: From homework sheet 1 \#1 we know
\[
    T \in L(H) \text{ is normal} \iff ||Tx|| = ||T^*x|| \quad \forall x \in H.  
\]
Thus, $\ker P = \ker P^*$. We proved earlier that the range of $P$ is closed. By the closed range theorem we have 
\[
    \ker P^* = (\ran P)^\perp \implies \ker P = (\ran P)^\perp.
\]
\end{itemize}
\end{proof}

\end{document}