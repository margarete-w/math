\documentclass[12pt,letterpaper]{article}
\usepackage{amsmath,amsthm,amsfonts,amssymb,amscd}
\usepackage{enumerate}

\setlength\parindent{0pt}

\begin{document}
Let $E$ be a Banach space, and let $T \in L(E)$.

\textbf{To show:} $\ker T^* = (\text{ran} T)^\perp$

\begin{proof}
Let $x \in E$ such that $T^*x = 0$. Let $y \in H$. One sees that
\[
    \langle y, T^*x \rangle = \langle y, 0 \rangle = 0
\]
Using the adjoint operator one obtains
\[
    \langle Ty, x \rangle = 0.
\]
This holds for all $y \in H$ and $x \in \ker T^*$, and hence $x \in (\text{ran} T)^{\perp}$.

For the reverse direction, let $x \in E$. Let $z \in \ker T^*$. Observe that
\[
    \langle z, Tx \rangle = \langle T^* z, x \rangle = 0,
\]
implying $Tx \in (\ker T^*)^\perp$.

In the end, one obtains $\ker T^* = (\text{ran} T)^\perp$.
\end{proof}

Let $H$ be a Hilbert space, and let $F \in L(E)$.
\textbf{To show:} $\ker F = (\overline{\text{ran} F^*})^\perp$.
\begin{proof}
It is already known that $\ker T^* = (\text{ran} T)^\perp$ in a Banach space for any linear and bounded operator $T$. Now, set $T = F^*$, and one obtains $\ker F = (\text{ran} F^*)^\perp$, since it holds ${F^*}^* = F$ for a reflexive space. Knowing that $S^\perp = \overline{S}^\perp$ for any subset $S \subset H$, one immediately sees $\ker F = (\text{ran} F^*)^\perp = (\overline{\text{ran} F^*})^\perp$.
\end{proof}

\textbf{To show:} Let $H$ be an inner product space. $S^\perp = \overline{S}^\perp$ for any subset $S \subset H$.

\begin{proof}
    Let $x \in S^\perp$. Per definition, one has $\langle x,y \rangle = 0$ for all $y \in S$. We want to show that even $\langle x,y \rangle$ holds for all $y \in \bar S$. Let $y \in \bar S$ with $\lim y_n = y$ with $y_n \in S$ for all $n \in \mathbb N$. Observe that
    \begin{align*}
        0 = \lim \langle x,y_n \rangle = \langle x,y \rangle.
    \end{align*}
    Thus, $S^\perp \subset \bar S^\perp$. For the other direction, note that $S \subset \bar S$, and as a result, one gets $\bar S^\perp \subset \bar S^\perp$.

    In the end, $S^\perp = \overline{S}^\perp$.
\end{proof}

\end{document}