\documentclass{article}

\usepackage[utf8]{inputenc}
\usepackage{amsthm,amsmath,amssymb,mathtools}


\usepackage{geometry}
\usepackage{fancyhdr} % Kopfzeile
\usepackage{mathabx} % orthogonal direct sum sign
\usepackage{enumitem}
\usepackage{framed}
\usepackage{ulem}
\usepackage{wasysym} %lightning symbol

\usepackage{titlesec}
\titleformat*{\section}{\large\bfseries}

% benutzerdefinierte Kommandos
\newcommand{\crown}[1]{\overset{\symking}{#1}}
\newcommand{\xcrown}[1]{\accentset{\symking}{#1}}
\newcommand{\R}{\mathbb{R}}

\def\doubleunderline#1{\underline{\underline{#1}}}

% roman numbers
\makeatletter
\newcommand*{\rom}[1]{\expandafter\@slowromancap\romannumeral #1@}
\makeatother



\newtheorem*{theorem}{Theorem}
\newtheorem*{definition}{Definiton}

\newtheoremstyle{named}{}{}{\itshape}{}{\bfseries}{.}{.5em}{\thmnote{#3}#1}
\theoremstyle{named}
\newtheorem*{namedtheorem}{}




% Kopfzeile
\pagestyle{fancy}
\fancyhf{}
\rhead{Nguyen (395220)}
\lhead{\textbf{Lecture notes Mathematical Physics I WS1819}}
\cfoot{Page \thepage}


\setlength\parindent{0pt}


\begin{document}
	
\begin{proof}
	\textbf{Sufficiency:} We consider a candidate Lyapunov function: $F(x) = \langle x, Px \rangle, P \in GL(n, \mathbb R)$ is positive, definitive, symmetric.\\
	
	$F(x) > 0, \forall x \in \mathbb R^n  \setminus \{0 \}, F(0) = 0.$ We need to prove that $\mathcal L_vF(x) < 0, \forall x \in \mathbb R^n \setminus \{  0 \}$. 
	
	\[
		v = \sum^n_{i=1} (Ax)_i \frac{\partial}{\partial x_i}, grad_n F(x) = 2 P_x.
	\]
	
	Then $(\mathcal L_v F)(x) = 2 \langle Ax, Px \rangle = \langle x , A^TPx \rangle + \langle x, PAx \rangle = \langle x, (A^TP+PA)x \rangle = - \langle x,Qx \rangle < 0$. \\
	
	\textbf{Necessity:} we assume $\Re \lambda_i < 0 \forall i=1,...,p$. We define $P = \int^\infty_0 \exp(A^Tt) Q \exp(At) dt$. We observe that $P$ is positive definite and the $\int < + \infty$ because $A$ is the matrix of an symptotically stable IVP. We reconstruct Lyapuniv equation: 
	\begin{align*}
		&- A^T (\int^\infty_0 \exp(A^Tt) Q \exp(At) dt) - (\int^\infty_0 \exp(A^Tt) Q\exp(At)dt)A \\
		&= - \int^\infty_0(A^T \exp(A^Tt) Q\exp(At) + \exp(A^Tt)Q\exp(At)A)dt \\
		&= - \int^\infty_0 \frac{d}{dt} (\exp(A^Tt)Q \exp(At)) dt = - \exp(A^Tt)Q\exp(At) \Big \vert^\infty_0 = Q
	\end{align*}
	which shows that our $P$ solves the Lyapunov equation.
\end{proof}

Consider the IVP:
\begin{align*}
\begin{cases}
\dot x = f(x) \\x(0) = x_0
\end{cases}
f \in \mathcal C^k(M, \mathbb R^n), k \geq 1, \tilde x \text{ to be a fixed point }(f(\tilde x) = 0).
\end{align*}
We can always assume that $\tilde x = 0$. \\

\textbf{Idea:} Linearization around $\tilde x = 0$ is $\begin{cases}\dot x = Ax,\\ x(0) = x_0\end{cases}$ where $A = \frac{\partial f}{\partial x} \Big \vert_{\tilde x = 0}$ (Jacobian matrix) with
\begin{gather*}
\dot x = f(x) = f(0) + Ax + g(x) \\
g(0) = 0, \Vert g(x) \Vert = O(\Vert x \Vert) \text{ as } \Vert x \Vert \to 0.
\end{gather*}

\textit{Remark:} $\tilde x = 0$ is hyperbolic if  $\tilde x = 0$ is hyperbolic for the linearized problem.

\begin{theorem}[Poincare-Lyapunov]
Two statements:
\begin{enumerate}[label=(\arabic*)]
	\item If $\tilde x = 0$ is asymptotically stable for the linearization then $\tilde x= 0$ is asymptotically stable for the nonlinear problem.
	
	\item As abov, if  $\tilde x= 0$ is unstable.
\end{enumerate}
\end{theorem}


\begin{proof} 
$\tilde x = 0$ is asymptotically stable for $\tilde x = Ax, A = \frac{\partial f}{\partial x} \Big \vert_0 \implies \Re \lambda_i < 0 \forall i$, From the previous Theorem, we know that $\exists P \in GL(n, \mathbb R)$ positive, definitive, symmetric Matrix such that $F(x) = \langle x, Px \rangle$ is a strict Lyapunow function if $-PA - A^TP = Q$ is positive, definitive, symmetric matrix. \\

Let $q_i > 0$ be eigenvalues of $Q$, $p_i > 0$ egeinvalues of P, $p_0 = \max p_i > 0, q_0 ) \min q_i > 0$.
\begin{align*}
\Vert P_x \Vert \leq p_0 \Vert x \Vert, \langle x,Qx \rangle \geq q_0 \Vert x \Vert^2.
\end{align*}
Look at $F$ as a candidate strict Lyapunov function for $\dot x = f(x)$.
\[
	F(x) > 0, F(x) = 0 \iff x = 0.
\]
The function $g$ is such that $\forall \epsilon > 0 \exists \delta > 0$ such that $\Vert y(x) \Vert < \epsilon \Vert x \Vert$ for all $\Vert x \Vert < \delta$. Then
\begin{align*}
(\mathcal L_vF)(x) = 2 \langle f(x), Px \rangle &= \langle f(x), Px \rangle + \langle x, Pf(x) \rangle \\
&= \langle Ax + g(x), Px \rangle + \langle x, P(Ax+y(x)) \rangle \\
&= - \langle x, Qx \rangle + 2 \langle g(x), Px \rangle \\
& \leq - q_0 \Vert x\Vert^2 + 2 \Vert g(x) \Vert \Vert Px \Vert \\
&\leq - q_0 \Vert x \Vert^2 + 2\epsilon p_0 \Vert x \Vert^2 \\
& \leq (-q_0 + 2 \epsilon p_0) \Vert x \Vert^2 .
\end{align*}
That is $< 0$ if we choose $\epsilon$ such that $-q_0 + 2 \epsilon p_0 < 0$.
\end{proof}

\begin{theorem}[Poincare-Lyapunov]
Let $\lambda_1,...,\lambda_p$ eigenvalues of $A = \frac{\partial f}{\partial x}\Big \vert_0$.
\begin{enumerate}
	\item If $\Re \lambda_i < -y \forall i = 1,..., p$ with some $y > 0$ then $\exists$ neighbourhoog $\tilde M$ of $\tilde x=0$ such that
	\begin{enumerate}
		\item $\Phi_t$ is such that $\Phi_t(x) \in \tilde M \forall x \in \tilde M \forall t \geq 0$
		\item $\exists C > 0$ such that $\Vert \Phi_t(x) \Vert \leq C \exp(-y\frac{t}{2}) \Vert x \Vert \quad  \forall \tilde x \in \tilde M, t \geq 0$
	\end{enumerate}

	\item If $\exists \lambda_k$ with $\Re \lambda_k > 0$ then $\tilde x = 0$ is unstable.
\end{enumerate}
\end{theorem}

\textbf{Example:} 
\begin{align*}
\begin{cases}
\dot x_1 = x_2 \\
\dot x_2 = x_2(1-x_1^2)- x_1
\end{cases} \tag{\text{Van der Pol system}}
\end{align*}
Detect stability of fixed point $(0,0)$. Linearize this problem
\begin{align*}
\dot x = Ax, \quad A = \begin{pmatrix}
0 & 1 \\ -1 & 1
\end{pmatrix}.
\end{align*}

If this point $(0,0)$ is hyperbolic, we can use Poincare-Lyapunov (?). We find the eigenvalues
\[
	\lambda_{1,2} = \frac{(1 \pm i \sqrt 3)}{2}
\]
Unstable for the linearized problem, so for the origin problem. \\

\section*{Topological equivalence / conjugacy of dynamical systems}
\textbf{Idea:} find some equivalence classes for dynamical systems.\\

Consider $\{ \Phi_t, \mathbb R, M_1 \}, M_1 \subset \mathbb R^n$ and $\{ \Psi_t, \mathbb R, M_2 \}, M_2 \subset \mathbb R^n$ with the vector fields $v = \sum^n_{i=1} f_i \frac{\partial}{\partial x_i}$ and $w = \sum^n_{i=1} g_i \frac{\partial}{\partial x_i}$, $f(x) = \frac{d}{dt} \Big \vert_{t=0} \Phi_t(x)$ and $g(x) = \frac{d}{dt} \Big \vert_{t=0} \Psi_t(x)$

\begin{definition}
	$\{ \Phi_t, \mathbb R, M_1 \}$ and $\{ \Psi_t, \mathbb R, M_2 \}$ are topologically equivalent if there exists a homeomorphism  $\Delta: M_1 \to M_2$ which maps the orbits of $\{ \Phi_t, \mathbb R, M_1 \}$ onto the orbits of $\{ \Psi_t, \mathbb R, M_2 \}$ preserving the direction of time:
	\[
		(\Delta \circ \Phi_t)(x) = (\Psi_t \circ \Delta)(x), \tau: \mathbb R \to \mathbb R: t \mapsto \tau(t) \text{ reapproximation of time}.
	\]
	If $\Delta$ is $C^k$ then $\{ \Phi_t, \mathbb R, M_1 \}$ and $\{ \Psi_t, \mathbb R, M_2 \}$ are $C^k$-diffeomorphic. If $\tau = id$ then $\{ \Phi_t, \mathbb R, M_1 \}$ and $\{ \Psi_t, \mathbb R, M_2 \}$ are conjugate.
\end{definition}

Assume that $\{ \Phi_t, \mathbb R, M_1 \}$ and $\{ \Psi_t, \mathbb R, M_2 \}$ are $C^k$-diffeomorphics with $\tau(t) = t$.
\begin{gather*}
	\Delta(\Phi_t(x)) = \Psi_t(\Delta(x))  \text{ differentiate with respect to } t \text{ at } t= 0. \\
	\frac{\partial \Delta}{\partial x}f(x) = y(\Delta(x)) \implies f(x) = (\frac{\partial \Delta}{\partial x})^{-1}g(\Delta(x)) \quad \text{Formular for transformating coordinates}
\end{gather*}

\textbf{Example:} Let us consider three dynamical systems.
\begin{align*}
\begin{cases}
\dot x_1 = -2x_1 \\
\dot x_2 = -2x_2
\end{cases} 
\begin{cases}
\dot x_1 = -2x_1 +x_2\\
\dot x_2 = -x_1-2x_2
\end{cases}
\begin{cases}
\dot x_1 = -2x_1 \\
\dot x_2 = x_1-2x_2
\end{cases}
\end{align*}
Eigenvalues:
\begin{align*}
\lambda_{1,2} = -2, \lambda_{1,2} = -2 \pm i, \lambda_{1,2} = -2
 \end{align*}

\begin{align*}
(x_1,x_2) = r (\sin(\sigma), \cos(\sigma)) \\
\begin{cases}
\dot r = -2r \\\dot \sigma = 0
\end{cases}
\begin{cases}
\dot s = -2s \\
\dot \varphi = -1
\end{cases}
\end{align*}
$A: (s, \varphi) \mapsto (r, \sigma) = (s, \varphi - \frac{1}{2}\ln(s))$

\begin{theorem}[Hartman-Grobman]
Any hyperbolic IVP in $\mathbb R^n$
\[
	\begin{cases}
		\dot x = Ax \\ 
		x(0) = x_0
	\end{cases}, A \in gl(n, \mathbb R) \text{ can be non invertible}
\]
is locally topologically conjugate in $\tilde M$ (neighbourhood of $\tilde x = 0$) to any nonlinear IVP in $\tilde M$:
\begin{align}\label{la}
	\begin{cases}
		\dot y = Ay + f(y) \\
		y(0) = x_0
	\end{cases}
\end{align}
where $f \in C^1(\tilde M, \mathbb R^n)$ with $f(0) = 0, \frac{\partial f}{\partial y}\Big \vert_0 = 0$. Precisely, if $\Phi_t$ is the flow of \eqref{la} then there exists homeomorphism $x \mapsto y \Delta (x)$ for which
\[
	(\Delta \circ \exp(tA)) (a) = (\Phi_t \circ \Delta) (x) , \quad \forall x \in \tilde M.
\]
\end{theorem}

\end{document}
