\documentclass{article}

\usepackage[utf8]{inputenc}
\usepackage{amsthm,amsmath,amssymb,mathtools}
\usepackage{geometry}
\usepackage{fancyhdr} % Kopfzeile
\usepackage{mathabx} % orthogonal direct sum sign
\usepackage{enumitem}
\usepackage{framed}
\usepackage{ulem}
\usepackage{wasysym} %lightning symbol
\usepackage{tikz}
\usepackage{adjustbox} % Used to constrain images to a maximum size 
\usepackage{titlesec}
\titleformat*{\section}{\large\bfseries}

% benutzerdefinierte Kommandos
\newcommand{\crown}[1]{\overset{\symking}{#1}}
\newcommand{\xcrown}[1]{\accentset{\symking}{#1}}
\newcommand{\R}{\mathbb{R}}

\def\doubleunderline#1{\underline{\underline{#1}}}

% roman numbers
\makeatletter
\newcommand*{\rom}[1]{\expandafter\@slowromancap\romannumeral #1@}
\makeatother



\newtheorem*{theorem}{Theorem}

\newtheoremstyle{named}{}{}{\itshape}{}{\bfseries}{.}{.5em}{\thmnote{#3}#1}
\theoremstyle{named}
\newtheorem*{namedtheorem}{}




% Kopfzeile
\pagestyle{fancy}
\fancyhf{}
\rhead{Duc (395220), Victor (405592)}
\lhead{\textbf{MP I UE04} - Jan Techter (Monday 10-12 am)}
\cfoot{Page \thepage}


\setlength\parindent{0pt}


\begin{document}
\section*{Exercise 1}
We have
\[
	f(x_1,x_2) = \begin{pmatrix}x_1x_2 \\ x_2^2\end{pmatrix}
	g(x_1,x_2) = \begin{pmatrix}x_1 \\ 0\end{pmatrix}
	h(x_1,x_2) = \begin{pmatrix}x_2 \\ 0\end{pmatrix}
\]
with $v_1 = \sum f_i \frac{\partial}{\partial x_i}, v_2 = \sum g_i \frac{\partial}{\partial x_i}, v_3 = \sum h_i \frac{\partial}{\partial x_i}$.\\

Now we calculate the Lie brackets:
\begin{align*}
	[v_1,v_2] = (x_1x_2 \frac{\partial}{\partial x_1} + x_2^2\frac{\partial}{\partial x_2}) x_1\frac{\partial}{\partial x_1} - (x_1\frac{\partial}{\partial x_1})(x_1x_2\frac{\partial}{\partial x_1} + x_2^2\frac{\partial}{\partial x_2}) = x_1x_2\frac{\partial}{\partial x_1} - x_1x_2\frac{\partial}{\partial x_1} = 0.
\end{align*}
$\Phi^t$ and $\Psi^s$ commute due to $	[v_1,v_2] \neq 0$.
\begin{align*}
	[v_1,v_3] = (x_1x_2 \frac{\partial}{\partial x_1} + x_2^2\frac{\partial}{\partial x_2})x_2\frac{\partial}{\partial x_1} - (x_2\frac{\partial}{\partial x_1})(x_1x_2\frac{\partial}{\partial x_1} + x_2\frac{\partial}{\partial x_2}) = 0 + x_2^2\frac{\partial}{\partial x_1} - x_2^2 \frac{\partial}{\partial x_1} - 0 = 0.
\end{align*}
$\Phi^s$ and $\Theta^w$ commute.
\begin{align*}
[v_2,v_3] = (x_1 \frac{\partial}{\partial x_1})x_2\frac{\partial}{\partial x_1} - (x_2\frac{\partial}{\partial x_1})x_1\frac{\partial}{\partial x_1} = -x_2 \frac{\partial}{\partial x_1} \neq 0.
\end{align*}
$\Psi^s$ and $\Theta^w$ do not commute.


\section*{Exercise 2}
Characteristic equation:
\[
	\lambda^2-4\lambda + 2\alpha = 0.
\]
Solution is $\lambda_{1,2} = 2 \pm \sqrt{4-2\alpha}$. 
\begin{itemize}
	\item For $\alpha \leq 2$ we obtain the solution
	\[
	x(t) = c_1 e^{\lambda_1 t} + c_2 e^{\lambda_2 t}, \quad c_1,c_2 \in \mathbb R.
	\]
	We know that $x(0) = 0$, so $c_1 = -c_2$. Additionally, $x(\pi) = 0$ which means
	\[
	c_1 e^{\lambda_1 \pi} - c_1 e^{\lambda_2 \pi} = 0 \implies e^{\lambda_1 \pi} = e^{\lambda_2 \pi}.
	\]
	That is false since $\lambda_1 \neq \lambda_2$ for $\alpha < 2$. For $\alpha = 2$, we obtain the trivial solution $x\equiv0$. So for $\alpha \leq 2$, there exists only the trivial solution.
	
	\item For $\alpha > 2$, let $\beta \coloneqq -(2-\alpha)$. So we get $\lambda_{1,2} = 2 \pm i\sqrt{2\beta}$ with $\beta > 0$. One solution for the ODE is 
	\[
		x(t) = ce^{2t}(\cos(t\sqrt{2\beta}) + i \sin(t \sqrt{2\beta})).
	\]
	The general, real solution for the ODE is
	\[
		x(t) = c_1 e^{2t}\cos(t\sqrt{2\beta}) + c_2 e^{2t}\sin(t\sqrt{2\beta}), \quad c_1, c_2 \in \mathbb R.
	\]
	For $x(0) = 0$ we get $c_1 = 0$. For $x(\pi) =0$ we get
	\[
		x(\pi) = \underbrace{c_2e^{2\pi}}_{\neq 0}\sin(\pi \sqrt{2\beta}) = 0, \quad c_2 \neq 0.
	\]
	This means that $\sqrt{2\beta} \in \mathbb N$. This is the case if
	\[
		\sqrt2 \sqrt{\alpha -2} = z \iff 2(\alpha-2) = z^2 \iff \alpha = \frac{z^2}{2}+2, \quad z \in \mathbb N
	\]
	So for $\alpha > 2$ and $\alpha = \frac{z^2}{2}+2, z \in \mathbb N$ there exists a non trivial solution.
\end{itemize}

\section*{Exercise 3}
\begin{enumerate}[label=(\roman *)]
	\item The zeros of $f(x) = (x+\alpha)^2(x^2-\alpha)$ are the fixed points. Fixed points only exist for $\alpha \geq 0$ since $(x^2-\alpha) = 0$ for $x = \pm \sqrt{\alpha}$.
	
	\item We see that the positive fixed is not stable since small perturbations move away from the fixed point. The negative fixed point is stable. If $x$ is positive and smaller than the positive fixed point, it approaches the negative fixed point.
	\begin{center}
		\adjustimage{max size={0.9\linewidth}{0.9\paperheight}}{output_1_0.png}
	\end{center}
\end{enumerate}

\end{document}
