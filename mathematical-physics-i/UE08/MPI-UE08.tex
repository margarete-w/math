\documentclass[a4paper,DIV=1]{article}

\usepackage[utf8]{inputenc}
\usepackage[ngerman]{babel}     %Wortdefinitionen
\usepackage{mathtools,amssymb,amsthm}
\usepackage{geometry}
\usepackage{fancyhdr} % Kopfzeile
\usepackage{accents}
\usepackage{enumitem}
\usepackage{framed}
\usepackage{ulem}
\usepackage{adjustbox} % Used to constrain images to a maximum size 

% benutzerdefinierte Kommandos
\newcommand{\crown}[1]{\overset{\symking}{#1}}
\newcommand{\xcrown}[1]{\accentset{\symking}{#1}}

\makeatletter
\newcommand*{\rom}[1]{\expandafter\@slowromancap\romannumeral #1@}
\makeatother
%
\theoremstyle{plain}
\newtheorem{lemma}{Lemma}
\newtheorem*{satz}{Satz}
\newtheorem*{zz}{Zu zeigen}
\newtheorem*{formel}{Formel}



% Kopfzeile
\pagestyle{fancy}
\fancyhf{}
\rhead{362307(Marcel), 395220 (Duc), 391511 (Sofia)}
\lhead{\textbf{MP I}, Jan Techter (Mon 10-12)}
\cfoot{Seite \thepage}

\setlength\parindent{0pt}


\begin{document}
\section*{Exercise 8.1 }
\begin{enumerate}[label=(\alph*)]
	\item Linearise around $(0,0)$:
	\[
		A \coloneqq \frac{df}{dx} \Big \vert_{x=(0,0)} = \begin{pmatrix}
			4x_1^3+x_2 & x_1 \\ -2x_1+x_2^2 & -2+2x_1x_2
		\end{pmatrix} \Bigg\vert_{x=(0,0)} = \begin{pmatrix}
		 0 & 0 \\ 0 & -2
		\end{pmatrix}.
	\]
	We see that the eigenvalues are $\lambda_1 = 0$ and $\lambda_2 = -2$. As $\Re(\lambda_1)=0$, the fixed point $(0,0)$ is not hyperbolic and thus the theorem of Poincaré-Lyapunov is not applicable.
	
	\item The center subspace $E^C(0,0)$ can be obtained by 
	\[
		(A-\lambda_1E)x = 0 \iff Ax = 0 \iff -2x_2 = 0 \iff x_2 = 0.
	\]
	Therefore $E^C(0,0) = \{ (x_1,x_2)  \in \mathbb R^2 : x_2 = 0 \} = \mathrm{span}((1,0))$.
	
	Idea: $W^C(0,0) = \{ (x_1, h(x_1)) \eqqcolon H(x_1) : |x_1| \leq \epsilon \}$. Due to $\Phi^t(W^C(0,0)) \subset W^C(0,0)$, it follows 
	\[
		f(H(x_1)) \in T_{H(x_1)}W^C(0,0) \implies \langle f(H(x_1)), N(x_1) \rangle = 0
	\]
	 with $N(x_1) = \begin{pmatrix}
		 1 \\ h'(x_1)
	 \end{pmatrix}^{\perp} = \begin{pmatrix}
		 h'(x_1) \\ -1
	 \end{pmatrix}$. So we obtain
	 \begin{align}\label{monkey}
	 	\langle \begin{pmatrix}
		 	x_1^4+x_1h(x_1) \\ -2h(x_1)-x_1^2+x_1h^2(x_1)
	 	\end{pmatrix}, \begin{pmatrix}
		 	h'(x_1) \\ -1
	 	\end{pmatrix}\rangle  = 0.
	 \end{align}
	 Now we expand $h(x_1)$ by Taylor Expansion up to degree $5$:
	 \begin{align*}
	 	h(x_1) &= h(0) + h'(0)x_1 + \frac{1}{2}h''(0)x_1^2 + \frac{1}{6}h^{(3)}x_1^3 + \frac{1}{24}h^{(4)}x_1^4 + O(x_1^5) \\
	 	&= \frac{1}{2}ax_1^2 + \frac{1}{6}bx_1^3 + \frac{1}{24}cx_1^4 + O(x_1^5),
	 \end{align*}
	 where $a,b,c$ are yet to be determined. Derivating the local approximation of $h(x_1)$ yields
	 \[
	 	h'(x_1) = ax_1 + \frac{1}{2}bx_1^2+\frac{1}{6}cx_1^3 + O(x^4)
	 \]
	 Substituting in \eqref{monkey} gives
	 \[
	 	 \begin{pmatrix}
	 	x_1^4+x_1h(x_1) \\ -2h(x_1)-x_1^2+x_1h^2(x_1)
	 	\end{pmatrix}, \begin{pmatrix}
	 	ax_1 + \frac{1}{2}bx_1^2+\frac{1}{6}cx_1^3 + O(x^4) \\ -1
	 	\end{pmatrix}\rangle  = 0
	 \]
	 The first row gives a term $O(x^5)$ and therefore
	 \[
	 	2h(x_1)+x_1^2-\underbrace{x_1h^2(x_1)}_{= O(x^5)} + O(x^5)= 0 \iff ax_1^2 + \frac{1}{3}bx_1^3 + \frac{1}{12}cx_1^4 +x_1^2 + O(x^5) = 0.
	 \]
	 Now we get
	 \[
	 	0 = (a+1)x_1^2+\frac{1}{3}bx_1^3 + \frac{1}{12}cx_1^4  \implies a=-1, b=c=0.
	 \]
	 Therefore, $h(x_1) = -\frac{1}{2}x_1^2$ and
	 \[
	 	W^C(0,0) = \{ \begin{pmatrix}
		 	2x_1 \\-x_1^2
	 	\end{pmatrix} : |x_1| \leq \epsilon \}.
	 \]
	 
	 \item Restricting the dynamical system locally on $W^C(0,0)$ results in
	 \[
	 	\dot x_1 = x_1^4+x_1(-0.5x_1^2) \iff \dot x_1 = x_1^4-0.5x_1^3.
	 \]
	 \begin{center}
		\adjustimage{max size={0.5\linewidth}{0.5\paperheight}}{output_1_0.png}
	 \end{center}
 	We that the fixed point $(0,0)$ is stable. For positive $x_1$ values and $x_1 < 0.5$, $\dot x_1$ is negative, which results in convergence to $0$. Similarly, for negative $x_1$ values, $\dot x_1$ is positive, which results in convergence to $0$.

\end{enumerate}

\section*{Exercise 8.2}


\end{document}
