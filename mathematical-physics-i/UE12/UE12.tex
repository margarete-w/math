\documentclass[a4paper]{article}

\usepackage[utf8]{inputenc}
\usepackage[ngerman]{babel}     %Wortdefinitionen
\usepackage{mathtools,amssymb,amsthm}
\usepackage{geometry}
\usepackage{fancyhdr} % Kopfzeile
\usepackage{accents}
\usepackage{enumitem}
\usepackage{framed}
\usepackage{ulem}
\usepackage{adjustbox} % Used to constrain images to a maximum size 
\usepackage{hyperref}

% benutzerdefinierte Kommandos
\newcommand{\crown}[1]{\overset{\symking}{#1}}
\newcommand{\xcrown}[1]{\accentset{\symking}{#1}}

\makeatletter
\newcommand*{\rom}[1]{\expandafter\@slowromancap\romannumeral #1@}
\makeatother
%
\theoremstyle{plain}
\newtheorem{lemma}{Lemma}
\newtheorem*{theorem}{Theorem}
\newtheorem*{zz}{Zu zeigen}
\newtheorem*{formel}{Formel}



% Kopfzeile
\pagestyle{fancy}
\fancyhf{}
\rhead{405592 Victor, 405661 Pantelis, 395220 Duc}
\lhead{\textbf{MP I}, Jan Techter (Mon 10-12)}
\cfoot{Page \thepage}

\setlength\parindent{0pt}


\begin{document}
\section*{Exercise 12.1}
\begin{enumerate}[label=(\roman*)]
	\item Given $f(q,p) = \begin{pmatrix}
		p^{\alpha}q^{\beta} \\ -p^{\alpha + 1} q^{\delta}
	\end{pmatrix}$
	we need to find $\alpha, \beta, \delta \in \mathbb R$ such that $f$ is a Hamiltonian vector field. We use a theorem from the lecture which states that
	\begin{theorem}
		Let $f: \mathbb R^{2n} \to \mathbb R^{2n}$ be a continuous differentiable vector field. It is a Hamiltonian vector field if and only if $\frac{\partial f}{\partial x} \in \mathfrak{Sp}_{2n}$.
	\end{theorem}
	$\mathfrak{Sp}_{2n}$ is the symplectic Lie algebra, i.e. matrices $D$ that satisfy $JD^TJ = D$. If we have $2 \times 2$ matrices, then $JD^TJ = D$ is the same as 
	\[
		\begin{pmatrix}
			0 & 1 \\ -1 & 0
		\end{pmatrix} \begin{pmatrix}
			a & c \\ b & d
		\end{pmatrix} \begin{pmatrix}
		0 & 1 \\ -1 & 0
		\end{pmatrix}  = \begin{pmatrix}
		a & b \\ c & d
		\end{pmatrix}. 
	\]
	Thus we get
	\[
		\begin{pmatrix}
		0 & 1 \\ -1 & 0
		\end{pmatrix} \begin{pmatrix}
			-c & a \\ -d & b
		\end{pmatrix} = \begin{pmatrix}
		a & b \\ c & d
		\end{pmatrix}. 
	\]
	Therefore,
	\[
		\begin{pmatrix}
		-d & b \\ c & -a
		\end{pmatrix} = \begin{pmatrix}
		a & b \\ c & d
		\end{pmatrix}.
	\]
	So, $-a = d \iff 0 = a+d \iff \mathrm{tr}A = 0$. In the end, we must show that $\mathrm{tr}(\frac{\partial f}{\partial x}) = 0$.
	
	First, let's compute the Jacobi matrix
	\[
		\frac{\partial f}{\partial(q,p)} = \begin{pmatrix}
			\beta p^{\alpha} q^{\beta -1} & \alpha p^{\alpha -1}q^{\beta} \\
			- \delta p^{\alpha +1} q^{\delta -1}& -(\alpha +1)p^{\alpha}q^{\delta}
		\end{pmatrix}.
	\]
	The trace is then given by $\beta p^{\alpha} q^{\beta -1} -(\alpha +1)p^{\alpha}q^{\delta}$, which should equal to zero for all $p,q \in \mathbb R$. If $p = 0$ or $q = 0$, the trace is trivially zero for all $\alpha, \beta, \delta$. So, assume $p \neq 0$ and $q \neq 0$. We must find $\alpha, \beta, \delta$ such that $\beta p^{\alpha} q^{\beta -1} = (\alpha +1)p^{\alpha}q^{\delta}$. We divide by $p^{\alpha} \neq 0$, and obtain
	\[
		\beta q^{\beta -1} = (\alpha +1)q^{\delta} \implies \alpha = \beta q^{\beta - 1 - \delta} - 1, \qquad q^{\delta} \neq 0.
	\]
	Now, if $\alpha = \beta q^{\beta - 1 - \delta} - 1$ should hold for fixed $\alpha, \beta, \delta$ and for all $p,q \in \mathbb R$, then $q^{\beta - 1 - \delta}$ must evaluate to a constant for all $q$, which is only the case if $\beta - 1 - \delta$ equals zero. Hence, we get for any $\delta \in \mathbb R$
	\[
		\beta = 1 + \delta.
	\]
	And then,
	\[
		\alpha = \delta.
	\]
	The Hamiltonian vector field has the form
	\[
		f_{Ham}(q,p) =  \begin{pmatrix}
		p^{\delta}q^{1+ \delta} \\ -p^{\delta + 1} q^{\delta}
		\end{pmatrix}.
	\]
	
	\item The Hamilton function can be obtained by integrating $f_{Ham}$ with respect to $p$ and $q$. Both yield the same result
	\[
		H(q,p) = \frac{1}{\delta + 1}p^{\delta + 1}q^{\delta + 1} + \mathrm{const}.
	\]
\end{enumerate}


\end{document}
