\documentclass[a4paper]{article}

\usepackage{fontspec}
\usepackage{fontspec}
\setmainfont[Ligatures=TeX]{Georgia}
\setsansfont[BoldFont="HelveticaNeue-Bold"]{Helvetica Neue}

\usepackage{amsmath, amsthm, amssymb}
\usepackage{mathtools}
\usepackage[many]{tcolorbox}
\usepackage{blindtext}
\usepackage{xcolor}
\usepackage{titlesec}
\usepackage{titling}
\usepackage{enumitem}   
\usepackage{hyperref}


\definecolor{grey}{rgb}{0.5,0.5,0.5}
\definecolor{lightgrey}{rgb}{0.9,0.9,0.9}
\definecolor{darkgrey}{rgb}{0.3,0.3,0.3}
\definecolor{orange}{rgb}{0.94, 0.55, 0.294}
\definecolor{pink}{rgb}{0.94, 0.29, 0.7}
\definecolor{yellow}{rgb}{1, 0.749, 0}

\newcommand{\chapfnt}{\fontsize{16}{19}}
\newcommand{\secfnt}{\fontsize{18}{17}}
\newcommand{\ssecfnt}{\fontsize{14}{14}}
\renewcommand{\hline}{\noindent\makebox[\linewidth]{\rule{12cm}{1pt}}}
\newcommand{\vip}[1]{\textit{\textbf{#1}}}

\titleformat{\chapter}[display]
{\normalfont\chapfnt\bfseries}{\chaptertitlename\ \thechapter}{20pt}{\chapfnt}

\titleformat{\section}
{\normalfont\sffamily\secfnt\mdseries}{\thesection}{1em}{}

\titleformat{\subsection}
{\normalfont\sffamily\ssecfnt\mdseries\color{grey}}{\thesubsection}{1em}{}

\titlespacing*{\chapter} {0pt}{50pt}{40pt}
\titlespacing*{\section} {0pt}{0pt}{16pt}
\titlespacing*{\subsection} {0pt}{12pt}{8pt}


%\usepackage{geometry}
%\setlength{\columnsep}{32mm}
%\geometry{
% left=22mm,
% right=22mm,
% bottom=32mm,
% top = 20mm
%}


\newtcbtheorem[auto counter,number within=section]{theorem}%
  {Theorem}{
  		fonttitle=\upshape, 
  		fontupper=\upshape,
  		boxrule=0pt,
  		leftrule=3pt,
  		arc=0pt,auto outer arc,
  		colback=white,
  		colframe=pink,
  		colbacktitle=white,
  		coltitle=pink,
  		oversize,
  		enlarge top by=1mm,
  		enlarge bottom by=1mm,
    	enhanced jigsaw,
    	interior hidden, 
    	before skip=12pt,
    	overlay={
    		\draw[line width=1.5pt,pink] (frame.north west) -- (frame.south west);
  		}, 
  		frame hidden}{theorem}
  		
\newtcbtheorem[]{issue}%
  {To prove}{
        theorem name,
  		fonttitle=\upshape, 
  		fontupper=\upshape,
  		boxrule=0pt,
  		leftrule=3pt,
  		arc=0pt,auto outer arc,
  		colback=white,
  		colframe=pink,
  		colbacktitle=white,
  		coltitle=pink,
  		oversize,
  		enlarge top by=1mm,
  		enlarge bottom by=1mm,
    	enhanced jigsaw,
    	interior hidden, 
    	before skip=12pt,
    	after skip=0pt,
    	overlay={
    		\draw[line width=1.5pt,pink] (frame.north west) -- (frame.south west);
  		}, 
  		frame hidden}{issue}

\newtcbtheorem[auto counter,number within=section]{lemma}%
  {Lemma}{
  		fonttitle=\upshape, 
  		fontupper=\upshape,
  		boxrule=1pt,
  		toprule=0pt,
  		leftrule=3pt,
  		arc=0pt,auto outer arc,
  		colback=white,
  		colframe=yellow,
  		colbacktitle=white,
  		coltitle=yellow,
  		oversize,
  		enlarge top by=1mm,
  		enlarge bottom by=1mm,
    	enhanced jigsaw,
    	interior hidden, 
    	before skip=12pt,
    	after skip=0pt,
    	overlay={
    		\draw[line width=1.5pt,yellow] (frame.north west) -- (frame.south west);
  		}, 
  		frame hidden}{lemma}
  		
 \newtcbtheorem[auto counter,number within=section]{definition}%
  {Definition}{
  		fonttitle=\upshape, 
  		fontupper=\upshape,
  		boxrule=1pt,
  		toprule=0pt,
  		leftrule=3pt,
  		arc=0pt,auto outer arc,
  		colback=white,
  		colframe=orange,
  		colbacktitle=white,
  		coltitle=orange,
  		oversize,
  		enlarge top by=1mm,
  		enlarge bottom by=1mm,
    	enhanced jigsaw,
    	interior hidden, 
    	before skip=12pt,
    	overlay={
    		\draw[line width=1.5pt,orange] (frame.north west) -- (frame.south west);
  		}, 
  		frame hidden}{definition}
    	
\newtcbtheorem[auto counter,number within=section]{example}%
  {Beispiel}{
  		fonttitle=\upshape, 
  		fontupper=\upshape,
  		boxrule=0pt,
  		leftrule=3pt,
  		arc=0pt,auto outer arc,
  		colback=white,
  		colframe=grey,
  		colbacktitle=white,
  		coltitle=grey,
  		oversize,
  		enlarge top by=1mm,
  		enlarge bottom by=1mm,
    	enhanced jigsaw,
    	interior hidden, 
    	before skip=12pt,
    	overlay={
    		\draw[line width=1.5pt,grey] (frame.north west) -- (frame.south west);
  		}, 
  		frame hidden}{example}
    	
\newtcbtheorem[auto counter,number within=section]{note}%
  {Notiz}{
  		fonttitle=\upshape, 
  		fontupper=\upshape,
  		boxrule=0pt,
  		leftrule=3pt,
  		arc=0pt,auto outer arc,
  		colback=white,
  		colframe=yellow,
  		colbacktitle=white,
  		coltitle=yellow,
  		oversize,
  		enlarge top by=1mm,
  		enlarge bottom by=1mm,
    	enhanced jigsaw,
    	interior hidden, 
    	before skip=12pt,
    	overlay={
    		\draw[line width=1.5pt,yellow] (frame.north west) -- (frame.south west);
  		}, 
  		frame hidden}{note}
  		
\newtcbtheorem[]{important}%
  {Wichtig}{
  		fonttitle=\upshape, 
  		fontupper=\upshape,
  		boxrule=0pt,
  		leftrule=3pt,
  		arc=0pt,auto outer arc,
  		colback=white,
  		colframe=pink,
  		colbacktitle=white,
  		coltitle=pink,
  		oversize,
  		enlarge top by=1mm,
  		enlarge bottom by=1mm,
    	enhanced jigsaw,
    	interior hidden, 
    	before skip=12pt,
    	overlay={
    		\draw[line width=1.5pt,pink] (frame.north west) -- (frame.south west);
  		}, 
  		frame hidden}{important}
    	
\renewcommand{\baselinestretch}{1.4} 
\makeatletter
\let\old@rule\@rule
\def\@rule[#1]#2#3{\textcolor{lightgrey}{\old@rule[#1]{#2}{#3}}}
\makeatother

\begin{document}

\section*{Problem Sheet 01}
\textit{Viet Duc Nguyen (395220)}


\hline 

\subsection*{Exercise 1}

Given a chart $\varphi: U \to V$ and $\tilde \varphi: \tilde U \to V$. Let $\tau: \varphi^{-1} \circ \tilde \varphi$ be the coordinate transformation between the charts. We want to show that
\[
	\det \tilde g(y) = |\det D(\tau(y)) |^2 \det g(\tau(y))
\]
where $g, \tilde g$ denotes the Gramian matrix of $\varphi, \tilde \varphi$ respectively.

\begin{proof}
\begin{align*}
	\det \tilde g(y) = \det (D\tilde \varphi(y)^T D\varphi(y)) = \det ( D( \varphi \circ \tau)(y) )^2 &= \det(D \varphi(\tau(y)) D\tau(y))^2 \\
	&= \det g(\tau (y)) |\det D\tau(y)|^2
\end{align*}
\end{proof}

Next we want to show that the integral over a manifold is independent of the chart $\varphi$. Define:
\[
	I(f) = \int_U f(\varphi(x)) \sqrt{\det g(x)} dx.
\]

\begin{proof}
\begin{align*}
	I(f) = \int_{\tilde U}f(\tilde \varphi (x)) \sqrt{\det \tilde g(x)} dx &= \int_{\tilde U} f(\varphi(\tau(x))) \sqrt{\det D(\tau(x))^2 \det g(\tau(x))} dx \\
	&= \int_U f(\varphi(y)) \sqrt{\det g(y)} dy.
\end{align*}
\end{proof}

\subsection*{Exercise 2}
To show: Let $U \subset \mathbb R^d$ and $f \in C^0(U)$, $u \in C^2(U)$. For every $\varphi \in C^\infty_c(U)$ it holds
\[
	\int_U u\Delta\varphi dx = \int_U f \varphi  dx
\]
iff $\Delta u = f$.
\begin{proof}
First, notice that for $f \in C^2(U)$ and $g \in C^2_c(U)$:
\[
	\int_U D_i^2fg dx = - \int_U D_ifD_ig dx = \int_U f D_i^2g dx
\]
Summing over $i$ yields
\[
	\int_U \Delta fg dx = - \int_U \langle \nabla f, \nabla g \rangle dx = \int_U f \Delta g dx.
\]
Let $\delta u = f$, then
\[
	\int_U u \Delta \varphi dx = \int_U \Delta u \varphi dx = \int_U f\varphi dx.
\]
Now, let $\int_U u \Delta \varphi dx = \int_U f\varphi dx$. We know that
\[
	\int_U u \Delta \varphi dx = \int_U \Delta u \varphi dx = \int_U f\varphi dx
\]
Thus, $\int_U (\Delta u - f) \varphi dx = 0$ for every $\varphi \in C^\infty_c(U)$. So, $\Delta u = f$ after a well known lemma. For completeness, we will state this lemma.

\begin{lemma}{}{}
Let $h \in C(U)$. If $\int_U h(x)g(x) dx = 0$ for every $g \in C^\infty_c(U)$ then $h(x) = 0$ for all $x \in U$.
\end{lemma}
\end{proof}

\subsection*{Exercise 3}

Consider the sequences
\[
	g_n(x) = \frac{2}{\pi} \arctan(nx)  \quad \text{and} \quad f_n(x) = \int^x_0 \sqrt{g_n(t)} dt
\]
\textbf{Goal:} $(f_n)_n$ is a Cauchy sequence in $(C^1_0[0,1], ||\cdot||_\nabla)$ but does not converge in this metric space.

To show that $(f_n)_n$ is indeed a Cauchy sequence observe that
\begin{align*}
	||f_n - f_m||^2_\nabla = \int^1_0 (f_n'(x) - f_m'(x))^2 dx &\leq  \int^1_0 (f_n'(x))^2 + (f_m'(x))^2 dx \\
	 &\leq \int^1_0 |g_n(x) - g_m(x)|dx
\end{align*}

We show that $(g_n)_n$ is a Cauchy sequence with standard norm $|\cdot|$. Let $\epsilon > 0$ and wlog $n \geq m$:
\[
	\forall x > \epsilon: | g_n(x) - g_m(x) | \leq | 1 - g_m(x) | \leq |1 - g_m(\epsilon)| \to 0 \quad \text{für $m \to \infty$}.
\]

Thus,
\begin{align*}
||f_n - f_m||^2_\nabla \leq \int^1_0 |g_n(x) - g_m(x)|dx \leq \epsilon + |1 - g_m(\epsilon)|
\end{align*}

So, $(f_n)_n$ is a Cauchy sequence with respect to $||\cdot||_\nabla$.

Assume $f_n \to f \in C^1_0[0,1]$. Then, $f(0) = 0$ because $f(0) = \int^0_0 f'(x) dx = 0$ for a $f' \in C^0[0,1]$. Nun ist $f$ stetig und somit gibt es ein $\delta > 0$, sodass $f(x) < 1$ für alle $0\leq x < \delta$.

\url{https://math.stackexchange.com/questions/1932023/is-c1a-b-with-the-norm-left-f-right-1-int-ab-left-ft-r/1935026?noredirect=1#comment3975378_1935026}

\hline

\textbf{Fail.}

\begin{enumerate}
\item 

\item Consider $f_n(x) = 2\sqrt{x+\frac{1}{n}}$. It is differentiable with $f'_n(x) = \frac{1}{\sqrt{x+\frac{1}{n}}}$. Thus, $f_n \in C^1_0[0,1]$. 

Let's see if this is a Cauchy sequence. Consider
\begin{align*}
	||f_n - f_m||^2_{\nabla} &= \int^1_0 ( \frac{1}{\sqrt{x+\frac{1}{n}}} -  \frac{1}{\sqrt{x+\frac{1}{m}}})^2 \\
	&= \int^1_0 \frac{1}{x+\frac{1}{n}} - \frac{2}{\sqrt{(x + \frac{1}{n})(x + \frac{1}{m})}} + \frac{1}{x+\frac{1}{m}} dx \\
	&= \int^1_0 \frac{1}{x+\frac{1}{n}} - 
	\frac{2}{\underbrace{\sqrt{x^2 +x\frac{m+n}{mn} + \frac{1}{nm}}}_{sad}}
	+ \frac{1}{x+\frac{1}{m}} dx \\
	&\leq [\ln(x+\frac{1}{n})]^1_0 + [\ln(x+\frac{1}{m})]^1_0
\end{align*}

\textbf{This is no good. Overestimated too generously.}

Assume that $f_n \to f$ with $f \in C^1_0[0,1]$. For every $\epsilon > 0$ we find $N \geq \mathbb N$ such that for all $n \geq N$:
\[
	|| f_n - f||^2_{\nabla} = \int_0^1 \left(\frac{1}{\sqrt{x+\frac{1}{n}}} - f'(x)\right )^2 dx = \int^1_0 \frac{1}{x+\frac{1}{n}} - \frac{2f'(x)}{\sqrt{x + \frac{1}{n}}} + f'(x)^2 dx < \epsilon
\]
Because $f'$ is continuous on $[0,1]$ we can estimate the left hand side term with
\[
	\int^1_0 \frac{1}{x + \frac{1}{n}} - \frac{2c}{\sqrt{x+ \frac{1}{n}}} < \epsilon,
\]
where $c = \inf_{x \in [0,1]} f'(x)$. Integrating then yields
\begin{align*}
	[\ln(x + \frac{1}{n})]^1_0 - 4c[\sqrt{x + \frac{1}{n}}]^1_0 &=\underbrace{ \ln(1 + \frac{1}{n})}_{\to 0} - \underbrace{\ln(\frac{1}{n})}_{\to - \infty} - 4c\underbrace{\sqrt{1 + \frac{1}{n}}}_{\to 1} - 4c\underbrace{\sqrt{\frac{1}{n}}}_{\to 0} \\
\end{align*}
As we see, increasing $n$ will lead to a value bigger than $\epsilon$. Thus, $f$ is not in $C^1_0[0,1]$.
\item
\end{enumerate}


\subsection*{Exercise 4}

\begin{itemize}
\item It holds $\mu(\emptyset) = 0$. To see this, consider the sequence $A_n = \emptyset$. It holds $A_{n+1} \subset A_n$ and $\bigcap A_n = \emptyset$. Thus, $\mu(A_n) = \mu(\emptyset) \to 0$ implies $\mu(\emptyset) = 0$.

\item We show the \vip{monotonicity} of $\mu$. Let $A \subset B$. We know that
\[
	B = (A \cap B) \dot \cup (B \setminus A).
\]
So, $\mu(B) = \mu(A \cap B) + \mu (B \setminus A) \geq \mu(A)$. We only needed \emph{finite additivty} and the \emph{positivity} of $\mu$ to show the monotonicity.

\item Now to the $\sigma$-additivity. Let $A_i \in \mathcal A$, $A_i$ and $A_j$ be pairwise disjoint and $A = \bigcup^\infty_{i=1} A_i$. Define the sequence
\[
	B_n = \bigcup^\infty_{i=n}A_i.
\]	
It holds the following properties:
\begin{itemize}
\item $B_{n+1} \subset B_{n}$
\item $\bigcap B_n = \emptyset$
\item $B_1 = A$
\end{itemize}

From the first two properties it follows that $$\lim_{n \to \infty}\mu(B_n) = \lim_{n \to \infty} \mu(\bigcup_{i=n}^\infty A_n) = 0.$$

We see that $A = B_n \dot \cup \bigcup^{n-1}_{i=1}A_i$.
\begin{align*}
	\mu(A) &= \mu(B_n) + \mu (\bigcup^{n-1}_{i=1}A_i), \quad \forall n \in \mathbb N\\
	&\Downarrow \text{limit} \\
	&= \lim_{n \to \infty} \mu(B_n) +  \mu (\bigcup^{n}_{i=1}A_i) \\
	&= \lim_{n \to \infty} \mu (\bigcup^{n}_{i=1}A_i) \\
	&\Downarrow \text{finite additivity} \\
	&= \lim_{n \to \infty} \sum^n_{i=1}\mu(A_i)\\
	&= \sum^\infty_{i=1}\mu(A_i) < \infty \quad \text{due to monotonicity}
\end{align*}

\end{itemize}



\end{document}