\documentclass{article}

\usepackage[utf8]{inputenc}
\usepackage{mathtools}
\usepackage{amssymb}
\usepackage{amsmath}
\usepackage{amsthm}
\usepackage{enumitem}

%Viktor
\DeclareMathOperator{\A}{\mathcal{A}}
\DeclareMathOperator{\B}{\mathcal{B}}
\renewcommand{\c}{\complement}

\newcommand{\vip}[1]{\textit{\textbf{#1}}}
\newcommand{\R}{\mathbb{R}} % Reelle Zahlen
\newcommand{\N}{\mathbb{N}} % Natürliche Zahlen
\newcommand{\Z}{\mathbb{Z}} % Ganze Zahlen
\newcommand{\C}{\mathbb{C}} % Komplexe Zahlen
\newcommand{\Q}{\mathbb{Q}} % Rationale Zahlen
\newcommand{\K}{\mathbb{K}}
\DeclareMathOperator{\spn}{span}
\DeclareMathOperator{\ran}{ran}


\title{Measure- and Integrationtheory - Assignment 01}
\author{Duc (395220), Viktor (392636), Jacky (391049)}



\begin{document}

\maketitle

\subsection*{Aufgabe 1}
\textbf{Let $(X, \mathcal A, \mu)$ be a measure space.
Show that $\mathcal{B} = \big\{ A \in \mathcal A: \mu(A) = 0 \text{ or } \mu(A^{c}) = 0\big\}$ is a $\sigma$ algebra on $X$.}

\begin{proof}
    We show the three properties from definition 3.3.
    \begin{enumerate}[label = (\roman*)]
        \item 
        We have $\mu(X^{\c}) = \mu(\emptyset) = 0$ by definition 3.4, implying $X \in \B$.
        
        \item
        Let $A \in \B$.
        Then either $\mu(A) = 0$ or $\mu\big(A^{\c}\big) = 0$ holds.
        
        In the first case $A^{\c} \in \B$ as $\mu\big((A^{\c})^{\c}\big) = \mu(A) = 0$ holds.
        In the second case $A^{\c} \in \B$ is immediate.
        
        \item
        Let $(A_k)_{k \in \N} \subset \B$.
        We first show that $\mu$ is monotone, i.e. $A \subset B$ $\implies$ $\mu(A) \le \mu(B)$ for $A,B \in \A$.
        As $B = (B \setminus A) \cup A$ is a disjoint union, we have $\mu(B) = \mu(B \setminus A) + \mu(A) \ge \mu(A)$ as $\mu(C) \ge 0$ for all $C \in \A$.
        (We have $B \setminus A= B \cap A^{\c} = \big(B^{\c} \cup A\big)^{\c} \in \A$ by properties two and three).
        This argument holds for countable families, too.
        
        Next we show that $\mu\left(\bigcup_{k \in \N} A_k\right) \le \sum_{k \in \N}\mu(A_k)$.
        We can rewrite
        \begin{equation*}
            \bigcup_{k \in \N} A_k
            = \bigcup_{k \in \N} \left(A_k \setminus \bigcup_{j = 1}^{k - 1} A_k\right)    
        \end{equation*}
        as a disjoint union, implying the statement by the monotonicity of $\mu$.
        
        \underline{Case 1: $\mu(A_k) = 0$ for all $k \in \N$.}
        We have
        \begin{equation*}
            0
            \le \mu\left(\bigcup_{k \in \N} A_k\right)
            \le \sum_{k \in \N}\mu(A_k)
            = 0.
        \end{equation*}
        
        \underline{Case 2: There exists a subset $A \subset \N$ such that $\mu(A_k^{\c}) = 0$ for $k \in A$.} 
        By \textsc{deMorgan}s laws and the monotonicity of $\mu$ it holds
        \begin{equation*}
            \mu\left(\left(\bigcup_{k \in \N} A_k\right)^{\c} \right)
            = \mu\left(\bigcap_{k \in \N} A_k^{\c} \right)
            \le \mu(A_i^{\c})
            = 0
        \end{equation*}
        for some $i \in A$, as $\bigcap_{k \in \N} B_k \subset B_i$.
    \end{enumerate}
\end{proof}

\textbf{
    Let $(X, \A)$ be a measurable space and $X_0 \subset X$ a non-emptyset.
    Show that the trace $\sigma$ algebra $\B \coloneqq \big\{A \cap X_0: A \in \A \big\}$ is a $\sigma$-algebra.
}

\begin{proof}
    We show the three properties from definition 3.3.
    \begin{enumerate}[label = (\roman*)]
        \item
        We have $X_0 = X \cap X_0 \in \B$, as $X \in \A$ as $\A$ is a $\sigma$ algebra $(\star)$.
        
        \item
        For $B \coloneqq A \cap X_0 \in \B$ with $A \in \A$
        \begin{equation*}
            X_0 \setminus B
            = X_0 \setminus (A \cap X_0)
            = X_0 \setminus A 
            = X_0 \cap A^{\c} \in \B
        \end{equation*}
        holds, as by $(\star)$, $A^{\c} \in \A$.
        
        \item
        For $(B_k \coloneqq A_k \cap X_0)_{k \in \N} \subset \B$ with $(A_k)_{k \in \N} \subset \A$
        \begin{equation*}
            \bigcup_{k \in \N} B_k
            = \bigcup_{k \in \N} (A_k \cap X_0)
            = X_0 \cap \bigcup_{k \in \N} A_k
            \in \B,
        \end{equation*}
        holds, as $\bigcup_{k \in \N} A_k \in \A$ by $(\star)$.
    \end{enumerate}
\end{proof}

\subsection*{Aufgabe 2}
\textbf{To show:
    $$
    \mu (\liminf_{n \to \infty} A_n) \overset{(1)}{\leq} \liminf_{n \to \infty} \mu(A_n) \overset{(2)}{\leq} \limsup \mu(A_n) \overset{(3)}{\leq} \mu(\limsup A_n).
$$ 
}


Zur Erinnerung: 
$$\liminf_{n \to \infty} A_n = \bigcup_{k=1}^\infty \bigcap_{n = k}^\infty A_n = \lim_{n \to \infty} \bigcap_{i=n}^\infty A_i$$

\begin{proof}

Seien $A_i \in \mathcal A$ für alle $i \in \mathbb N$.

\begin{enumerate}[label=(\arabic*)]
    \item
    Sei $B_n \coloneqq \bigcap_{i = n}^\infty A_i$. Dann ist die Folge $(B_n)_{n \in \N}$ aufsteigend: $B_n \subset B_{n+1}$ gilt für alle $n \in \mathbb N$.
    Aus Satz 3.7 (a) folgt
    $$
        \lim_{n \to \infty} \mu(B_n) = \mu(\lim_{n \to \infty} B_n) = \mu(\liminf_{n \to \infty} A_n).
    $$
    
    Nun ist $B_n = \bigcap_{i = n}^\infty A_i \subset A_k$ für alle $k \geq n$. Daher $\mu(B_n) \leq \mu(A_k)$ für alle $k \geq n$. Somit auch $\mu(B_n) \leq \inf_{k \geq n}{\mu(A_k)}$. Grenzwertbildung ergibt
    $$
       \mu(\liminf_{n \to \infty} A_n) \le  \liminf_{n \to \infty} \mu(A_k).
    $$

    \item
    Das ist klar aus der Definition des Limes superiors und Limes inferiors, da für alle $\varepsilon > 0$ und für fast alle Folgenglieder einer Folge $(x_n)_{n \in \N} \subset \R$ 
    \begin{equation*}
        \liminf_{n \to \infty} x_n - \varepsilon
        < x_n
        < \limsup_{n \to \infty} x_n + \varepsilon
    \end{equation*}
    gilt.
    
    \item
    Das Maß ist endlich. Daher ist $\mu(A_i)$ endlich für alle Mengen $A_i$.
    Damit können wir die Stetigkeit von oben von $\mu$ ausnutzen $(*)$.
    Definiere die Menge $B_n \coloneqq \bigcup_{i=n}^\infty A_i$. Aus der Definition folgt, dass $(B_n)_{n \in \mathbb N}$ absteigend ist. Damit folgt
    $$
        \lim_{n \to \infty} \mu(B_n)
        \overset{(*)}{=} \mu (\lim_{n \to \infty} B_n)
        = \mu \left(\bigcap_{i=1}^\infty \bigcup_{n=i}^\infty A_n\right).
    $$
    Nun ist $B_n \supset A_i$ für alle $i \geq n$. Damit ist $\mu(B_n) \geq \mu(A_i)$ für alle $i \geq n$. Sodann haben wir $\mu(B_n) \geq \sup_{i \geq n}\mu(A_i)$. Grenzwertbildung ergibt dann
    $$
        \limsup_{n \to \infty} \mu(A_n)
        \leq \mu \left(\bigcap_{i=1}^\infty \bigcup_{n=i}^\infty A_n\right).
    $$
\end{enumerate}
\end{proof}

Zum Schluss noch ein Beispiel, wo $\mu (\liminf_{n \to \infty} A_n) < \liminf_{n \to \infty} \mu(A_n)$ gilt. Betrachte die Lebesge-messbaren Mengen mit dem Lebesguemaß in $\mathbb R^2$. Sei $A_i = [0,1]^2$, falls $i$ gerade ist; andernfalls sei $A_i = [42,43]^2$. $\mu (\liminf_{n \to \infty} A_n) = 0$, da $\liminf_{n \to \infty} A_n$ die leere Menge ist. Wir sehen, dass $\liminf_{n \to \infty} \mu(A_n) = 1$ ist.

\end{document}