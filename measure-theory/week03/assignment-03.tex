\documentclass{article}

\usepackage[utf8]{inputenc}
\usepackage{mathtools}
\usepackage{amssymb}
\usepackage{amsmath}
\usepackage{amsthm}
\usepackage{enumitem}
\usepackage{bbm}

%Viktor
\DeclareMathOperator{\A}{\mathcal{A}}
\DeclareMathOperator{\B}{\mathcal{B}}
\renewcommand{\c}{\complement}

\newcommand{\vip}[1]{\textit{\textbf{#1}}}
\newcommand{\R}{\mathbb{R}} % Reelle Zahlen
\newcommand{\N}{\mathbb{N}} % Natürliche Zahlen
\newcommand{\Z}{\mathbb{Z}} % Ganze Zahlen
\newcommand{\C}{\mathbb{C}} % Komplexe Zahlen
\newcommand{\Q}{\mathbb{Q}} % Rationale Zahlen
\newcommand{\K}{\mathbb{K}}
\DeclareMathOperator{\spn}{span}
\DeclareMathOperator{\ran}{ran}



\title{Measure- and Integrationtheory - Assignment 03}
\author{Duc (395220), Viktor (392636), Jacky (391049)}


%%
% TO-DO:
%   [ x ] Aufgabe 1 (Duc)
%   [ x ] Aufgabe 2 (Duc)
%   [ x ] Aufgabe 3 (Duc)
%
%%


\begin{document}

\maketitle

\section*{Exercise 1}
\begin{enumerate}[label=(\roman*)]
    \item Seien zwei messbare Räume $(\Omega, \mathcal A)$ und $(\Omega', \mathcal A')$ gegeben. Sei $\mathcal E \supset \mathcal A$ und $\mathcal E' \subset \mathcal A'$. Sei $f$ zudem $(\mathcal A, \mathcal A')$-messbar.
    
    Sei $A \in \mathcal E'$. Dann ist $A \in \mathcal A'$. Wegen der $(\mathcal A, \mathcal A')$-Messbarkeit ist das Urbild $f^{-1}(A)$ in $\mathcal A$ enthalten und somit auch in $\mathcal E$. Somit ist $f$ auch $(\mathcal E, \mathcal E')$-messbar.
 
    \item Seien zwei messbare Räume $(\Omega, \mathcal A)$ und $(\Omega', \mathcal A')$ gegeben. Sei $\Omega'_0 \subset \Omega'$. Sei $f: \Omega \to \Omega'_0$. Zeige, dass $f$ genau dann $(\mathcal A, \mathcal A')$-messbar ist, wenn $f$ $(\mathcal A, \mathcal A'|_{\Omega'_0})$ messbar ist.
    
    "$\implies$": Sei $f$ $(\mathcal A, \mathcal A')$-messbar. Sei $A \in \mathcal A'|_{\Omega'_0}$. Dann ist $A = A' \cap \Omega'_0$ für ein $A' \in \mathcal A'$. 
    $$
        f^{-1}(A) = f^{-1}(A' \cap \Omega'_0) = f^{-1}(A') \cap \underbrace{f^{-1}(\Omega'_0)}_{= \Omega} = f^{-1}(A') \in \mathcal A.
    $$
    Somit ist $f$ $(\mathcal A, \mathcal A'|_{\Omega'_0})$-messbar.

    "$\impliedby$": Sei $f$ $(\mathcal A, \mathcal A'|_{\Omega'_0})$-messbar. Sei $A \in \mathcal A'$. Dann ist $f^{-1}(A \cap \Omega'_0) \in \mathcal A$. Nun ist $f^{-1}(A \cap \Omega'_0) = f^{-1}(A) \cap f^{-1}(\Omega'_0) = f^{-1}(A)$ und daher ist $f^{-1}(A)  \in \mathcal A$. Also ist $f$ $(\mathcal A, \mathcal A')$-messbar.
\end{enumerate}




\section*{Exercise 2}

Sei $(\Omega, \mathcal A)$ ein messbarer Raum und sei $(X,d)$ ein metrischer Raum. Seien $f,f_n: \Omega \to X$ Abbildungen mit $f_n \to f$. \textbf{Zeige, dass die $(\mathcal A, \mathfrak B(X))$-Messbarkeit von $f_n$ die $(\mathcal A, \mathfrak B(X))$-Messbarkeit von $f$ impliziert.}

\begin{itemize}
    \item Sei $A\subset X$ abgeschlossen. Sei $U_n = \{ x \in X: d(x,A) < \frac{1}{n}\}$. Zuerst zeigen wir 
    $$
        f^{-1}(A) = \bigcap_{n \in \mathbb N} \bigcup_{k \in \mathbb N} \bigcap_{j=k}^\infty f_j^{-1}(U_n)
    $$
    oder handlicher
    $$
        f^{-1}(A) = \bigcap_{n \in \mathbb N}\liminf_{j \to \infty}f_j^{-1}(U_n).
    $$
    
    \begin{enumerate}
        \item "$\subset$": Sei $x \in f^{-1}(A)$. Dann gibt es ein $a \in A$ mit 
        $$
            f(x) = \lim_{i \to \infty} f_i(x) = a.
        $$ 
        Sei $n \in \mathbb N$ beliebig. Dann gibt es einen Index $I \in \mathbb N$, sodass für alle $i \geq I$ gilt 
        \begin{align*}
            d(f_i(x), a) < \frac{1}{n}.
        \end{align*}
        Sei $j \geq I$. Dann ist $f_j(x) \in U_n$, da 
        \begin{align*}
            d(f_j(x), A) \leq d(f_j(x), a) < \frac{1}{n}.
        \end{align*}
        Somit ist $x \in f_j^{-1}(U_n)$ für alle $j \geq I$. Das ist nichts anderes als 
        $$
            x \in \liminf_{j \to \infty} f^{-1}_j(U_n).
        $$
        Da $n \in \mathbb N$ beliebig war, folgt 
        $$
            x \in \bigcap_{n \in \mathbb N} \liminf_{j \to \infty} f^{-1}_j(U_n)
        $$
        und somit 
        $$  
            f^{-1}(A) \subset \bigcap_{n \in \mathbb N} \liminf_{j \to \infty} f^{-1}_j(U_n).
        $$

        \item "$\supset$": Sei $x \in \bigcap_{n \in \mathbb N} \liminf_{j \to \infty}f^{-1}_j(U_n)$. Sei $n \in \mathbb N$. Dann gibt es einen Index $I(n) \in \mathbb N$, sodass für alle $j \geq I(n)$ gilt, dass $f_j(x) \in U_n$. Nun ist 
        \begin{align*}
            f_j(x) \in U_n &\iff d(f_j(x), A)< \frac{1}{n}\\ &\iff \exists a_{jn} \in A: d(f_j(x), a_{jn}) < \frac{1}{n}.
        \end{align*}            
        Außerdem gibt es einen Index $I^*$, sodass 
        $$
            \forall j \geq I^*: d(f(x), f_j(x)) < \frac{1}{n}.
        $$
        Sei $I = \max\{ I^*, I(n) \}$. Dann gilt für alle $j \geq I$, dass 
        \begin{align*}
            d(f(x), a_{jn}) \leq d(f(x), f_j(x)) + d(f_j(x), a_{jn}) < \frac{1}{n} + \frac{1}{n} = \frac{1}{2n}
        \end{align*}
        Damit gilt $d(f(x), A) < \frac{1}{2n}$ für alle $n \in \mathbb N$. Das heißt, 
        $$
            d(f(x), A) = 0
        $$
        und somit $f(x) \in A$. Also $x\in f^{-1}(A)$ und daher 
        $$  
            f^{-1}(A) \supset \bigcap_{n \in \mathbb N} \liminf_{j \to \infty} f^{-1}_j(U_n).
        $$
    \end{enumerate}

    \item Nun wollen wir die $(\mathcal A, \mathfrak B(X))$-Messbarkeit von $f$ zeigen. Da $\mathfrak B(X)$ durch die abgeschlossene Mengen in $X$ erzeugt wird, reicht es, nur diese Mengen zu überprüfen. Das heißt, wir zeigen $f^{-1}(A) \in \mathcal A$ für alle abgeschlossenen Teilmengen $A \subset X$. Sei $A \subset X$ abgeschlossen. Wir haben gezeigt
    $$
        f^{-1}(A) = \bigcap_{n \in \mathbb N} \bigcup_{k \in \mathbb N} \bigcap_{j=k}^\infty f_j^{-1}(U_n).
    $$

    Nun ist $U_n$ offen in $X$ für alle $n \in \mathbb N$ und es folgt $U_n \in \mathfrak B(X)$. Damit ist $f_j^{-1}(U_n) \in \mathcal A$ für alle $j \in \mathbb N$ wegen der Messbarkeit von $f_j$. Abzählbare Schnitte und Vereinigungen von messbaren Mengen wie $f_j^{-1}(U_n)$ sind auch wieder in $\mathcal A$ enthalten. Also 
    $$
        f^{-1}(A) \in \mathcal A.
    $$
    Die Funktion $f$ ist $(\mathcal A, \mathfrak B(X))$-messbar.
\end{itemize}




\section*{Exercise 3}

Sei $(\Omega, \mathcal A, \mu)$ ein Maßraum und $f: \Omega \to \overline{\mathbb R}$ $\mu$-integrierbar. \textbf{Zeige, dass für alle $\epsilon > 0$ ein $\delta > 0$ existiert, sodass}

$$
    \forall A \in \mathcal A: \mu(A) < \delta \implies \left|\int_A f d\mu\right| < \epsilon.
$$

\begin{enumerate}
    \item Sei $\epsilon > 0$. Sei $f$ beschränkt, d.h. es gibt ein $C > 0$, sodass
    $$
        f(x) < c \quad \forall x \in \Omega.
    $$ 
    Definiere $\delta = \frac{\epsilon}{C}$. Sei $A \in \mathcal A$ mit $\mu(A) < \delta$. Dann folgt
    \begin{align*}
        \left|\int_A f d\mu\right| \leq \int_A |f| d\mu \leq C \int_A 1 d\mu = C \mu(A) < \epsilon.
    \end{align*}

    \item Sei $f$ unbeschränkt und $\epsilon > 0$. Da $f$ $\mu$-integrierbar ist, gilt 
    \begin{align*}
        \int f^+ d\mu &= \sup\left\{ \int g d\mu : g \leq f^+, g \in \mathfrak E_+ \right\}, \\
        \int f^- d\mu &= \sup\left\{ \int g d\mu : g \leq f^-, g \in \mathfrak E_+ \right\}.
    \end{align*}
    Wir betrachten im Folgenden $f^+$; der Fall $f^-$ geht analog. 
    \begin{enumerate}
        \item Es gibt eine Elementarfunktion $g \in \mathfrak E_+$ mit $g \leq f$ und 
        $$
            \int (f^+ - g) d\mu < \frac{\epsilon}{2}.
        $$
        Für jedes $A \in \mathcal A$ gilt 
        \begin{align}\label{klo}
            \int_A (f^+ - g) d\mu < \frac{\epsilon}{2}.
        \end{align}

        \item Die Elementarfunktion $g$ ist beschränkt. Damit gibt es, wie wir im ersten Teil gezeigt haben, ein $\delta^+ > 0$, sodass für alle messbaren Mengen $A$ mit $\mu(A) < \delta^+$ gilt 
        \begin{align}\label{kraft}
            \int_A g d\mu < \frac{\epsilon}{2}.
        \end{align}

        \item Sei $A$ eine messbare Menge mit $\mu(A) < \delta^+$. Es gilt 
        \begin{align*}
            \int_A f^+ d\mu \overset{\eqref{klo}}{<} \frac{\epsilon}{2} + \int_A g d\mu \overset{\eqref{kraft}}{<} \epsilon.
        \end{align*}

        \item Die Schritte (a) bis (c) verlaufen analog für $f^-$ und wir erhalten ein $\delta^- > 0$. Sei $A$ eine messbare Menge mit $\mu(A) < \delta$, wobei $\delta = \min\{ \delta^+, \delta^- \}$. Nun gilt 
        \begin{align*}
            \left|\int_A f d\mu\right| = \Bigg|\underbrace{\int_A f^+ d\mu}_{< \epsilon} - \underbrace{\int_A f^- d\mu}_{<\epsilon}\Bigg| < \epsilon.
        \end{align*}
        Damit haben wir die Aussage bewiesen.
    \end{enumerate}
\end{enumerate}

\end{document}