\documentclass[9pt]{extarticle}

\usepackage[ngerman]{babel}
\usepackage[utf8]{inputenc}
\usepackage{amsthm,amsmath,amssymb,mathtools}

\usepackage{kpfonts}
\usepackage{geometry}
\usepackage{fancyhdr} % Kopfzeile
\usepackage{mathabx} % orthogonal direct sum sign
\usepackage{enumitem}
\usepackage{framed}
\usepackage{ulem}
\usepackage{wasysym} %lightning symbol
\usepackage{titlesec}
\titleformat*{\section}{\large\bfseries}


% benutzerdefinierte Kommandos
\newcommand{\crown}[1]{\overset{\symking}{#1}}
\newcommand{\xcrown}[1]{\accentset{\symking}{#1}}
\newcommand{\R}{\mathbb{R}}

\def\doubleunderline#1{\underline{\underline{#1}}}

% roman numbers
\makeatletter
\newcommand*{\rom}[1]{\expandafter\@slowromancap\romannumeral #1@}
\makeatother



\newtheorem*{theorem}{Theorem}

\newtheoremstyle{named}{}{}{\itshape}{}{\bfseries}{.}{.5em}{\thmnote{#3}#1}
\theoremstyle{named}
\newtheorem*{namedtheorem}{}






% Kopfzeile
\pagestyle{fancy}
\fancyhf{}
\rhead{Duc (395220), Linda (349481), Frido (374658)}
\lhead{\textbf{Wir rocken Numerik I WS 2018/19} - }
\cfoot{Seite \thepage}


\setlength\parindent{0pt}


\begin{document}
	
\section*{Aufgabe 1}
\begin{enumerate}[label=(\alph*)]
	\item Der absolute Fehler $f_1$ beträgt $f_1 = \tilde s_1 - s_1 = x_1 - x_1 = 0.$ Für $f_2$ ergibt sich:
	\begin{align*}
		f_2 = \tilde s_2 - s_2 = (\tilde s_1 \oplus x_2) - (x_1 + x_2) = (x_1 \oplus x_2) - (x_1 + x_2)= (x_1+x_2)(1+\epsilon_{f_2})-(x_1 + x_2) 
		= \epsilon_{f_2}(x_1 + x_2)
	\end{align*}
	Der absolute Fehler $f_3$ lautet
	\begin{align*}
		f_3 = \tilde s_3 - s_3 &= (\tilde s_2 \oplus x_3) - (x_1+x_2+x_3)\\
		&= ((x_1 \oplus x_2) \oplus x_3) - (x_1+x_2+x_3)\\
		&= ((x_1+x_2)(1+\epsilon_{f_2}) \oplus x_3) - (x_1+x_2+x_3)\\
		&= ((x_1+x_2 + \epsilon_{f_2}(x_1+x_2)) \oplus x_3) - (x_1+x_2+x_3) \\
		&= ((x_1+x_2+x_3 + \epsilon_{f_2}(x_1+x_2)) (1+\epsilon_{f_3})) - (x_1+x_2+x_3) \\
		&= \epsilon_{f_2}(x_1+x_2) + \epsilon_{f_3}(x_1+x_2+x_3+ \epsilon_{f_2}(x_1+x_2)) \\
		&= \epsilon_{f_3}(x_1+x_2+x_3) +  \epsilon_{f_2}(x_1+x_2)(1+\epsilon_{f_3}) \\
		&=  \epsilon_{f_3}(x_1+x_2+x_3) + f_2(1+\epsilon_{f_3})
	\end{align*}
	Für $f_4$ ergibt sich dann
	\begin{align*}
		f_4 = \tilde s_4 - s_4 &= (\tilde s_3 \oplus x_4) - \underbrace{(x_1+x_2+x_3+x_4)}_{ \coloneqq \xi} \\
		&= \Big(\big((x_1+x_2+x_3 + \epsilon_{f_2}(x_1+x_2)) (1+\epsilon_{f_3}) \big) \oplus x_4\Big) - \xi \\
		&= \big((x_1+x_2+x_3 + \epsilon_{f_2}(x_1+x_2))(1+\epsilon_{f_3}) + x_4 \big)  (1+\epsilon_{f_4}) - \xi \\
		&= \big( x_1+x_2+x_3 + x_4 +\epsilon_{f_2}(x_1+x_2) + \epsilon_{f_3}(x_1+x_2+x_3+\epsilon_{f_2}(x_1+x_2)) \big)(1+\epsilon_{f_4}) - \xi \\
		&= ( \xi +f_2 + f_3)(1+\epsilon_{f_4}) - \xi  \\
		&=  f_2+f_3 + \epsilon_{f_4}(\xi +f_2 + f_3) \\
		&= \epsilon_{f_4}\xi + (f_2+f_3)(1 + \epsilon_{f_4})\\
		&= \epsilon_{f_4}(x_1+x_2+x_3+x_4) + (f_2+f_3)(1 + \epsilon_{f_4})
	\end{align*}
	
	
	\item \textbf{Diskussion:} Betrachte den relativen Fehler $\tau_{rel}$
	\[
			\tau_{rel} = \frac{|s_n+f_n|}{|s_n|} = 1 + \frac{|f_n|}{|\sum^{n}_{i=0}x_i|} \leq  (1+\epsilon^*)^{n}\frac{ \sum^{n}_{i=0}|x_i|}{|\sum^{n}_{i=0}x_i|}.
	\]
	Da $ \sum^{n}_{i=0}|x_i| \geq |\sum^{n}_{i=0}x_i|$, kann $\tau_{rel}$ besonders groß werden, wenn sich die einzelnen Summanden $x_i$ gegenseitig auslöschen. Das heißt, wenn $\sum x_i \approx 0$ gilt. Andererseits wird der relative Fehler klein, wenn alle Summanden das selbe Vorzeichen besitzen.
	
	Wir sehen auch, dass der relative Fehler größer wird, umso mehr Summanden hinzuaddiert werden, da $(1+\epsilon^*)^n \geq (1+\epsilon^*)^{n-1} \geq ... \geq 1$.\\
	
	Im folgenden werden wir die Abschätzung per Induktion über $n$ beweisen.
	\begin{proof}
		Der Induktionsanfang für $n=1$ ergibt: $f_1 = 0 \leq (1+\epsilon^*-1)|x_1| \quad \checkmark$ 
		
		Angenommen, die Abschätzung gelte für ein beliebiges $n \in \mathbb N$. Für $n \leadsto n+1$ ergibt sich:
		\begin{align*}
			f_{n+1} = \tilde s_{n+1} - s_{n+1} &= \tilde s_{n} \oplus x_{n+1} - s_{n+1}\\
			&= (s_n + f_n) \oplus x_{n+1} - s_{n+1} \\
			&= (s_{n+1} + f_n)(1+ \epsilon) - s_{n+1} \\
			&= s_{n+1} + f_n + \epsilon(s_{n+1} + f_n) - s_{n+1} \\
			&= f_n(1+\epsilon) + \epsilon s_{n+1} \\
			& \overset{(IV)}{\leq} \big((1+\epsilon^*)^n-1\big) \sum^{n}_{i=0}|x_i| (1+\epsilon) +  \epsilon \sum^{n+1}_{i=0}|x_i| \\
			& \leq \big((1+\epsilon^*)^n-1\big) \sum^{n+1}_{i=0}|x_i| (1+\epsilon^*) +  \epsilon^*\sum^{n+1}_{i=0}|x_i| \\
			&=  [\big((1+\epsilon^*)^n-1\big)(1+\epsilon^*) + \epsilon^* ] \sum^{n+1}_{i=0}|x_i|  \\
			&= [(1+\epsilon^*)^{n+1}-1 ] \sum^{n+1}_{i=0}|x_i| 
		\end{align*}
		Nach Prinzip der vollständigen Induktion folgt die Behauptung für alle $n > 0$.
	\end{proof}

	\item \begin{proof}[Beweis per Induktion über $n$]
		Für $n=1$ setze $\delta_1 \coloneqq 0$ und dann gilt:
		\[
			\tilde s_1 = x_1 = x_1(1+0) \text{ und } -\epsilon^{*} \leq 0 \leq \epsilon^{*} \quad \quad \checkmark
		\]
		Angenommen, es gäbe $\delta_i$ für $i=1,...,n$, sodass $\tilde s_n = \sum x_i(1+\delta_i)$ und $(1-\epsilon^*)^n-1 \leq \delta_i \leq (1+\epsilon^*)^n-1$. Betrache nun $n \leadsto n+1$. Es gilt
		\begin{align*}
			\tilde s_{n+1} = \tilde s_n \oplus x_{n+1} \overset{IV}{=} \left(\sum^n_{i=1} x_i(1+\delta_i)\right) \oplus x_{n+1} &= [\left(\sum^n_{i=1} x_i(1+\delta_i)\right) + x_{n+1} ](1+\epsilon)\\ &= \sum^n_{i=1} x_i(1+\delta_i)(1+\epsilon) + x_{n+1}(1+\epsilon)
		\end{align*}
		Definiere $\tilde \delta_i \coloneqq (1+\delta_i)(1+\epsilon) - 1$ für alle $i = 1,...,n$ und $\tilde \delta_{n+1} \coloneqq \epsilon$. Dann gilt
		\[
			\tilde s_{n+1} = \sum^{n+1}_{i=1} x_i(1 + \tilde \delta_i).
		\]
	 	Zum Schluss überprüfen wir noch, dass $\tilde \delta_i$ im richtigen Intervall liegt. Es gilt $\epsilon > 0$. Für $\tilde \delta_{n+1}$ ergibt sich
	 	\[
	 			 	\tilde \delta_{n+1} = \epsilon \leq (1+\epsilon)-1 \leq (1+\epsilon)^n-1 \leq (1+\epsilon^*)^n-1.
	 	\]
		Die Abschätzung in die andere Richtung ergibt
		\[
				 	\tilde \delta_{n+1} = \epsilon \geq (1-\epsilon)-1 \geq (1-\epsilon)^n-1 \geq (1-\epsilon^*)^n-1.
		\]
		Für alle $i = 1,...,n$ gilt 
		\[
			\tilde \delta_i = (1+\delta_i)(1+\epsilon) -1 \leq (1+\epsilon^*)^{n+1}-1, \text{ da wegen IV gilt: } 1+\delta_i \leq (1+\epsilon^*)^n.
		\]
		In die andere Richtung ergibt sich
		\[
			\tilde \delta_i = (1+\delta_i)(1+\epsilon) -1 \overset{IV}{\geq} (1-\epsilon^*)^n(1+\epsilon) - 1 \geq (1-\epsilon^*)^{n+1}-1
		\]
		Tatsächlich liegen die $\tilde \delta$ im richtigen Intervall. Nach Prinzip der vollständigen Induktion ist die Aussage für alle $n>0$ gezeigt.
		\end{proof}
		
		Wir wollen nun folgende Abschätzung zeigen:
		\[
			|\delta_i| \leq \frac{n\epsilon^*}{1-n\epsilon^*}, \quad \text{ falls } n\epsilon^* <1.
		\]

		\begin{proof}
			Sei $n \in \mathbb N_{>0}$. Sei $x$ reell mit $x < \frac{1}{n}$. Daraus folgt, dass $1-nx > 0$.
			\begin{align*}
				(1+x)^n - 1 \leq \frac{nx}{1-nx} \overset{1-nx > 0}{\iff}  \big( (1+x)^n - 1  \big)(1-nx) \leq nx \iff (1+x)^n(1-nx) - 1 \leq 0.
			\end{align*}
			Wir zeigen nun per Induktion über $n$, dass $(1+x)^n(1-nx) - 1$ immer negativ bzw. null ist für alle $x < \frac{1}{n}$. Für $n=1$ ergibt sich 
			\[
				(1+x) (1-x) - 1 = 1-x^2-1 = -x^2 \leq 0, \quad \forall x \in \mathbb R.
			\]
			Die Behauptung gelte für ein $n \in \mathbb N$. $n \leadsto n+1$.
			\begin{align*}
				(1+x)^{n+1}(1-(n+1)x) - 1 = (1+x)^{n+1}(1-nx-x)-1 = (1+x)^{n+1}(1-nx) -  x(1+x)^{n+1}-1
			\end{align*}
			Nach Induktionsvoraussetzung gilt
			\begin{align*}
				(1+x)^{n+1}(1-nx) -  x(1+x)^{n+1}-1 &= (1+x)^{n}(1-nx)(1+x) - x(1+x)^{n+1} - 1\\ 
				&= \underbrace{(1+x)}_{\geq 0}(\underbrace{(1+x)^{n}(1-nx)}_{\leq 0 \text{ nach IV}}-\underbrace{x(1+x)^{n}}_{\geq 0})-1 \leq 0, \quad \forall x \in \mathbb R.
			\end{align*}
			Damit ist $(1+x)^n(1-nx) - 1 \leq 0$ und es folgt, dass $(1+x)^n - 1 \leq \frac{nx}{1-nx}$ für alle $xn < 1$. Für $x = \epsilon^*$ folgt die Behauptung, die zu beweisen war.
		\end{proof}
\end{enumerate}

\section*{Aufgabe 3}
Aus der Dreiecksungleichung ergibt sich $|a+b|-|b|\leq |a|$. Sei $a \coloneqq \alpha+\beta$ und $b \coloneqq -\beta$. Damit $|\alpha| - |\beta| \leq |\alpha+\beta|$. Sei $b \coloneqq -a$, so ergibt sich $|\beta| - |\alpha| \leq |\alpha+\beta|$. Insgesamt 
\[
	|\alpha+\beta| \geq \Vert |\alpha|-|\beta| \Vert.
\]
Die Ungleichung gilt für jede Norm, da wir nur die Dreiecksungleichung ausnutzen.

\end{document}
