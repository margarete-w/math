\documentclass[a4paper, landscape,twocolumn,fontsize=9pt]{scrartcl}

\usepackage{fontspec}
\setmainfont[Ligatures=TeX]{Georgia}
\setsansfont[BoldFont="HelveticaNeue-Medium"]{Helvetica Neue}

\usepackage{amsmath, amsthm, amssymb}
\usepackage{mathtools}
\usepackage[most]{tcolorbox}
\usepackage{blindtext}
\usepackage{xcolor}
\usepackage{titlesec}
\usepackage{titling}


\newfontfamily\Menlo[Ligatures=TeX]{Menlo}

\definecolor{grey}{rgb}{0.5,0.5,0.5}
\definecolor{lightgrey}{rgb}{0.9,0.9,0.9}
\definecolor{darkgrey}{rgb}{0.3,0.3,0.3}
\definecolor{orange}{rgb}{0.94, 0.55, 0.294}
\definecolor{pink}{rgb}{0.94, 0.29, 0.7}
\definecolor{yellow}{rgb}{1, 0.749, 0}

\newcommand{\chapfnt}{\fontsize{16}{19}}
\newcommand{\secfnt}{\fontsize{18}{17}}
\newcommand{\ssecfnt}{\fontsize{14}{14}}
\renewcommand{\hline}{\noindent\makebox[\linewidth]{\rule{12cm}{1pt}}}
\newcommand{\code}[1]{{\Menlo{\color{darkgrey}#1}}}
\newcommand{\vip}[1]{\textit{\textbf{#1}}}

\titleformat{\chapter}[display]
{\normalfont\chapfnt\bfseries}{\chaptertitlename\ \thechapter}{20pt}{\chapfnt}

\titleformat{\section}
{\normalfont\sffamily\secfnt\mdseries}{\thesection}{1em}{}

\titleformat{\subsection}
{\normalfont\sffamily\ssecfnt\mdseries\color{grey}}{\thesubsection}{1em}{}

\titlespacing*{\chapter} {0pt}{50pt}{40pt}
\titlespacing*{\section} {0pt}{0pt}{16pt}
\titlespacing*{\subsection} {0pt}{12pt}{8pt}

    
\usepackage{geometry}
\setlength{\columnsep}{32mm}
\geometry{
 left=22mm,
 right=22mm,
 bottom=32mm,
 top = 20mm
}


\newtcbtheorem[auto counter,number within=section]{theorem}%
  {Theorem}{
  		fonttitle=\upshape, 
  		fontupper=\upshape,
  		boxrule=0pt,
  		leftrule=3pt,
  		arc=0pt,auto outer arc,
  		colback=white,
  		colframe=pink,
  		colbacktitle=white,
  		coltitle=pink,
  		oversize,
  		enlarge top by=1mm,
  		enlarge bottom by=1mm,
    	enhanced jigsaw,
    	interior hidden, 
    	before skip=12pt,
    	overlay={
    		\draw[line width=1.5pt,pink] (frame.north west) -- (frame.south west);
  		}, 
  		frame hidden}{theorem}

\newtcbtheorem[auto counter,number within=section]{lemma}%
  {Lemma}{
  		fonttitle=\upshape, 
  		fontupper=\upshape,
  		boxrule=1pt,
  		toprule=0pt,
  		leftrule=3pt,
  		arc=0pt,auto outer arc,
  		colback=white,
  		colframe=orange,
  		colbacktitle=white,
  		coltitle=orange,
  		oversize,
  		enlarge top by=1mm,
  		enlarge bottom by=1mm,
    	enhanced jigsaw,
    	interior hidden, 
    	before skip=12pt,
    	overlay={
    		\draw[line width=1.5pt,orange] (frame.north west) -- (frame.south west);
  		}, 
  		frame hidden}{lemma}
  		
 \newtcbtheorem[auto counter,number within=section]{definition}%
  {Definition}{
  		fonttitle=\upshape, 
  		fontupper=\upshape,
  		boxrule=1pt,
  		toprule=0pt,
  		leftrule=3pt,
  		arc=0pt,auto outer arc,
  		colback=white,
  		colframe=orange,
  		colbacktitle=white,
  		coltitle=orange,
  		oversize,
  		enlarge top by=1mm,
  		enlarge bottom by=1mm,
    	enhanced jigsaw,
    	interior hidden, 
    	before skip=12pt,
    	overlay={
    		\draw[line width=1.5pt,orange] (frame.north west) -- (frame.south west);
  		}, 
  		frame hidden}{definition}
    	
\newtcbtheorem[auto counter,number within=section]{example}%
  {Beispiel}{
  		fonttitle=\upshape, 
  		fontupper=\upshape,
  		boxrule=0pt,
  		leftrule=3pt,
  		arc=0pt,auto outer arc,
  		colback=white,
  		colframe=grey,
  		colbacktitle=white,
  		coltitle=grey,
  		oversize,
  		enlarge top by=1mm,
  		enlarge bottom by=1mm,
    	enhanced jigsaw,
    	interior hidden, 
    	before skip=12pt,
    	overlay={
    		\draw[line width=1.5pt,grey] (frame.north west) -- (frame.south west);
  		}, 
  		frame hidden}{example}
    	
\newtcbtheorem[auto counter,number within=section]{note}%
  {Notiz}{
  		fonttitle=\upshape, 
  		fontupper=\upshape,
  		boxrule=0pt,
  		leftrule=3pt,
  		arc=0pt,auto outer arc,
  		colback=white,
  		colframe=yellow,
  		colbacktitle=white,
  		coltitle=yellow,
  		oversize,
  		enlarge top by=1mm,
  		enlarge bottom by=1mm,
    	enhanced jigsaw,
    	interior hidden, 
    	before skip=12pt,
    	overlay={
    		\draw[line width=1.5pt,yellow] (frame.north west) -- (frame.south west);
  		}, 
  		frame hidden}{note}
  		
\newtcbtheorem[]{important}%
  {Wichtig}{
  		fonttitle=\upshape, 
  		fontupper=\upshape,
  		boxrule=0pt,
  		leftrule=3pt,
  		arc=0pt,auto outer arc,
  		colback=white,
  		colframe=pink,
  		colbacktitle=white,
  		coltitle=pink,
  		oversize,
  		enlarge top by=1mm,
  		enlarge bottom by=1mm,
    	enhanced jigsaw,
    	interior hidden, 
    	before skip=12pt,
    	overlay={
    		\draw[line width=1.5pt,pink] (frame.north west) -- (frame.south west);
  		}, 
  		frame hidden}{important}
    	
\renewcommand{\baselinestretch}{1.4} 
\makeatletter
\let\old@rule\@rule
\def\@rule[#1]#2#3{\textcolor{lightgrey}{\old@rule[#1]{#2}{#3}}}
\makeatother

\begin{document}

\section{Mathematische Modellierung von Zufallsexperimenten}

\subsection{Axiomatischer Aufbau der Wahrscheinlichkeitstheorie}
Ausgangspunkt der Wahrscheinlichkeitstheorie ist die mathematische Beschreibung von \vip{Zufallsexperimenten}, das heißt zeitlich wie örtliche fest umrissene Vorgänge mit unbestimmten Ausgang.

\begin{example}{}{}
\begin{itemize}
	\item Werfen eines Würfels oder einer Münze
	\item Zufälliges Ziehen einer Kugel aus einer Urne
	\item Kartenspiele
	\item Wahlergebnis der nächsten Europawahl
	\item Temperatur am Alexanderplatz am 11. April 2019 um 12:00 Uhr
	\item Lebensdauer von technischen Geräten
\end{itemize}	
\end{example}

Die Gesamtheit aller möglichen Ausgänge/ Ergebnisse eines Zufallsexperimentes heißt \vip{Ergebnisraum} oder \vip{Stichprobenraum} (letzteres wird eher in der Statistik verwendet). Dies wird mit $\Omega$ bezeichnet. Ein Element $\omega \in \Omega$ heißt \vip{Ergebnis} oder \vip{Stichprobe}. 

\begin{example}{}{}
	\begin{itemize}
		\item einmaligen Würfeln: 
		\[
			\Omega = \{ 1,..., 6 \}, \quad |\Omega| = 6
		\]
		
		\item zweimaligen Würfeln:
		\[
			\Omega = \{ (i,j) : i,j \in \{ 1,...,6 \} \} = \{1,...,6\} \times \{1,...,6 \} = \{ 1,...,6 \}^2, \quad |\Omega| = 36
		\]
		
		\item Münzwurf: $\Omega = \{ \text{Kopf}, \text{ Zahl}\}$ oder $\Omega = \{ 0,1 \}$.
		\item Anzahl der Autos am Funkturm am 11. April 2019: $\Omega = \mathbb N \cup \{ 0 \}$
		
		\item Temperatur in Grad Kelvin am Alexanderplatz, d.h. $\Omega = [0,\infty)$ oder realistischer $\Omega = [270, 310]$

	\end{itemize}	
\end{example}

In den ersten drei Beispielen ist der Ergebnisraum \emph{endlich}, im vorletzten Beispiel \emph{abzählbar unendlich} und im letzten Beispiel \emph{kontinuierlich} (d.h. überabzählbar unendlich). (i)-(iv) nennt man \textbf{diskrete} Ergebnisräume.

\vip{Ereignisse} sind Teilmengen $A \subset \Omega$. Die Gesamtheit aller Ereignisse wird mit $\mathcal P(\Omega)$ bezeichnet (Potenzmenge). Besonders hervorzuheben sind
\begin{itemize}
	\item das \emph{sichere Ereignis}: $\Omega$
	\item das \emph{unmögliche Ereignis}: $\emptyset$ 
	\item die \emph{Elementarereignisse}: $\{ \omega \}$ für $\omega \in \Omega$
\end{itemize}

\begin{example}{}{}
\begin{enumerate}
	\item $A = \{ 1,3,5 \} = \text{ "Augenzahl ungerade"}$
	\item $A = \{ (5,6), (6,5), (6,6) \} = \text{ "Augensumme $>$ 10"}$
	\item $A = \{ n : n \geq 40000\} =\text{ "ungewöhnlich hohes Verkehrsaufkommen"} $
\end{enumerate}	
\end{example}

\textbf{Operationen auf Ereignissen}
\begin{enumerate}
	\item $A \cup B = A \text{ oder } B \text{ tritt ein}$
	\item $A \cup ... \cup A_n = \bigcup_{k=1}^n A_k =$ mindestens eins der $A_k$ tritt ein
	\item $A \cap B$ = $A$ und $B$ treten ein
	\item $A \cap ... \cap A_n = \bigcap_{k=1}^n A_k = $ alle $A_k$ treten ein
	\item $A^c \coloneqq \Omega \setminus A = \{ \omega : \omega \notin A \} =$ $A$ tritt \emph{nicht} ein ($\emptyset^c = \Omega, \Omega^c = \emptyset$)
\end{enumerate}

\begin{important}{}{}
Insbesondere im Falle kontinuierlicher Mengen ist es im allgemeinen unmöglich \emph{jedem} Ereignis in konsistenter Weise Wahrscheinlichkeiten zuzuordnen (siehe Satz 1.15). Daher schränkt man sich auf \emph{kleinere} Mengensysteme ein.
\end{important}

Dies führt auf den Begriff der $\mathbf{\sigma}$\vip{-Algebra}.

\begin{definition}{}{}
	Es sei $\Omega \neq \emptyset$. Ein Mengensystem $\mathcal A \subset \mathcal P(\Omega)$ heißt $\sigma$-Algebra, falls gilt
	\begin{itemize}
		\item $\Omega \in \mathcal A$
		\item $A \in \mathcal A \implies A^C = \Omega \setminus A \in \mathcal A$
		\item $(A_n)_{n \in \mathbb N} \in \mathcal A \implies \bigcup_{n \geq 1} A_n \in \mathcal A$
	\end{itemize}
	Das Paar $(\Omega, \mathcal A)$ heißt \vip{messbarer Raum}. Eine Teilmenge $A \in \Omega$ heißt \vip{$\mathcal A$-messbar}, falls $A \in \mathcal A$
\end{definition}

\begin{lemma}{}{}
Es sei $\mathcal A$ eine $\sigma$-Algebra. Dann gilt
\begin{enumerate}
	\item $\Omega \notin \mathcal A$
	\item $A,B \in \mathcal A \implies A \cup B, A \cap B, A \setminus B \in \mathcal A$.
	\item $(A_n)_{n \in \mathbb N} \in \mathcal A \implies \bigcap_{n \geq 1} A_n \in \mathcal A$
\end{enumerate}	
\end{lemma}

\begin{proof}
	Siehe Skript.
\end{proof}

\begin{example}{}{}
\begin{enumerate}
	\item $\mathcal P(\Omega)$ ist eine $\sigma$-Algebra und zwar die \emph{größtmögliche}.
	\item $\{ \emptyset, \Omega \}$ ist $\sigma$-Algebra und zwar die kleinstmögliche, \emph{triviale} $\sigma$-Algebra.
	\item Ist $A \subset \Omega$ ein Ereignis, so ist $\{ \emptyset, \Omega, A, A^c \}$ eine $\sigma$-Algebra und zwar die kleinste, die $A$ enthält.
	\item \textbf{Hüllenoperator:} Ist $\mathcal C \subset \mathcal P(\Omega)$ ein beliebiges Mengensystem, so ist
	\[
		\sigma(\mathcal C) = \bigcap_{\substack{\mathcal A \text{ $\sigma$-Algebra auf $\Omega$} \\ \text{mit $\mathcal C \subset \mathcal A$}}} \mathcal A
	\]
	eine $\sigma$-Algebra (Beweis!). Sie ist die kleinste die $\mathcal C$ enthält. Sie heißt die \vip{von $\mathcal C$ erzeugte $\sigma$-Algebra}. $\sigma(\mathcal C)$ ist wohldefiniert, da der Schnitt nicht leer ist (siehe $\mathcal P(\Omega)$).
\end{enumerate}	
\end{example}

Es gelten für den Hüllenoperator:
\begin{enumerate}
	\item $\mathcal C \subset \sigma(\mathcal C)$
	\item $\mathcal C_1 \subset \mathcal C_2 \implies \sigma(\mathcal C_1) \subset \sigma(\mathcal C_2)$
	\item $\mathcal C$ ist eine $\sigma$-Algebra genau dann, wenn $\sigma(\mathcal C) = \mathcal C$.
\end{enumerate}

\textbf{Wahrscheinlichkeiten}

\textit{Nicht gottgegeben!!!} Im nächsten Schritt wollen wir für jedes messbare Ereignis $A$ eine Wahrscheinlichkeit $P(A)$ angeben zwischen $0$ und $1$. $P(A)$ soll ein Ma0 dafür sein, dass $A$ auftritt:
\begin{itemize}
	\item tritt $A$ niemals ein, so setzen wir $P(A) = 0$,
	\item tritt $A$ sicher sein, so setzen wir $P(A) = 1$.
\end{itemize}
Zusätzlich sollte gelten: Sind $A$ und $B$ \emph{disjunkte} Ereignisse, d.h. $A \cap B = \emptyset$, so ist \begin{equation*}P (A \cup B) = P(A) + P(B) \qquad \tag{{Additivität}}.\end{equation*}
Unmittelbare Folgerung $A_1,...,A_n$ paarweise disjunkte Ereignisse, so gilt
\[
	P(A_1 \cup ... \cup A_n) = P(A_1) + ... + P_(A_n).
\]



\end{document}
