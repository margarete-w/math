\documentclass[a4paper, landscape,twocolumn,fontsize=9pt]{scrartcl}

\usepackage{fontspec}
\setmainfont[Ligatures=TeX]{Georgia}
\setsansfont[BoldFont="HelveticaNeue-Medium"]{Helvetica Neue}

\usepackage{amsmath, amsthm, amssymb}
\usepackage{mathtools}
\usepackage[most]{tcolorbox}
\usepackage{blindtext}
\usepackage{xcolor}
\usepackage{titlesec}
\usepackage{titling}


\newfontfamily\Menlo[Ligatures=TeX]{Menlo}

\definecolor{grey}{rgb}{0.5,0.5,0.5}
\definecolor{lightgrey}{rgb}{0.9,0.9,0.9}
\definecolor{darkgrey}{rgb}{0.3,0.3,0.3}
\definecolor{orange}{rgb}{0.94, 0.55, 0.294}
\definecolor{pink}{rgb}{0.94, 0.29, 0.7}
\definecolor{yellow}{rgb}{1, 0.749, 0}

\newcommand{\chapfnt}{\fontsize{16}{19}}
\newcommand{\secfnt}{\fontsize{18}{17}}
\newcommand{\ssecfnt}{\fontsize{14}{14}}
\renewcommand{\hline}{\noindent\makebox[\linewidth]{\rule{12cm}{1pt}}}
\newcommand{\code}[1]{{\Menlo{\color{darkgrey}#1}}}
\newcommand{\vip}[1]{\textit{\textbf{#1}}}

\titleformat{\chapter}[display]
{\normalfont\chapfnt\bfseries}{\chaptertitlename\ \thechapter}{20pt}{\chapfnt}

\titleformat{\section}
{\normalfont\sffamily\secfnt\mdseries}{\thesection}{1em}{}

\titleformat{\subsection}
{\normalfont\sffamily\ssecfnt\mdseries\color{grey}}{\thesubsection}{1em}{}

\titlespacing*{\chapter} {0pt}{50pt}{40pt}
\titlespacing*{\section} {0pt}{0pt}{16pt}
\titlespacing*{\subsection} {0pt}{12pt}{8pt}

    
\usepackage{geometry}
\setlength{\columnsep}{32mm}
\geometry{
 left=22mm,
 right=22mm,
 bottom=32mm,
 top = 20mm
}


\newtcbtheorem[auto counter,number within=section]{theorem}%
  {Theorem}{
  		fonttitle=\upshape, 
  		fontupper=\upshape,
  		boxrule=0pt,
  		leftrule=3pt,
  		arc=0pt,auto outer arc,
  		colback=white,
  		colframe=pink,
  		colbacktitle=white,
  		coltitle=pink,
  		oversize,
  		enlarge top by=1mm,
  		enlarge bottom by=1mm,
    	enhanced jigsaw,
    	interior hidden, 
    	before skip=12pt,
    	overlay={
    		\draw[line width=1.5pt,pink] (frame.north west) -- (frame.south west);
  		}, 
  		frame hidden}{theorem}

\newtcbtheorem[auto counter,number within=section]{lemma}%
  {Lemma}{
  		fonttitle=\upshape, 
  		fontupper=\upshape,
  		boxrule=1pt,
  		toprule=0pt,
  		leftrule=3pt,
  		arc=0pt,auto outer arc,
  		colback=white,
  		colframe=yellow,
  		colbacktitle=white,
  		coltitle=yellow,
  		oversize,
  		enlarge top by=1mm,
  		enlarge bottom by=1mm,
    	enhanced jigsaw,
    	interior hidden, 
    	before skip=12pt,
    	overlay={
    		\draw[line width=1.5pt,yellow] (frame.north west) -- (frame.south west);
  		}, 
  		frame hidden}{lemma}
  		
 \newtcbtheorem[auto counter,number within=section]{definition}%
  {Definition}{
  		fonttitle=\upshape, 
  		fontupper=\upshape,
  		boxrule=1pt,
  		toprule=0pt,
  		leftrule=3pt,
  		arc=0pt,auto outer arc,
  		colback=white,
  		colframe=orange,
  		colbacktitle=white,
  		coltitle=orange,
  		oversize,
  		enlarge top by=1mm,
  		enlarge bottom by=1mm,
    	enhanced jigsaw,
    	interior hidden, 
    	before skip=12pt,
    	overlay={
    		\draw[line width=1.5pt,orange] (frame.north west) -- (frame.south west);
  		}, 
  		frame hidden}{definition}
    	
\newtcbtheorem[auto counter,number within=section]{example}%
  {Beispiel}{
  		fonttitle=\upshape, 
  		fontupper=\upshape,
  		boxrule=0pt,
  		leftrule=3pt,
  		arc=0pt,auto outer arc,
  		colback=white,
  		colframe=grey,
  		colbacktitle=white,
  		coltitle=grey,
  		oversize,
  		enlarge top by=1mm,
  		enlarge bottom by=1mm,
    	enhanced jigsaw,
    	interior hidden, 
    	before skip=12pt,
    	overlay={
    		\draw[line width=1.5pt,grey] (frame.north west) -- (frame.south west);
  		}, 
  		frame hidden}{example}
    	
\newtcbtheorem[auto counter,number within=section]{note}%
  {Notiz}{
  		fonttitle=\upshape, 
  		fontupper=\upshape,
  		boxrule=0pt,
  		leftrule=3pt,
  		arc=0pt,auto outer arc,
  		colback=white,
  		colframe=yellow,
  		colbacktitle=white,
  		coltitle=yellow,
  		oversize,
  		enlarge top by=1mm,
  		enlarge bottom by=1mm,
    	enhanced jigsaw,
    	interior hidden, 
    	before skip=12pt,
    	overlay={
    		\draw[line width=1.5pt,yellow] (frame.north west) -- (frame.south west);
  		}, 
  		frame hidden}{note}
  		
\newtcbtheorem[]{important}%
  {Wichtig}{
  		fonttitle=\upshape, 
  		fontupper=\upshape,
  		boxrule=0pt,
  		leftrule=3pt,
  		arc=0pt,auto outer arc,
  		colback=white,
  		colframe=pink,
  		colbacktitle=white,
  		coltitle=pink,
  		oversize,
  		enlarge top by=1mm,
  		enlarge bottom by=1mm,
    	enhanced jigsaw,
    	interior hidden, 
    	before skip=12pt,
    	overlay={
    		\draw[line width=1.5pt,pink] (frame.north west) -- (frame.south west);
  		}, 
  		frame hidden}{important}
    	
\renewcommand{\baselinestretch}{1.4} 
\makeatletter
\let\old@rule\@rule
\def\@rule[#1]#2#3{\textcolor{lightgrey}{\old@rule[#1]{#2}{#3}}}
\makeatother

\begin{document}

\section{Mathematische Modellierung von Zufallsexperimenten}

\subsection{Axiomatischer Aufbau der Wahrscheinlichkeitstheorie}
Ausgangspunkt der Wahrscheinlichkeitstheorie ist die mathematische Beschreibung von \vip{Zufallsexperimenten}, das heißt zeitlich wie örtliche fest umrissene Vorgänge mit unbestimmten Ausgang.

\begin{example}{}{}
\begin{itemize}
	\item Werfen eines Würfels oder einer Münze
	\item Zufälliges Ziehen einer Kugel aus einer Urne
	\item Kartenspiele
	\item Wahlergebnis der nächsten Europawahl
	\item Temperatur am Alexanderplatz am 11. April 2019 um 12:00 Uhr
	\item Lebensdauer von technischen Geräten
\end{itemize}	
\end{example}

Die Gesamtheit aller möglichen Ausgänge/ Ergebnisse eines Zufallsexperimentes heißt \vip{Ergebnisraum} oder \vip{Stichprobenraum} (letzteres wird eher in der Statistik verwendet). Dies wird mit $\Omega$ bezeichnet. Ein Element $\omega \in \Omega$ heißt \vip{Ergebnis} oder \vip{Stichprobe}. 

\begin{example}{}{}
	\begin{itemize}
		\item einmaligen Würfeln: 
		\[
			\Omega = \{ 1,..., 6 \}, \quad |\Omega| = 6
		\]
		
		\item zweimaligen Würfeln:
		\[
			\Omega = \{ (i,j) : i,j \in \{ 1,...,6 \} \} = \{1,...,6\} \times \{1,...,6 \} = \{ 1,...,6 \}^2, \quad |\Omega| = 36
		\]
		
		\item Münzwurf: $\Omega = \{ \text{Kopf}, \text{ Zahl}\}$ oder $\Omega = \{ 0,1 \}$.
		\item Anzahl der Autos am Funkturm am 11. April 2019: $\Omega = \mathbb N \cup \{ 0 \}$
		
		\item Temperatur in Grad Kelvin am Alexanderplatz, d.h. $\Omega = [0,\infty)$ oder realistischer $\Omega = [270, 310]$

	\end{itemize}	
\end{example}

In den ersten drei Beispielen ist der Ergebnisraum \emph{endlich}, im vorletzten Beispiel \emph{abzählbar unendlich} und im letzten Beispiel \emph{kontinuierlich} (d.h. überabzählbar unendlich). (i)-(iv) nennt man \textbf{diskrete} Ergebnisräume.

\vip{Ereignisse} sind Teilmengen $A \subset \Omega$. Die Gesamtheit aller Ereignisse wird mit $\mathcal P(\Omega)$ bezeichnet (Potenzmenge). Besonders hervorzuheben sind
\begin{itemize}
	\item das \emph{sichere Ereignis}: $\Omega$
	\item das \emph{unmögliche Ereignis}: $\emptyset$ 
	\item die \emph{Elementarereignisse}: $\{ \omega \}$ für $\omega \in \Omega$
\end{itemize}

\begin{example}{}{}
\begin{enumerate}
	\item $A = \{ 1,3,5 \} = \text{ "Augenzahl ungerade"}$
	\item $A = \{ (5,6), (6,5), (6,6) \} = \text{ "Augensumme $>$ 10"}$
	\item $A = \{ n : n \geq 40000\} =\text{ "ungewöhnlich hohes Verkehrsaufkommen"} $
\end{enumerate}	
\end{example}

\textbf{Operationen auf Ereignissen}
\begin{enumerate}
	\item $A \cup B = A \text{ oder } B \text{ tritt ein}$
	\item $A \cup ... \cup A_n = \bigcup_{k=1}^n A_k =$ mindestens eins der $A_k$ tritt ein
	\item $A \cap B$ = $A$ und $B$ treten ein
	\item $A \cap ... \cap A_n = \bigcap_{k=1}^n A_k = $ alle $A_k$ treten ein
	\item $A^c \coloneqq \Omega \setminus A = \{ \omega : \omega \notin A \} =$ $A$ tritt \emph{nicht} ein ($\emptyset^c = \Omega, \Omega^c = \emptyset$)
\end{enumerate}

\begin{important}{}{}
Insbesondere im Falle kontinuierlicher Mengen ist es im allgemeinen unmöglich \emph{jedem} Ereignis in konsistenter Weise Wahrscheinlichkeiten zuzuordnen (siehe Satz 1.15). Daher schränkt man sich auf \emph{kleinere} Mengensysteme ein.
\end{important}

Dies führt auf den Begriff der $\mathbf{\sigma}$\vip{-Algebra}.

\begin{definition}{}{}
	Es sei $\Omega \neq \emptyset$. Ein Mengensystem $\mathcal A \subset \mathcal P(\Omega)$ heißt $\sigma$-Algebra, falls gilt
	\begin{itemize}
		\item $\Omega \in \mathcal A$
		\item $A \in \mathcal A \implies A^C = \Omega \setminus A \in \mathcal A$
		\item $(A_n)_{n \in \mathbb N} \in \mathcal A \implies \bigcup_{n \geq 1} A_n \in \mathcal A$
	\end{itemize}
	Das Paar $(\Omega, \mathcal A)$ heißt \vip{messbarer Raum}. Eine Teilmenge $A \in \Omega$ heißt \vip{$\mathcal A$-messbar}, falls $A \in \mathcal A$
\end{definition}

\begin{lemma}{}{}
Es sei $\mathcal A$ eine $\sigma$-Algebra. Dann gilt
\begin{enumerate}
	\item $\Omega \notin \mathcal A$
	\item $A,B \in \mathcal A \implies A \cup B, A \cap B, A \setminus B \in \mathcal A$.
	\item $(A_n)_{n \in \mathbb N} \in \mathcal A \implies \bigcap_{n \geq 1} A_n \in \mathcal A$
\end{enumerate}	
\end{lemma}

\begin{proof}
	Siehe Skript.
\end{proof}

\begin{example}{}{}
\begin{enumerate}
	\item $\mathcal P(\Omega)$ ist eine $\sigma$-Algebra und zwar die \emph{größtmögliche}.
	\item $\{ \emptyset, \Omega \}$ ist $\sigma$-Algebra und zwar die kleinstmögliche, \emph{triviale} $\sigma$-Algebra.
	\item Ist $A \subset \Omega$ ein Ereignis, so ist $\{ \emptyset, \Omega, A, A^c \}$ eine $\sigma$-Algebra und zwar die kleinste, die $A$ enthält.
	\item \textbf{Hüllenoperator:} Ist $\mathcal C \subset \mathcal P(\Omega)$ ein beliebiges Mengensystem, so ist
	\[
		\sigma(\mathcal C) = \bigcap_{\substack{\mathcal A \text{ $\sigma$-Algebra auf $\Omega$} \\ \text{mit $\mathcal C \subset \mathcal A$}}} \mathcal A
	\]
	eine $\sigma$-Algebra (Beweis!). Sie ist die kleinste die $\mathcal C$ enthält. Sie heißt die \vip{von $\mathcal C$ erzeugte $\sigma$-Algebra}. $\sigma(\mathcal C)$ ist wohldefiniert, da der Schnitt nicht leer ist (siehe $\mathcal P(\Omega)$).
\end{enumerate}	
\end{example}

Es gelten für den Hüllenoperator:
\begin{enumerate}
	\item $\mathcal C \subset \sigma(\mathcal C)$
	\item $\mathcal C_1 \subset \mathcal C_2 \implies \sigma(\mathcal C_1) \subset \sigma(\mathcal C_2)$
	\item $\mathcal C$ ist eine $\sigma$-Algebra genau dann, wenn $\sigma(\mathcal C) = \mathcal C$.
\end{enumerate}

\textbf{Wahrscheinlichkeiten}

\textit{Nicht gottgegeben!!!} Im nächsten Schritt wollen wir für jedes messbare Ereignis $A$ eine Wahrscheinlichkeit $P(A)$ angeben zwischen $0$ und $1$. $P(A)$ soll ein Maß dafür sein, dass $A$ auftritt:
\begin{itemize}
	\item tritt $A$ niemals ein, so setzen wir $P(A) = 0$,
	\item tritt $A$ sicher sein, so setzen wir $P(A) = 1$.
\end{itemize}
Zusätzlich sollte gelten: Sind $A$ und $B$ \emph{disjunkte} Ereignisse, d.h. $A \cap B = \emptyset$, so ist \begin{equation*}P (A \cup B) = P(A) + P(B) \qquad \tag{{Additivität}}.\end{equation*}
Unmittelbare Folgerung $A_1,...,A_n$ paarweise disjunkte Ereignisse, so gilt
\[
	P(A_1 \cup ... \cup A_n) = P(A_1) + ... + P_(A_n).
\]

\begin{definition}{Wahrscheinlichkeitsmaß}{}
Sei $(\Omega, A)$ ein messbarer Raum. EIne Abbildung $P: \mathcal A \to [0,1]$ heißt \vip{Wahrscheinlichkeitsmaß} auf $\mathcal A$, falls gilt
\begin{enumerate}
	\item \textbf{Normiertheit}: $P(\Omega) = 1$
	\item \textbf{$\sigma$-Additivität}: Für jede Folge $(A_n)_{n \in \mathbb N}$ paarweiser disjunkter Ereignisse in $\mathcal A$ gilt
	\[
		P(\bigcup_{n \geq 1}A_n) = \sum_{n \geq 1} P(A_n).
	\]
\end{enumerate}
\end{definition}

Die Begriffe Wahrscheinlichkeitsmaß und $\sigma$-Algebra sind absolut essentiell.

\begin{lemma}{Rechenregeln für $P$}{}
Sei $P$ ein Wahrscheinlichkeitsmaß auf $\Omega, \mathcal A$ und $A,B, A_1,A_2,... \in \mathcal A$. Dann gilt
\begin{enumerate}
\item $P(\emptyset) = 0$
\item $P$ ist insbesondere \emph{endlich additiv}, d.h. für paarweise disjunkte $A_1,...,A_n$ gilt
\[
	P(A_1 \cup ... \cup A_n) = P(A_1) + ... + P(A_n).
\]

\item \textbf{Komplementärwahrscheinlichkeit} $P(A^c) = 1 - P(A)$

\item \textbf{Monotonie} (die wichtigste Eigenschaft): $A \subset B \implies P(A) \leq P(B)$

\item \textbf{Subadditivität}: $A \subset \bigcup_{n} A_n \implies P(A) \leq \sum_n P(A_n)$.
\end{enumerate}
\end{lemma}

\begin{proof}
\begin{enumerate}
	\item Betrachte die triviale Folge $A_n \coloneqq \emptyset$ für $n= 1,2,...$. Offensichtlich sind die $A_n, n \geq 1$ paarweise disjunkt. Wegen $\bigcup_n A_n = \bigcup \emptyset = \emptyset$ und aus der $\sigma$-Additivität folgt jetzt
\[
	P(\emptyset) = P(\bigcup_n A_n) = \sum_{n} P(\emptyset).
\]
Wäre $P(\emptyset) > 0$, so würde $\sum P(\emptyset)$ divergieren. Also $P(\emptyset) \in [0,1] \implies P(\emptyset) = 0$.

\item Setze $A_k \coloneqq \emptyset$ für $k \geq n + 1$ und wir erhalten somit eine Folge paarweiser disjunkter Ereignisse $(A_k)_{k \geq 1}$ mit $\bigcup_{k \geq 1}A_k = A_1 \cup ... \cup A_n \implies P(A_1 \cup ... \cup A_n) = P(\bigcup_{k \geq 1}A_k) = \sum_{k \geq 1}P(A_k) = P(A_1) + ... + P(A_n)$,  denn $P(A_k) = P(\emptyset) = 0$ für $k \geq n+1$.

\item Sei $A \in \mathcal A$. Dann sind $A,A^c$ paarweise disjunkt und $A \cup A^c = \Omega$. Damit $1 = P(\Omega) = P(A \dot \cup A^c) = P(A) + P(A^c) \implies P(A^c) = 1 - P(A)$. Der Vereinigungsoperator $\dot \cup$ bedeutet, dass man zwei disjunkte Mengen vereinigt.

\item $A \subset B$, d.h. $B = A \dot \cup B \setminus A \implies P(B) = P(A) + \underbrace{P(B \setminus A)}_{\geq 0} \geq P(A)$.


\item Definiere induktiv $B_1 \coloneqq A_1, B_2 = A_2 \setminus A_1 = A_2 \setminus B_1$. Also definiere einfach für alle $n \geq 2$: $B_n \coloneqq A_n \setminus (A_1 \cup ... \cup A_{n-1})$. Dann sind die $B_n$ paarweise disjunkte Ereignisse und $B_1 \cup ... \cup B_n = A_1 \cup ... \cup A_n$ für alle $n \in \mathbb N$. Damit
\[
	P(\bigcup_n A_n) = P(\bigcup_n B_n) = \sum_n P(B_n) \leq \sum_n P(A_n)
\]
und ist $A \subset \bigcup_n A_n \implies P(A) \leq P(\bigcup_n A_n) \leq \sum_n P(A_n)$.
\end{enumerate}
\end{proof}

\begin{definition}{Wahrscheinlichkeitsraum}{}
Es sei $\Omega \neq \emptyset$, $\mathcal A$ eine $\sigma$-Algebra in $\Omega$ und $P$ ein Wahrscheinlichkeitsmaß auf $\mathcal A$. Dann heißt das Tripel $(\Omega, \mathcal A, P)$ ein \vip{Wahrscheinlichkeitsraum}.
\end{definition}


\begin{theorem}{Einschluss-Ausschluss-Prinzip, Formel von Sylvester}{}
Es sei $(\Omega, \mathcal A, P$ ein Wahrscheinlichkeitsraum. Dann gilt für beliebige $A_1,...A_n \in \mathcal A$:
\begin{align*}
	P(A_1 \cup ... \cup A_n) &= \sum_{\emptyset \neq I \subset \{ 1,...,n\}}(-1)^{|I| - 1} \cdot P(\bigcap_{i \in I} A_i) \\
	&= \sum^n_{j=1}(-1)^{j-1} \sum_{1 \leq i_1 < i_2 < ... < i_j \leq n} P(A_{i_1} \cap A_{i_2} \cap ... \cap A_{i_j}),
\end{align*}
\end{theorem}

\textbf{Spezialfälle}
\begin{align*}
	P(A \cup B) &= P(A) + P(B) - P(A \cap B) \\
	P(A \cup B \cup C) &= P(A) + P(B) + P(C) - P(A \cap B) - P(A \cap C) - P(B \cap C) + P(A \cap B \cap C).
\end{align*}

\begin{proof}
Induktion nach $n$: Induktionsanfang $n = 1$. Für $n = 2$: Es gilt $A_1 \cup A_2 = A_1 \dot \cup (A_2 \setminus A_1) \implies P(A_1 \cup A_2) = P(A_1) + P(A_2 \setminus A_1)$ und $A_2 = (A_2 \setminus A_1) \dot \cup (A_2 \cap A_1) \implies P(A_2) = P(A_2 \setminus A_1) + P(A_2 \setminus A_1)$. Damit folgt
\begin{align}\label{1:lukas}
	P(A_1 \cup A_2) = P(A_1) + P(A_2 \setminus A_1) = P(A_1) + P(A_2) - P(A_1 \cap A_2).
\end{align}

Induktionsschluss: $n \leadsto n+1$ Wende \eqref{1:lukas} auf $A_1 \cup ... \cup A_n$ und $A_{n+1}$ an. Wir erhalten
\begin{align}\label{1: lol}
	P(A_1 \cup ... \cup A_{n+1}) &= P(A_1 \cup ... \cup A_n) + P(A_{n+1}) + P(A_{n+1}) + P((A_1 \cup ... \cup A_n) \cap A_{n+1}) \\
	&= P(A_1 \cup ... \cup A_n) + P(A_{n+1}) + P(A_{n+1}) + P((A_1 \cap A_{n+1}) \cup ... \cup (A_n \cap A_{n+1}))
\end{align}
Nach Induktionsannahme gilt
\[
	P(A_1 \cup ... \cup A_n) = \sum_{\emptyset \neq I \subset \{ 1,..., n \} } (-1)^{|I| - 1} P(\bigcup_{i \in I} A_i)
\]

und $P((A_1 \cap A_{n+1}) \cup ... \cup (A_n \cap A_{n+1})) = \sum_{\emptyset \neq I \subset \{ 1,..., n \}} (-1)^{|I| - 1} P(\bigcap_{i \in I} (A_i \cap A_{n+1})) = \sum_{\emptyset \neq I \subset \{ 1,..., n \}} (-1)^{|I| - 1} P(\bigcap_{i \in I \cup \{ n+1\}}A_1)$.

Einsetzen in \eqref{1:lol} ergibt:
\[
	P(A_1 \cup ... \cup A_{n+1}) = \sum_{\emptyset \neq I \subset \{ 1,..., n\}} (-1)^{|I| - 1} P(\bigcap_{i \in I} A_i) + P(A_{n+1}) - \sum_{\emptyset \neq I \subset \{ 1,..., n \}} (-1)^{|I| - 1} P(\bigcap_{i \in I \cup \{ n+1 \}} A_i) = ...
\]
und wir sind fertig.
\end{proof}

\begin{example}{}{}
Wie groß ist die Wahrscheinlichkeit bei 10-maligen Würfeln nicht alle 6 Augenzahlen zu würfeln?
\end{example}

$E_i \coloneqq i$ kommt in den 10 Würfen nicht vor, $i = 1,..., 6$

Gesuchte Wahrscheinlichkeit: $P(E_1 \cup ... \cup E_6)$

Es gilt: $P(E_1 \cap ... \cap E_i) = \frac{(6-i)^{10}}{6^{10}}$. Allgemeiner $P(\bigcap_{i \in I} E_i) = \frac{(6- |I|)^{10}}{6^{10}}$.

Das heißt
\begin{align*}
	P(E_1 \cup ... \cup E_6) = \sum^6_{j=1} (-1)^{j-1} \sum_{1 \leq i_1 < .. < i_j \leq 6} \frac{(6-j)^{10}}{6^{10}} = \sum^6_{j=1} (-1)^{j-1} \binom{6}{j} \frac{(6-j)^{10}}{6^{10}}
\end{align*}

\begin{theorem}{}{}
Es sei $(\Omega, \mathcal A, P)$ ein Wahrscheinlichkeitsraum, $(A_n)_n$ eine Folge von Ereignissen in $\mathcal A$. Dann gilt
\begin{enumerate}
\item $A_n \uparrow A$, d.h. $A_1 \subset A_2 \subset ... $ und $\bigcup_n A_n = A$, so folgt
\begin{align*}
	\lim_{n \to \infty}P(A_n) = P(A) \tag{Stetigkeit von unten}
\end{align*}

\item $A_n \downarrow A$, d.h. $A_1 \supset A_2 \supset ...$ und $\bigcap_n A_n = A$, so folgt
\begin{align*}
	\lim_{n \to \infty} P(A_n) = P(A) \tag{\text{Stetigkeit von oben}}
\end{align*}
\end{enumerate}
\end{theorem}

\begin{proof}
Ohne Beweis.
\end{proof}


\subsection{Diskrete Wahrscheinlichkeitsräume}
Ist $\Omega$ diskret, so ist Teilmenge $A \subset \Omega$ abzählbar. Also
\[
	A = \{ w_1, w_2, ... \} \quad \text{ und damit } \quad A = \bigcup_{n} \{ w_n \}
\]

Ist $\mathcal A$ eine $\sigma$-Algebra in $\Omega$  mit $\{ w \} \in \mathcal A, \forall w \in \Omega$ und $P$ ein Wahrscheinlichkeitsmaß auf $(\Omega, \mathcal A)$, so folgt
\[
	P(A) = \sum_n P(\{ w_n \}) = \sum_{w \in A} P(\{ w \})
\]
d.h. $P$ ist eindeutig bestimmt durch die Wahrscheinlichkeiten $p(w) \coloneqq P(\{ w \})$. $p$ definiert eine Funktion $p: \Omega \to [0,1]$ heißt \vip{Zähldichte}, d.h. eine Funktion $p: \Omega \to [0,1]$ mit $\sum_{w \in \Omega} p(w) = 1$. Ist umgekehrt $p$ eine Zähldichte, so definiert
\begin{align}\label{1.2:las}
	P(A) \coloneqq \sum_{w \in A}p(w), \quad A \in \mathcal P(\Omega)
\end{align}
ein Wahrscheinlichkeitsmaß auf \emph{ganz} $P(\Omega).$

\begin{important}{}{}
Im diskreten Fall ist $\mathcal P(\Omega)$ die natürliche $\sigma$-Algebra und daher ersetzt man im diskreten Falle $(\Omega, \mathcal P(\Omega), P)$ einfach durch $(\Omega, p)$ (siehe Satz 1.10 im Skript).
\end{important}

Satz 1.10 besagt also, dass für diskrete Ergebnisräume durch \eqref{1.2:las} eine 1-1 Beziehung zwischen Zähldichten und Wahrscheinlichkeitsmaßen auf $P(\Omega)$ gegeben ist.

\begin{example}{n-maliges Werfen einer fairen Münze}{}
Sei $\Omega = \{ (i_1,...,i_n) : i_j \in \{ 0,1 \}, 1 \leq j \leq n \}, |\Omega| = 2^n$. Damit $p(w) = \frac{1}{|\Omega|} = \frac{1}{2^n}$.

Für $k \in \{ 1,..., n \}$ gilt dann beispielsweise:
\begin{align*}
P(\text{im k-ten Münzwurf "Kopf" werfen}) = \frac{|\{ (i_1,...,i_{k-1}, 1 , i_{k+1}, ..., i_n) : i_j = 0,1 \}|}{|\Omega|} = \frac{2^{n-1}}{2^n} = \frac{1}{2}
\end{align*}
\end{example}

\end{document}
